\documentclass{MainStyle}

\usepackage{amsthm, amsfonts, amsmath, amssymb, quiver, mathrsfs, newclude, tikz-cd, ctex}

% Customise href Colours.
\usepackage[colorlinks = true,
            linkcolor = blue,
            urlcolor  = blue,
            citecolor = blue,
            anchorcolor = blue]{hyperref}

\newcommand{\changeurlcolor}[1]{\hypersetup{urlcolor=#1}}       

\newcommand*{\name}{张陈成}
\newcommand*{\id}{023071910029}
\newcommand*{\course}{三角范畴抄书笔记}
\newcommand*{\assignment}{三角范畴简介}

\theoremstyle{definition}
\newtheorem{example}{例}

\theoremstyle{definition}
\newtheorem{slogan}{原旨}

\theoremstyle{definition}
\newtheorem{definition}{定义}

\theoremstyle{definition}
\newtheorem{proposition}{命题}

\theoremstyle{definition}
\newtheorem{problem}{问题}

\theoremstyle{definition}
\newtheorem{assumption}{假定}

\theoremstyle{definition}
\newtheorem{theorem}{定理}

\theoremstyle{remark}
\newtheorem{remark}{注}

\theoremstyle{remark}
\newtheorem{lemma}{引理}
\allowdisplaybreaks

\begin{document}
\maketitle
\tableofcontents

\section{预三角范畴}

\begin{slogan}
    以下谈论的范畴都是本质小的. 换言之, $\mathsf{Ob}$ 在同构下的等价类构成集合.
\end{slogan}

\begin{definition}[加法范畴的自等价]
    称 $T:\mathcal C\to \mathcal C$ 为范畴 $\mathcal C$ 到自身的范畴等价, 当且仅当
    \begin{enumerate}
        \item (全) $T:\mathrm{Hom}_{\mathcal C}(X,Y)\to \mathrm{Hom}_{\mathcal C}(TX,TY)$ 对一切 $X,Y\in \mathsf{Ob}(\mathcal C)$ 满.
        \item (忠实) $T:\mathrm{Hom}_{\mathcal C}(X,Y)\to \mathrm{Hom}_{\mathcal C}(TX,TY)$ 对一切 $X,Y\in \mathsf{Ob}(\mathcal C)$ 单.
        \item (稠密/本质满) 对任意 $X'\in \mathsf{Ob}(C)$, 总存在 $X\in \mathrm{Ob}(C)$ 使得 $TX\simeq X'$.
    \end{enumerate}
\end{definition}

\begin{remark}
    等价地, 存在函子 $S:\mathcal C\to \mathcal C$ 与函子的自然同构 $TS\simeq \mathrm{id}_{\mathcal C}\simeq ST$. 不妨直接假定 $S^{-1}=T$\footnote{$\textcolor{red}{\text{待补充???}}$}.
\end{remark}

\begin{definition}[三角与三角射]
    称 $(\mathcal C, T)$ 中的三角为六元组 $(X,Y,Z,u,v,w)$, 即如下态射序列
    % https://q.uiver.app/#q=WzAsNCxbMCwwLCJYIl0sWzEsMCwiWSJdLFsyLDAsIloiXSxbMywwLCJUWCJdLFswLDEsInUiXSxbMSwyLCJ2Il0sWzIsMywidyJdXQ==
    \[\begin{tikzcd}
            X & Y & Z & TX
            \arrow["u", from=1-1, to=1-2]
            \arrow["v", from=1-2, to=1-3]
            \arrow["w", from=1-3, to=1-4]
        \end{tikzcd}.\]
    三角间的态射(下称``三角射'')对应态射范畴的三角. 即, 使得下图交换的三元组 $(f,g,h)$.
    % https://q.uiver.app/#q=WzAsOCxbMCwwLCJYIl0sWzEsMCwiWSJdLFsyLDAsIloiXSxbMywwLCJUWCJdLFswLDEsIlgnIl0sWzEsMSwiWSciXSxbMiwxLCJaJyJdLFszLDEsIlRYJyJdLFswLDEsInUiXSxbMSwyLCJ2Il0sWzIsMywidyJdLFswLDQsImYnIiwyXSxbMSw1LCJnJyIsMl0sWzIsNiwiaCciLDJdLFszLDcsIlRmJyJdLFs0LDUsInUnIiwyXSxbNSw2LCJ2JyIsMl0sWzYsNywidyciLDJdXQ==
    \[\begin{tikzcd}
            X & Y & Z & TX \\
            {X'} & {Y'} & {Z'} & {TX'}
            \arrow["u", from=1-1, to=1-2]
            \arrow["v", from=1-2, to=1-3]
            \arrow["w", from=1-3, to=1-4]
            \arrow["{f'}"', from=1-1, to=2-1]
            \arrow["{g'}"', from=1-2, to=2-2]
            \arrow["{h'}"', from=1-3, to=2-3]
            \arrow["{Tf'}", from=1-4, to=2-4]
            \arrow["{u'}"', from=2-1, to=2-2]
            \arrow["{v'}"', from=2-2, to=2-3]
            \arrow["{w'}"', from=2-3, to=2-4]
        \end{tikzcd}.\]
\end{definition}

\begin{definition}[三角同构]
    若三角间的某态射有左逆及右逆(从而左右逆相等), 则称两三角同构.
\end{definition}

\begin{example}
    三角 $(X,Y,Z,u,v,w)$ 与 $(X,Y,Z,\varepsilon_1u,\varepsilon_2v,\varepsilon_3w)$ 同构. 其中 $\varepsilon_i\in \{\pm1\}$, $\varepsilon_1\varepsilon_2\varepsilon_3=1$.
\end{example}

\begin{definition}[预三角范畴]
    给定范畴 $\mathcal C $, 范畴自同构函子 $T:\mathcal C\to \mathcal C$, 以及某些三角组成的类 $\mathcal E$. 称 $(\mathcal C,T,\mathcal E$ 为预三角范畴, 若满足以下命题.
    \begin{enumerate}
        \item $\mathcal E$ 中存在形如以下的三角.
              \begin{enumerate}
                  \item $\begin{tikzcd}
                                X & X & 0 & TX
                                \arrow["{\mathrm{id}_X}", from=1-1, to=1-2]
                                \arrow["0", from=1-2, to=1-3]
                                \arrow["0", from=1-3, to=1-4]
                            \end{tikzcd}$ 为三角.
                  \item 任意 $\begin{tikzcd}
                                X & Y
                                \arrow["f", from=1-1, to=1-2]
                            \end{tikzcd}$ 可嵌入形如 $\begin{tikzcd}
                                X & Y & Z & TX
                                \arrow["f", from=1-1, to=1-2]
                                \arrow["g", from=1-2, to=1-3]
                                \arrow["{h'}", from=1-3, to=1-4]
                            \end{tikzcd}$ 的三角.
                  \item 对任意 $(X,Y,Z, u,v,w)\in \mathcal E$, 若 $\mathcal C$ 中存在同构的三角 $(X,Y,Z, u,v,w)\simeq (X',Y',Z', u',v',w')$, 则后者也在 $\mathcal E$ 中.
              \end{enumerate}
        \item $\mathcal E$ 中三角的顺时针旋转也在 $\mathcal E$ 中. 此处顺时针旋转是指
              % https://q.uiver.app/#q=WzAsOCxbMSwwLCJZIl0sWzIsMCwiWiJdLFszLDAsIlRYXFxCaWddIl0sWzAsMCwiXFxCaWdbWCJdLFs0LDAsIlxcQmlnW1kiXSxbNSwwLCJaIl0sWzYsMCwiVFgiXSxbNywwLCJUWVxcQmlnXSJdLFszLDAsInUiXSxbMCwxLCJ2Il0sWzEsMiwidyJdLFs0LDUsInYiXSxbNSw2LCJ3Il0sWzYsNywiLVR1Il0sWzIsNCwiIiwwLHsibGV2ZWwiOjJ9XV0=
              \[\begin{tikzcd}
                      {\Big[X} & Y & Z & {TX\Big]} & {\Big[Y} & Z & TX & {TY\Big]}
                      \arrow["u", from=1-1, to=1-2]
                      \arrow["v", from=1-2, to=1-3]
                      \arrow["w", from=1-3, to=1-4]
                      \arrow["v", from=1-5, to=1-6]
                      \arrow["w", from=1-6, to=1-7]
                      \arrow["{-Tu}", from=1-7, to=1-8]
                      \arrow[Rightarrow, from=1-4, to=1-5]
                  \end{tikzcd}.\]
        \item 态射范畴的态射也可补全作三角. 换言之, 交换图 $\begin{tikzcd}
                      X & Y \\
                      {X'} & {Y'}
                      \arrow["u", from=1-1, to=1-2]
                      \arrow["{u'}"', from=2-1, to=2-2]
                      \arrow["f"', from=1-1, to=2-1]
                      \arrow["g", from=1-2, to=2-2]
                  \end{tikzcd}$ 总能被补全作交换图
              % https://q.uiver.app/#q=WzAsOCxbMCwwLCJYIl0sWzAsMSwiWCciXSxbMSwwLCJZIl0sWzEsMSwiWSciXSxbMiwwLCJaIl0sWzMsMCwiVFgiXSxbMiwxLCJaJyJdLFszLDEsIlRYJyJdLFswLDIsInUiXSxbMSwzLCJ1JyIsMl0sWzAsMSwiZiIsMl0sWzIsMywiZyJdLFsyLDQsInYiXSxbNCw1LCJ3Il0sWzMsNiwidiIsMl0sWzYsNywidyciLDJdLFs0LDYsImgiLDAseyJzdHlsZSI6eyJib2R5Ijp7Im5hbWUiOiJkYXNoZWQifX19XSxbNSw3LCJUZiJdXQ==
              \[\begin{tikzcd}
                      X & Y & Z & TX \\
                      {X'} & {Y'} & {Z'} & {TX'}
                      \arrow["u", from=1-1, to=1-2]
                      \arrow["{u'}"', from=2-1, to=2-2]
                      \arrow["f"', from=1-1, to=2-1]
                      \arrow["g", from=1-2, to=2-2]
                      \arrow["v", from=1-2, to=1-3]
                      \arrow["w", from=1-3, to=1-4]
                      \arrow["v"', from=2-2, to=2-3]
                      \arrow["{w'}"', from=2-3, to=2-4]
                      \arrow["h", dashed, from=1-3, to=2-3]
                      \arrow["Tf", from=1-4, to=2-4]
                  \end{tikzcd}.\]
    \end{enumerate}
    称 $\mathcal E$ 中的三角为``好三角'', 也可想象之为``正合列''.
\end{definition}

\begin{remark}\label{proof}
    以上定义中条件可改进如下
    \begin{enumerate}
        \item 1-(b) 中嵌入位置是任意的,
        \item 1-(b) 中嵌入的三角在同构意义下唯一.
        \item 2 是充要的, 可定义``顺时针旋转''的逆变换为``逆时针旋转''.
        \item 3 中 $\{f,g,h\}$ 中任意两者的存在性推出第三者的存在性(不必唯一)\footnote{$\textcolor{red}{\text{该条结论位置不妥, 往后调整之.}}$}.
    \end{enumerate}
    往后依次证明之.
\end{remark}

\begin{proposition}
    注 \ref{proof} 中的第三条成立.
    \begin{proof}
        任意三角在 $T^{-1}$ 作用下得到同构的三角, 此处 $T^{-2}$ 的逆变换为六次顺时针旋转. 因此, 好三角的六次逆时针旋转仍为好三角. 验证知逆时针旋转为 $T^{-2}$ 与五次顺时针旋转之/复合. 反之, 若逆时针变换定义, 则定义顺时针变换为 $T^2$ 与五次逆时针旋转之复合.
    \end{proof}
\end{proposition}

\begin{proposition}
    注 \ref{proof} 中的第一条成立.
    \begin{proof}
        依定义, 存在三角
        % https://q.uiver.app/#q=WzAsNCxbMCwwLCJUXnstMX1YIl0sWzEsMCwiVF57LTF9WSJdLFsyLDAsIlReey0xfVoiXSxbMywwLCJYIl0sWzAsMSwiLVReey0xfXUiXSxbMSwyLCItVF57LTF9diJdLFsyLDMsIi1UXnstMX13Il1d
        \[\begin{tikzcd}[column sep=large]
                {T^{-1}X} & {T^{-1}Y} & {T^{-1}Z} & X
                \arrow["{-T^{-1}u}", from=1-1, to=1-2]
                \arrow["{-T^{-1}v}", from=1-2, to=1-3]
                \arrow["{-T^{-1}w}", from=1-3, to=1-4]
            \end{tikzcd}.\]
        考虑顺时针旋转, 得
        % https://q.uiver.app/#q=WzAsNCxbMCwwLCJUXnstMX1ZIl0sWzEsMCwiVF57LTF9WiJdLFsyLDAsIlgiXSxbMywwLCJZIl0sWzAsMSwiLVReey0xfXYiXSxbMSwyLCItVF57LTF9dyJdLFsyLDMsInUiXV0=
        \[\begin{tikzcd}[column sep=large]
                {T^{-1}Y} & {T^{-1}Z} & X & Y
                \arrow["{-T^{-1}v}", from=1-1, to=1-2]
                \arrow["{-T^{-1}w}", from=1-2, to=1-3]
                \arrow["u", from=1-3, to=1-4]
            \end{tikzcd}. \]
        再次顺时针旋转, 遂有
        % https://q.uiver.app/#q=WzAsNCxbMCwwLCJUXnstMX1aIl0sWzEsMCwiWCJdLFsyLDAsIlkiXSxbMywwLCJaIl0sWzAsMSwiLVReey0xfXciXSxbMSwyLCJ1Il0sWzIsMywidiJdXQ==
        \[\begin{tikzcd}[column sep=large]
                {T^{-1}Z} & X & Y & Z
                \arrow["{-T^{-1}w}", from=1-1, to=1-2]
                \arrow["u", from=1-2, to=1-3]
                \arrow["v", from=1-3, to=1-4]
            \end{tikzcd}.\]
    \end{proof}
\end{proposition}

\begin{proposition}
    ``好三角''中相继映射之复合为 $0$.
    \begin{proof}
        不妨取好三角
        % https://q.uiver.app/#q=WzAsNCxbMCwwLCJYIl0sWzEsMCwiWSJdLFsyLDAsIloiXSxbMywwLCJUWCJdLFswLDEsInUiXSxbMSwyLCJ2Il0sWzIsMywidyJdXQ==
        \[\begin{tikzcd}
                X & Y & Z & TX
                \arrow["u", from=1-1, to=1-2]
                \arrow["v", from=1-2, to=1-3]
                \arrow["w", from=1-3, to=1-4]
            \end{tikzcd}.\]
        将原三角补全作以下态射范畴的三角
        % https://q.uiver.app/#q=WzAsOCxbMCwxLCJYIl0sWzEsMSwiWSJdLFsyLDEsIloiXSxbMywxLCJUWCJdLFswLDAsIlgiXSxbMSwwLCJYIl0sWzIsMCwiMCJdLFszLDAsIlRYIl0sWzAsMSwidSJdLFsxLDIsInYiXSxbMiwzLCJ3Il0sWzcsMywiVHUiXSxbNiw3XSxbNiwyLCIiLDIseyJzdHlsZSI6eyJib2R5Ijp7Im5hbWUiOiJkYXNoZWQifX19XSxbNCwwLCIiLDEseyJsZXZlbCI6Miwic3R5bGUiOnsiaGVhZCI6eyJuYW1lIjoibm9uZSJ9fX1dLFs1LDZdLFs0LDUsIiIsMix7ImxldmVsIjoyLCJzdHlsZSI6eyJoZWFkIjp7Im5hbWUiOiJub25lIn19fV0sWzUsMSwidSIsMl1d
        \[\begin{tikzcd}
                X & X & 0 & TX \\
                X & Y & Z & TX
                \arrow["u", from=2-1, to=2-2]
                \arrow["v", from=2-2, to=2-3]
                \arrow["w", from=2-3, to=2-4]
                \arrow["Tu", from=1-4, to=2-4]
                \arrow[from=1-3, to=1-4]
                \arrow[dashed, from=1-3, to=2-3]
                \arrow[Rightarrow, no head, from=1-1, to=2-1]
                \arrow[from=1-2, to=1-3]
                \arrow[Rightarrow, no head, from=1-1, to=1-2]
                \arrow["u"', from=1-2, to=2-2]
            \end{tikzcd}.\]
        从而 $vu=0$. 考虑一次顺时针旋转, 则 $wv=0$.
    \end{proof}
\end{proposition}

\section{同调核与同调余核}

\begin{definition}[同调核, 同调余核]
    预三角范畴中, 对好三角导出的映射链 、\[\begin{tikzcd}
            \cdots & {T^{-1}Z} & X & Y & Z & TX & TY & \cdots
            \arrow[from=1-1, to=1-2]
            \arrow["u", from=1-3, to=1-4]
            \arrow["v", from=1-4, to=1-5]
            \arrow["w", from=1-5, to=1-6]
            \arrow["{-T^{-1}w}", from=1-2, to=1-3]
            \arrow["{-Tu}", from=1-6, to=1-7]
            \arrow[from=1-7, to=1-8]
        \end{tikzcd},\]
    定义前一态射是后一态射的同调核, 后一态射是前一态射的同调余核.
\end{definition}

\begin{remark}
    依照预三角范畴之定义及若干推论, 不难有以下结论.
    \begin{enumerate}
        \item 恒等映射的同调核与同调余核均为 $0$;
        \item 任意态射均有同调核与同调余核, 且在同构意义下唯一;
        \item 同调核的同调余核即同调余核的同调核, 亦即映射本身.
    \end{enumerate}
\end{remark}

\begin{proposition}[同调余核的分解原理]
    给定好三角 $(X,Y,Z,u,v,w)$, $Y\overset\alpha\longrightarrow M$ 被 $v$ 分解当且仅当 $\alpha u=0$.
    \begin{proof}
        若存在 $\varphi$ 使得 $\varphi v=\alpha$, 考虑交换图 % https://q.uiver.app/#q=WzAsOCxbMCwwLCJYIl0sWzEsMCwiWSJdLFsyLDAsIloiXSxbMywwLCJUWCJdLFsxLDEsIk0iXSxbMiwxLCJNIl0sWzMsMSwiMCJdLFswLDEsIjAiXSxbMCwxLCJ1Il0sWzEsMiwidiJdLFsyLDMsInciXSxbMSw0LCJcXGFscGhhIiwyXSxbNCw1LCIiLDAseyJsZXZlbCI6Miwic3R5bGUiOnsiaGVhZCI6eyJuYW1lIjoibm9uZSJ9fX1dLFs3LDRdLFs1LDZdLFsyLDUsIlxcdmFycGhpIiwyLHsic3R5bGUiOnsiYm9keSI6eyJuYW1lIjoiZGFzaGVkIn19fV1d
        \[\begin{tikzcd}
                X & Y & Z & TX \\
                0 & M & M & 0
                \arrow["u", from=1-1, to=1-2]
                \arrow["v", from=1-2, to=1-3]
                \arrow["w", from=1-3, to=1-4]
                \arrow["\alpha"', from=1-2, to=2-2]
                \arrow[Rightarrow, no head, from=2-2, to=2-3]
                \arrow[from=2-1, to=2-2]
                \arrow[from=2-3, to=2-4]
                \arrow["\varphi"', dashed, from=1-3, to=2-3]
            \end{tikzcd}.\]
        根据``二推三''补全左侧正方形, 得 $\alpha u=0$. 反之, 若 $\alpha u=0$, 则有正合列间的交换图% https://q.uiver.app/#q=WzAsOCxbMCwwLCJYIl0sWzEsMCwiWSJdLFsyLDAsIloiXSxbMywwLCJUWCJdLFsxLDEsIk0iXSxbMiwxLCJNIl0sWzMsMSwiMCJdLFswLDEsIjAiXSxbMCwxLCJ1Il0sWzEsMiwidiJdLFsyLDMsInciXSxbMSw0LCJcXGFscGhhIiwyXSxbNCw1LCIiLDAseyJsZXZlbCI6Miwic3R5bGUiOnsiaGVhZCI6eyJuYW1lIjoibm9uZSJ9fX1dLFs3LDRdLFs1LDZdLFswLDddLFszLDZdLFsyLDUsIlxcdmFycGhpICIsMCx7InN0eWxlIjp7ImJvZHkiOnsibmFtZSI6ImRhc2hlZCJ9fX1dXQ==
        \[\begin{tikzcd}
                X & Y & Z & TX \\
                0 & M & M & 0
                \arrow["u", from=1-1, to=1-2]
                \arrow["v", from=1-2, to=1-3]
                \arrow["w", from=1-3, to=1-4]
                \arrow["\alpha"', from=1-2, to=2-2]
                \arrow[Rightarrow, no head, from=2-2, to=2-3]
                \arrow[from=2-1, to=2-2]
                \arrow[from=2-3, to=2-4]
                \arrow[from=1-1, to=2-1]
                \arrow[from=1-4, to=2-4]
                \arrow["{\varphi }", dashed, from=1-3, to=2-3]
            \end{tikzcd}.\]
        依照``二推三''补全的 $\varphi$ 给出分解 $\varphi v=\alpha$.
    \end{proof}
\end{proposition}

\begin{remark}
    等价地, 好三角的交换图 $\begin{tikzcd}
            & N \\
            X & Y & Z & TX
            \arrow["u"', from=2-1, to=2-2]
            \arrow["v"', from=2-2, to=2-3]
            \arrow["w"', from=2-3, to=2-4]
            \arrow["\psi"', from=1-2, to=2-2]
        \end{tikzcd}$ 中, $\psi$ 被 $u$ 分解当且仅当 $v\psi =0$.
\end{remark}

\begin{definition}[上同调函子]
    称预三角范畴 $\mathcal C$ 到 Abel 范畴 $\mathcal A$ 上的加法函子 $H$ 是上同调函子, 当且仅当 $H$ 在好三角上的作用导出长正合列
    % https://q.uiver.app/#q=WzAsNyxbMiwwLCJIKFgpIl0sWzMsMCwiSChZKSJdLFs0LDAsIkgoWikiXSxbNSwwLCJIKFRYKSJdLFsxLDAsIkgoVF57LTF9WikiXSxbMCwwLCJcXGNkb3RzIl0sWzYsMCwiXFxjZG90cyJdLFszLDYsIkgoLVR1KSJdLFsyLDMsIkgodykiXSxbMSwyLCJIKHYpIl0sWzAsMSwiSCh1KSJdLFs0LDAsIkgoLVReey0xfXcpIl0sWzUsNCwiSCgtVF57LTF9dikiXV0=
    \[\begin{tikzcd}[column sep=large]
            \cdots & {H(T^{-1}Z)} & {H(X)} & {H(Y)} & {H(Z)} & {H(TX)} & \cdots
            \arrow["{H(-Tu)}", from=1-6, to=1-7]
            \arrow["{H(w)}", from=1-5, to=1-6]
            \arrow["{H(v)}", from=1-4, to=1-5]
            \arrow["{H(u)}", from=1-3, to=1-4]
            \arrow["{H(-T^{-1}w)}", from=1-2, to=1-3]
            \arrow["{H(-T^{-1}v)}", from=1-1, to=1-2]
        \end{tikzcd}.\]
    $\mathcal C$ 到 $\mathcal A$ 的反变上同调函子等价于 $\mathcal C^{\mathrm{op}}$ 到 $\mathcal A$ 的上同调函子.
\end{definition}

\begin{example}
    对任意 $M\in \mathsf{Ob}(\mathcal C)$, 函子 $\mathrm{Hom}_{\mathcal C}(M,-)$ 与 $\mathrm{Hom}_{\mathcal C}(-,M)$ 均是上同调函子. \par
    对前者, 好三角 $\begin{tikzcd}
            X & Y & Z & TX
            \arrow["u", from=1-1, to=1-2]
            \arrow["v", from=1-2, to=1-3]
            \arrow["w", from=1-3, to=1-4]
        \end{tikzcd}$ 给出链复形(任意 $d\in\mathbb Z$)
    % https://q.uiver.app/#q=WzAsMyxbMCwwLCJcXG1hdGhybXtIb219X3tcXG1hdGhjYWwgQ30oTSxYKSJdLFsyLDAsIlxcbWF0aHJte0hvbX1fe1xcbWF0aGNhbCBDfShNLFkpIl0sWzQsMCwiXFxtYXRocm17SG9tfV97XFxtYXRoY2FsIEN9KE0sWikiXSxbMCwxLCJcXG1hdGhybXtIb219X3tcXG1hdGhjYWwgQ30oTSx1KSJdLFsxLDIsIlxcbWF0aHJte0hvbX1fe1xcbWF0aGNhbCBDfShNLHYpIl1d
    \[\begin{tikzcd}
            {\mathrm{Hom}_{\mathcal C}(M,T^dX)} && {\mathrm{Hom}_{\mathcal C}(M,T^dY)} && {\mathrm{Hom}_{\mathcal C}(M,T^dZ)}
            \arrow["{\mathrm{Hom}_{\mathcal C}(M,T^du)}", from=1-1, to=1-3]
            \arrow["{\mathrm{Hom}_{\mathcal C}(M,T^dv)}", from=1-3, to=1-5]
        \end{tikzcd}.\]
    下证明 $T^dY$ 处正合性. 对任意 $g\in \ker \mathrm{Hom}_{\mathcal C}(M,T^dv)$, 总存在 $f$ 使得下图交换
    % https://q.uiver.app/#q=WzAsOCxbMCwxLCJZIl0sWzEsMSwiWiJdLFsyLDEsIlRYIl0sWzMsMSwiVFkiXSxbMCwwLCJUXnstZH1NIl0sWzIsMCwiVF57MS1kfU0iXSxbMywwLCJUXnsxLWR9TSJdLFsxLDAsIjAiXSxbMCwxLCJ2Il0sWzEsMiwidyJdLFsyLDMsIi1UdSJdLFs1LDYsIlxcbWF0aHJte2lkfSJdLFs0LDddLFs3LDVdLFs0LDAsIlReey1kfWciLDJdLFs3LDFdLFs2LDMsIlReezEtZH1nIl0sWzUsMiwiZiIsMCx7InN0eWxlIjp7ImJvZHkiOnsibmFtZSI6ImRhc2hlZCJ9fX1dXQ==
    \[\begin{tikzcd}
            {T^{-d}M} & 0 & {T^{1-d}M} & {T^{1-d}M} \\
            Y & Z & TX & TY
            \arrow["v", from=2-1, to=2-2]
            \arrow["w", from=2-2, to=2-3]
            \arrow["{-Tu}", from=2-3, to=2-4]
            \arrow["{\mathrm{id}}", from=1-3, to=1-4]
            \arrow[from=1-1, to=1-2]
            \arrow[from=1-2, to=1-3]
            \arrow["{T^{-d}g}"', from=1-1, to=2-1]
            \arrow[from=1-2, to=2-2]
            \arrow["{T^{1-d}g}", from=1-4, to=2-4]
            \arrow["f", dashed, from=1-3, to=2-3]
        \end{tikzcd}.\]
    此时 $T^{1-d}g=-(Tu)f$. 故 $g=T^{d-1}(-(Tu)f)\in\mathrm{im\,} \mathrm{Hom}_{\mathcal C}(M,T^d u)$. 同理, $\mathrm{Hom}_{\mathcal C}(-,M)$ 是反变正合的.
\end{example}

\begin{proposition}
    若好三角的态射中有两处映射为同构, 则第三处亦然. 这也直接证明了注 \ref{proof} 中的第二条.
    \begin{proof}
        考虑三角旋转, 不失一般性地设以下交换图中 $f$ 与 $g$ 是同构.
        % https://q.uiver.app/#q=WzAsOCxbMCwwLCJYIl0sWzAsMSwiWCciXSxbMSwwLCJZIl0sWzEsMSwiWSciXSxbMiwwLCJaIl0sWzMsMCwiVFgiXSxbMiwxLCJaJyJdLFszLDEsIlRYJyJdLFswLDIsInUiXSxbMSwzLCJ1JyIsMl0sWzAsMSwiZiIsMl0sWzIsMywiZyJdLFsyLDQsInYiXSxbNCw1LCJ3Il0sWzMsNiwidiciLDJdLFs2LDcsIncnIiwyXSxbNCw2LCJoIiwwLHsic3R5bGUiOnsiYm9keSI6eyJuYW1lIjoiZGFzaGVkIn19fV0sWzUsNywiVGYiXV0=
        \[\begin{tikzcd}
                X & Y & Z & TX \\
                {X'} & {Y'} & {Z'} & {TX'}
                \arrow["u", from=1-1, to=1-2]
                \arrow["{u'}"', from=2-1, to=2-2]
                \arrow["f"', from=1-1, to=2-1]
                \arrow["g", from=1-2, to=2-2]
                \arrow["v", from=1-2, to=1-3]
                \arrow["w", from=1-3, to=1-4]
                \arrow["{v'}"', from=2-2, to=2-3]
                \arrow["{w'}"', from=2-3, to=2-4]
                \arrow["h", dashed, from=1-3, to=2-3]
                \arrow["Tf", from=1-4, to=2-4]
            \end{tikzcd}.\]
        记 $h^M:\mathcal C\to \mathrm{Ab}, X\mapsto \mathrm{Hom}_{\mathcal C}(M,X)$, 则有正合列间的交换图
        % https://q.uiver.app/#q=WzAsMTAsWzAsMCwiaF9aKFgpIl0sWzAsMSwiaF9aKFgnKSJdLFsxLDAsImhfWihZKSJdLFsxLDEsImhfWihZJykiXSxbMiwwLCJoX1ooWikiXSxbMywwLCJoX1ooVFgpIl0sWzIsMSwiaF9aKFonKSJdLFszLDEsImhfWihUWCcpIl0sWzQsMCwiaF9aKFRZKSJdLFs0LDEsImhfWihUWScpIl0sWzIsMCwiaF9aKHUpIiwyXSxbMywxLCJoX1oodScpIl0sWzEsMCwiaF9aKGYpIl0sWzMsMiwiaF9aKGcpIiwyXSxbNCwyLCJoXloodikiLDJdLFs1LDQsImheWih3KSIsMl0sWzYsMywiaF9aKHYnKSJdLFs3LDYsImhfWih3JykiXSxbNiw0LCJoX1ooaCkiLDIseyJzdHlsZSI6eyJib2R5Ijp7Im5hbWUiOiJkYXNoZWQifX19XSxbNyw1LCJoX1ooVGYpIiwyXSxbOCw1LCJoX1ooLVR1KSIsMl0sWzksNywiaF9aKC1UdScpIl0sWzksOCwiaF9aKFRnKSIsMl1d
        \[\begin{tikzcd}[column sep=large]
                {h_Z(X)} & {h_Z(Y)} & {h_Z(Z)} & {h_Z(TX)} & {h_Z(TY)} \\
                {h_Z(X')} & {h_Z(Y')} & {h_Z(Z')} & {h_Z(TX')} & {h_Z(TY')}
                \arrow["{h_Z(u)}"', from=1-2, to=1-1]
                \arrow["{h_Z(u')}", from=2-2, to=2-1]
                \arrow["{h_Z(f)}", from=2-1, to=1-1]
                \arrow["{h_Z(g)}"', from=2-2, to=1-2]
                \arrow["{h^Z(v)}"', from=1-3, to=1-2]
                \arrow["{h^Z(w)}"', from=1-4, to=1-3]
                \arrow["{h_Z(v')}", from=2-3, to=2-2]
                \arrow["{h_Z(w')}", from=2-4, to=2-3]
                \arrow["{h_Z(h)}"', dashed, from=2-3, to=1-3]
                \arrow["{h_Z(Tf)}"', from=2-4, to=1-4]
                \arrow["{h_Z(-Tu)}"', from=1-5, to=1-4]
                \arrow["{h_Z(-Tu')}", from=2-5, to=2-4]
                \arrow["{h_Z(Tg)}"', from=2-5, to=1-5]
            \end{tikzcd}.\]
        此处 $\{h_Z(f),h_Z(g),h_Z(Tf),h_Z(Tg)\}$ 均为同构. 根据五引理, 中间处 $h_Z(h)$ 为同构. 显然存在 $h'\in \mathrm{Hom}_{\mathcal C}(Z',Z)$ 使得 $h'\circ h=\mathrm{id}_Z\in \mathrm{End}_{\mathcal C}(Z)$. 同理地, 将 $h_Z$ 换作 $h^{Z'}$ 可知 $h$ 有左逆与右逆, 从而 $h$ 与 $h'$ 为互逆的同构.
    \end{proof}
\end{proposition}

\begin{remark}
    仿照以上证明, 有``二推三''推论. 即, 若 $\{f,g,h\}$ 中任意两者为同构, 则第三者亦然.
\end{remark}

\begin{slogan}
    若无特殊说明, 默认范畴中的决出极限(余极限)的定向系统是小的.
\end{slogan}

\begin{definition}[下降]
    记 $\mathcal A$ 是容许极限的 Abel 范畴. 称预三角范畴到 Abel 范畴的同调函子 $H:\mathcal C\to \mathcal A$ 为下降, 若其保持积.
\end{definition}

\begin{example}
    对任意 $M\in \mathsf{Ob}(\mathcal C)$, 形如 $h^M$ 的同调函子均为下降.
\end{example}

\begin{definition}[预三角]
    预三角即在一切同调函子下正合的三角. 预三角包括好三角.
\end{definition}

\begin{proposition}
    预三角之积仍为预三角. 考虑任意下降即可.
\end{proposition}


\section{好三角的可裂性}

\begin{proposition}[直和保持好三角]
    给定预三角范畴 $\mathcal C$, 则好三角的有限直和仍是好三角. 若范畴允许某种无穷直和, 则无穷个好三角的该种无穷直和仍是好三角.
    \begin{proof}
        考虑以下交换图
        % https://q.uiver.app/#q=WzAsMTIsWzAsMCwiWCJdLFsyLDAsIlkiXSxbMywwLCJaIl0sWzQsMCwiVFgiXSxbMCwyLCJYJyJdLFsyLDIsIlknIl0sWzMsMiwiWiciXSxbNCwyLCJUWiciXSxbMCwxLCJYXFxvcGx1cyBYJyJdLFsyLDEsIllcXG9wbHVzIFknIl0sWzMsMSwiVyJdLFs0LDEsIlQoWFxcb3BsdXMgWCcpIl0sWzAsMSwidSJdLFsxLDIsInYiXSxbMiwzLCJ3Il0sWzQsNSwidSciLDJdLFs1LDYsInYnIiwyXSxbNiw3LCJ3JyIsMl0sWzAsOCwiXFxiaW5vbTEwIiwyXSxbNCw4LCJcXGJpbm9tMDEiXSxbMSw5LCJcXGJpbm9tMTAiXSxbNSw5LCJcXGJpbm9tMDEiLDJdLFsyLDEwLCJpIiwwLHsic3R5bGUiOnsiYm9keSI6eyJuYW1lIjoiZGFzaGVkIn19fV0sWzYsMTAsImoiLDIseyJzdHlsZSI6eyJib2R5Ijp7Im5hbWUiOiJkYXNoZWQifX19XSxbOCw5LCJ1XFxvcGx1cyB1JyJdLFs5LDEwLCJnIiwwLHsic3R5bGUiOnsiYm9keSI6eyJuYW1lIjoiZGFzaGVkIn19fV0sWzEwLDExLCJoIiwwLHsic3R5bGUiOnsiYm9keSI6eyJuYW1lIjoiZGFzaGVkIn19fV0sWzMsMTEsIlRcXGJpbm9tMTAiXSxbNywxMSwiVFxcYmlub20wMSIsMl1d
        \[\begin{tikzcd}
                X && Y & Z & TX \\
                {X\oplus X'} && {Y\oplus Y'} & W & {T(X\oplus X')} \\
                {X'} && {Y'} & {Z'} & {TZ'}
                \arrow["u", from=1-1, to=1-3]
                \arrow["v", from=1-3, to=1-4]
                \arrow["w", from=1-4, to=1-5]
                \arrow["{u'}"', from=3-1, to=3-3]
                \arrow["{v'}"', from=3-3, to=3-4]
                \arrow["{w'}"', from=3-4, to=3-5]
                \arrow["\binom10"', from=1-1, to=2-1]
                \arrow["\binom01", from=3-1, to=2-1]
                \arrow["\binom10", from=1-3, to=2-3]
                \arrow["\binom01"', from=3-3, to=2-3]
                \arrow["i", dashed, from=1-4, to=2-4]
                \arrow["j"', dashed, from=3-4, to=2-4]
                \arrow["{u\oplus u'}", from=2-1, to=2-3]
                \arrow["g", dashed, from=2-3, to=2-4]
                \arrow["h", dashed, from=2-4, to=2-5]
                \arrow["T\binom10", from=1-5, to=2-5]
                \arrow["T\binom01"', from=3-5, to=2-5]
            \end{tikzcd}.\]
        其中, $g$ 与 $h$ 为 $u\oplus u'$ 嵌入的某个好三角中的映射. 连接映射 $i$ 与 $j$ 由好三角间的同态给出. 依照``二推三''推论, 只需证明下交换图中 $(T\binom10,T\binom01)$ 为同构:
        % https://q.uiver.app/#q=WzAsOCxbMCwwLCJYXFxvcGx1cyBYJyJdLFswLDEsIlhcXG9wbHVzIFgnIl0sWzEsMCwiWVxcb3BsdXMgWSciXSxbMSwxLCJZXFxvcGx1cyBZJyJdLFsyLDAsIlpcXG9wbHVzIFonIl0sWzMsMCwiVFhcXG9wbHVzIFRYJyJdLFszLDEsIlQoWFxcb3BsdXMgWCcpIl0sWzIsMSwiVyJdLFswLDJdLFsyLDRdLFs0LDVdLFsxLDNdLFszLDcsImciLDIseyJzdHlsZSI6eyJib2R5Ijp7Im5hbWUiOiJkYXNoZWQifX19XSxbNyw2LCJoIiwyLHsic3R5bGUiOnsiYm9keSI6eyJuYW1lIjoiZGFzaGVkIn19fV0sWzAsMSwiIiwxLHsibGV2ZWwiOjIsInN0eWxlIjp7ImhlYWQiOnsibmFtZSI6Im5vbmUifX19XSxbMiwzLCIiLDEseyJsZXZlbCI6Miwic3R5bGUiOnsiaGVhZCI6eyJuYW1lIjoibm9uZSJ9fX1dLFs1LDYsIihUXFxiaW5vbTEwLFRcXGJpbm9tMDEpIiwyLHsic3R5bGUiOnsiYm9keSI6eyJuYW1lIjoiZGFzaGVkIn19fV0sWzQsNywiKGksaikiLDIseyJzdHlsZSI6eyJib2R5Ijp7Im5hbWUiOiJkYXNoZWQifX19XV0=
        \[\begin{tikzcd}
                {X\oplus X'} & {Y\oplus Y'} & {Z\oplus Z'} & {TX\oplus TX'} \\
                {X\oplus X'} & {Y\oplus Y'} & W & {T(X\oplus X')}
                \arrow[from=1-1, to=1-2]
                \arrow[from=1-2, to=1-3]
                \arrow[from=1-3, to=1-4]
                \arrow[from=2-1, to=2-2]
                \arrow["g"', dashed, from=2-2, to=2-3]
                \arrow["h"', dashed, from=2-3, to=2-4]
                \arrow[Rightarrow, no head, from=1-1, to=2-1]
                \arrow[Rightarrow, no head, from=1-2, to=2-2]
                \arrow["{(T\binom10,T\binom01)}"', dashed, from=1-4, to=2-4]
                \arrow["{(i,j)}"', dashed, from=1-3, to=2-3]
            \end{tikzcd}.\]
        这是显然的: 根据熟知结论, 加法范畴间的函子为加法函子当且仅当其保持有限余积. 对无穷情形, 由于 $T$ 是自同构, 从而与极限交换.
    \end{proof}
\end{proposition}

\begin{example}
    对预三角范畴 $\mathcal C$ 与任意 $X,Y\in \mathsf{Ob}(\mathcal C)$, 总有直和 $\begin{tikzcd}
            X & {X\oplus Y} & Y & TX
            \arrow["\binom10", from=1-1, to=1-2]
            \arrow["{(0,1)}", from=1-2, to=1-3]
            \arrow["0", from=1-3, to=1-4]
        \end{tikzcd}$.
\end{example}

\begin{definition}[可裂单/满]
    可裂单态射即存在左逆的态射, 可裂满态射即存在右逆的态射.
\end{definition}

\begin{proposition}
    给定好三角 $\begin{tikzcd}
            X & Y & Z & TX
            \arrow["u", from=1-1, to=1-2]
            \arrow["v", from=1-2, to=1-3]
            \arrow["w", from=1-3, to=1-4]
        \end{tikzcd}$, 则 $u$ 可裂单等价于 $v$ 可裂满, 亦等价于 $w=0$.
    \begin{proof}
        $w=0$ 时有以下交换图(三角同构)
        % https://q.uiver.app/#q=WzAsOCxbMCwwLCJYIl0sWzEsMCwiWFxcb3BsdXMgWiJdLFsyLDAsIloiXSxbMywwLCJUWCJdLFswLDEsIlgiXSxbMSwxLCJZIl0sWzIsMSwiWiJdLFszLDEsIlRYIl0sWzAsMSwiXFxiaW5vbTEwIl0sWzEsMiwiKDAsMSkiXSxbMiwzLCIwIl0sWzAsNCwiIiwyLHsibGV2ZWwiOjIsInN0eWxlIjp7ImhlYWQiOnsibmFtZSI6Im5vbmUifX19XSxbMiw2LCIiLDIseyJsZXZlbCI6Miwic3R5bGUiOnsiaGVhZCI6eyJuYW1lIjoibm9uZSJ9fX1dLFszLDcsIiIsMCx7ImxldmVsIjoyLCJzdHlsZSI6eyJoZWFkIjp7Im5hbWUiOiJub25lIn19fV0sWzYsNywidyIsMl0sWzUsNiwidiIsMl0sWzQsNSwidSIsMl0sWzEsNSwiKFxcdmFycGhpLFxccHNpKSIsMCx7InN0eWxlIjp7ImJvZHkiOnsibmFtZSI6ImRhc2hlZCJ9fX1dXQ==
        \[\begin{tikzcd}
                X & {X\oplus Z} & Z & TX \\
                X & Y & Z & TX
                \arrow["\binom10", from=1-1, to=1-2]
                \arrow["{(0,1)}", from=1-2, to=1-3]
                \arrow["0", from=1-3, to=1-4]
                \arrow[Rightarrow, no head, from=1-1, to=2-1]
                \arrow[Rightarrow, no head, from=1-3, to=2-3]
                \arrow[Rightarrow, no head, from=1-4, to=2-4]
                \arrow["w"', from=2-3, to=2-4]
                \arrow["v"', from=2-2, to=2-3]
                \arrow["u"', from=2-1, to=2-2]
                \arrow["{(\varphi,\psi)}", dashed, from=1-2, to=2-2]
            \end{tikzcd}.\]
        依照交换图, $\varphi=u$, 且 $\psi$ 是 $v$ 的右逆. 反之, 有交换图(三角同构)
        % https://q.uiver.app/#q=WzAsOCxbMCwwLCJYIl0sWzEsMCwiWFxcb3BsdXMgWiJdLFsyLDAsIloiXSxbMywwLCJUWCJdLFswLDEsIlgiXSxbMSwxLCJZIl0sWzIsMSwiWiJdLFszLDEsIlRYIl0sWzAsMSwiXFxiaW5vbTEwIl0sWzEsMiwiKDAsMSkiXSxbMiwzLCIwIl0sWzAsNCwiIiwyLHsibGV2ZWwiOjIsInN0eWxlIjp7ImhlYWQiOnsibmFtZSI6Im5vbmUifX19XSxbMiw2LCIiLDIseyJsZXZlbCI6Miwic3R5bGUiOnsiaGVhZCI6eyJuYW1lIjoibm9uZSJ9fX1dLFszLDcsIiIsMCx7ImxldmVsIjoyLCJzdHlsZSI6eyJoZWFkIjp7Im5hbWUiOiJub25lIn19fV0sWzYsNywidyIsMl0sWzUsNiwidiIsMl0sWzQsNSwidSIsMl0sWzEsNSwiKHUsdl9yXnstMX0pIiwyXV0=
        \[\begin{tikzcd}
                X & {X\oplus Z} & Z & TX \\
                X & Y & Z & TX
                \arrow["\binom10", from=1-1, to=1-2]
                \arrow["{(0,1)}", from=1-2, to=1-3]
                \arrow["0", from=1-3, to=1-4]
                \arrow[Rightarrow, no head, from=1-1, to=2-1]
                \arrow[Rightarrow, no head, from=1-3, to=2-3]
                \arrow[Rightarrow, no head, from=1-4, to=2-4]
                \arrow["w"', from=2-3, to=2-4]
                \arrow["v"', from=2-2, to=2-3]
                \arrow["u"', from=2-1, to=2-2]
                \arrow["{(u,v_r^{-1})}"', from=1-2, to=2-2]
            \end{tikzcd}.\]
        其中 $v_r^{-1}$ 为 $v$ 的右逆, 从而只能有 $w=0$. 这表明 $w=0$ 与 $v$ 可裂满是等价的. \par
        $w=0$ 时亦有如下交换图(三角同构)
        % https://q.uiver.app/#q=WzAsOCxbMCwxLCJYIl0sWzEsMSwiWFxcb3BsdXMgWiJdLFsyLDEsIloiXSxbMywxLCJUWCJdLFswLDAsIlgiXSxbMSwwLCJZIl0sWzIsMCwiWiJdLFszLDAsIlRYIl0sWzAsMSwiXFxiaW5vbTEwIiwyXSxbMSwyLCIoMCwxKSIsMl0sWzIsMywiMCIsMl0sWzAsNCwiIiwyLHsibGV2ZWwiOjIsInN0eWxlIjp7ImhlYWQiOnsibmFtZSI6Im5vbmUifX19XSxbMiw2LCIiLDIseyJsZXZlbCI6Miwic3R5bGUiOnsiaGVhZCI6eyJuYW1lIjoibm9uZSJ9fX1dLFszLDcsIiIsMCx7ImxldmVsIjoyLCJzdHlsZSI6eyJoZWFkIjp7Im5hbWUiOiJub25lIn19fV0sWzYsNywidyJdLFs1LDYsInYiXSxbNCw1LCJ1Il0sWzUsMSwiKFxcYWxwaGEsXFxiZXRhKSIsMl1d
        \[\begin{tikzcd}
                X & Y & Z & TX \\
                X & {X\oplus Z} & Z & TX
                \arrow["\binom10"', from=2-1, to=2-2]
                \arrow["{(0,1)}"', from=2-2, to=2-3]
                \arrow["0"', from=2-3, to=2-4]
                \arrow[Rightarrow, no head, from=2-1, to=1-1]
                \arrow[Rightarrow, no head, from=2-3, to=1-3]
                \arrow[Rightarrow, no head, from=2-4, to=1-4]
                \arrow["w", from=1-3, to=1-4]
                \arrow["v", from=1-2, to=1-3]
                \arrow["u", from=1-1, to=1-2]
                \arrow["{(\alpha,\beta)}"', from=1-2, to=2-2]
            \end{tikzcd}.\]
        显然 $\beta =v$, $\alpha$ 是 $u$ 的左逆. 反之, 有交换图(三角同构)
        % https://q.uiver.app/#q=WzAsOCxbMCwxLCJYIl0sWzEsMSwiWFxcb3BsdXMgWiJdLFsyLDEsIloiXSxbMywxLCJUWCJdLFswLDAsIlgiXSxbMSwwLCJZIl0sWzIsMCwiWiJdLFszLDAsIlRYIl0sWzAsMSwiXFxiaW5vbTEwIiwyXSxbMSwyLCIoMCwxKSIsMl0sWzIsMywiMCIsMl0sWzAsNCwiIiwyLHsibGV2ZWwiOjIsInN0eWxlIjp7ImhlYWQiOnsibmFtZSI6Im5vbmUifX19XSxbMiw2LCIiLDIseyJsZXZlbCI6Miwic3R5bGUiOnsiaGVhZCI6eyJuYW1lIjoibm9uZSJ9fX1dLFszLDcsIiIsMCx7ImxldmVsIjoyLCJzdHlsZSI6eyJoZWFkIjp7Im5hbWUiOiJub25lIn19fV0sWzYsNywidyJdLFs1LDYsInYiXSxbNCw1LCJ1Il0sWzUsMSwiKHVfbF57LTF9LHYpIiwyXV0=
        \[\begin{tikzcd}
                X & Y & Z & TX \\
                X & {X\oplus Z} & Z & TX
                \arrow["\binom10"', from=2-1, to=2-2]
                \arrow["{(0,1)}"', from=2-2, to=2-3]
                \arrow["0"', from=2-3, to=2-4]
                \arrow[Rightarrow, no head, from=2-1, to=1-1]
                \arrow[Rightarrow, no head, from=2-3, to=1-3]
                \arrow[Rightarrow, no head, from=2-4, to=1-4]
                \arrow["w", from=1-3, to=1-4]
                \arrow["v", from=1-2, to=1-3]
                \arrow["u", from=1-1, to=1-2]
                \arrow["{(u_l^{-1},v)}"', from=1-2, to=2-2]
            \end{tikzcd}.\]
        从而只能有 $w=0$. 这表明 $w=0$ 与 $u$ 可裂单是等价的.
    \end{proof}
\end{proposition}

\begin{remark}
    特别地, 若 $X\overset u\longrightarrow Y$ 是同构, 则有好三角的同构 $(X,Y,Z,u,v,w)\simeq (X,Y,0,u,0,0)$.
\end{remark}

\begin{proposition}
    给定好三角 $\begin{tikzcd}
            X & Y & Z & TX
            \arrow["u", from=1-1, to=1-2]
            \arrow["v", from=1-2, to=1-3]
            \arrow["w", from=1-3, to=1-4]
        \end{tikzcd}$, 则 $u$ 可裂单等价于 $u$ 是单态射.
    \begin{proof}
        仅证明单态射可裂. 若 $u$ 单, 则根据好三角中相邻态射复合为零知 $w=0$. 此时存在 $\varphi$ 使得下图交换
        % https://q.uiver.app/#q=WzAsOCxbMCwwLCJYIl0sWzEsMCwiWSJdLFsyLDAsIloiXSxbMywwLCJUWCJdLFswLDEsIlgiXSxbMSwxLCJYIl0sWzIsMSwiMCJdLFszLDEsIlRYIl0sWzAsMSwidSJdLFsxLDIsInYiXSxbMiwzLCIwIl0sWzMsNywiIiwwLHsibGV2ZWwiOjIsInN0eWxlIjp7ImhlYWQiOnsibmFtZSI6Im5vbmUifX19XSxbNiw3XSxbMiw2XSxbNSw2XSxbNCw1LCIiLDIseyJsZXZlbCI6Miwic3R5bGUiOnsiaGVhZCI6eyJuYW1lIjoibm9uZSJ9fX1dLFswLDQsIiIsMix7ImxldmVsIjoyLCJzdHlsZSI6eyJoZWFkIjp7Im5hbWUiOiJub25lIn19fV0sWzEsNSwiXFx2YXJwaGkgIiwwLHsic3R5bGUiOnsiYm9keSI6eyJuYW1lIjoiZGFzaGVkIn19fV1d
        \[\begin{tikzcd}
                X & Y & Z & TX \\
                X & X & 0 & TX
                \arrow["u", from=1-1, to=1-2]
                \arrow["v", from=1-2, to=1-3]
                \arrow["0", from=1-3, to=1-4]
                \arrow[Rightarrow, no head, from=1-4, to=2-4]
                \arrow[from=2-3, to=2-4]
                \arrow[from=1-3, to=2-3]
                \arrow[from=2-2, to=2-3]
                \arrow[Rightarrow, no head, from=2-1, to=2-2]
                \arrow[Rightarrow, no head, from=1-1, to=2-1]
                \arrow["{\varphi }", dashed, from=1-2, to=2-2]
            \end{tikzcd}.\]
        此时 $\varphi$ 为 $u$ 的左逆, 因此 $u$ 可裂单.
    \end{proof}
\end{proposition}

\begin{remark}
    同理, 好三角中的满态射与可裂满等价. 从而好三角中以下条件等价:
    \begin{align*}
        u \text{ 单}\Longleftrightarrow u \text{ 可裂单}\Longleftrightarrow w=0\Longleftrightarrow v \text{ 满}\Longleftrightarrow v \text{ 可裂满}.
    \end{align*}
\end{remark}

\begin{proposition}[``二推三''的唯一性条件]\label{unique 2->3}
    给定好三角的交换图(实线处)
    % https://q.uiver.app/#q=WzAsOCxbMCwwLCJYIl0sWzEsMCwiWSJdLFsyLDAsIloiXSxbMywwLCJUWCJdLFswLDEsIlgnIl0sWzEsMSwiWSciXSxbMiwxLCJaJyJdLFszLDEsIlRYJyJdLFswLDEsInUiXSxbMSwyLCJ2Il0sWzYsNywidyciLDJdLFs1LDYsInYnIiwyXSxbMSw1LCJnIiwyXSxbMCw0LCJmIiwyXSxbMyw3LCJUZiJdLFsyLDMsInciXSxbNCw1LCJ1JyIsMl0sWzIsNiwiaCIsMCx7InN0eWxlIjp7ImJvZHkiOnsibmFtZSI6ImRhc2hlZCJ9fX1dLFsyLDUsIiIsMSx7InN0eWxlIjp7ImJvZHkiOnsibmFtZSI6ImRvdHRlZCJ9fX1dLFszLDYsIiIsMSx7InN0eWxlIjp7ImJvZHkiOnsibmFtZSI6ImRvdHRlZCJ9fX1dXQ==
    \[\begin{tikzcd}
            X & Y & Z & TX \\
            {X'} & {Y'} & {Z'} & {TX'}
            \arrow["u", from=1-1, to=1-2]
            \arrow["v", from=1-2, to=1-3]
            \arrow["{w'}"', from=2-3, to=2-4]
            \arrow["{v'}"', from=2-2, to=2-3]
            \arrow["g"', from=1-2, to=2-2]
            \arrow["f"', from=1-1, to=2-1]
            \arrow["Tf", from=1-4, to=2-4]
            \arrow["w", from=1-3, to=1-4]
            \arrow["{u'}"', from=2-1, to=2-2]
            \arrow["h", dashed, from=1-3, to=2-3]
            \arrow[dotted, from=1-3, to=2-2]
            \arrow[dotted, from=1-4, to=2-3]
        \end{tikzcd}\]
    依定义知存在 $h$ 使得上图交换. 若 $\mathrm{Hom}_{\mathcal C}(TX,Z')=0$ 或 $\mathrm{Hom}_{\mathcal C}(Z,Y')=0$, 则 $h$ 唯一.
    \begin{proof}
        若存在 $h,h'\in \mathrm{Hom}_{\mathcal C}(Z,Z')$ 使得上图交换, 则 $(h-h')v=0=w'(h-h')$. 若 $\mathrm{Hom}_{\mathcal C}(TX,Z')=0$ 或 $\mathrm{Hom}_{\mathcal C}(Z,Y')=0$, 依分解定理知 $h-h'=0$.
    \end{proof}
\end{proposition}



\section{三角范畴}
\begin{definition}[三角范畴]\label{triangular}
    称预三角范畴为三角范畴, 若满足以下命题.
    \begin{itemize}
        \item 将 $\begin{tikzcd}
                      X & Y & Z
                      \arrow["u", from=1-1, to=1-2]
                      \arrow["v", from=1-2, to=1-3]
                      \arrow["vu"', curve={height=12pt}, from=1-1, to=1-3]
                  \end{tikzcd}$ 中映射 $\{u,v,uv\}$ 分别嵌入三个好三角, 则存在虚线处的好三角使得下图交换
              % https://q.uiver.app/#q=WzAsMTAsWzQsMCwiWCJdLFszLDEsIlkiXSxbMSwzLCJaJyJdLFs0LDIsIloiXSxbNSwzLCJYJyJdLFs2LDQsIlRZIl0sWzQsMywiWSciXSxbNCw0LCJUWCJdLFs2LDMsIlRaJyJdLFswLDQsIlRYIl0sWzAsMSwidSIsMix7ImNvbG91ciI6WzEyMCw2MCw2MF19LFsxMjAsNjAsNjAsMV1dLFsxLDIsImkiLDIseyJjb2xvdXIiOlsxMjAsNjAsNjBdfSxbMTIwLDYwLDYwLDFdXSxbMSwzLCJ2IiwyLHsiY29sb3VyIjpbMjQwLDYwLDYwXX0sWzI0MCw2MCw2MCwxXV0sWzMsNCwiaiIsMCx7ImNvbG91ciI6WzI0MCw2MCw2MF19LFsyNDAsNjAsNjAsMV1dLFs0LDUsImonIiwyLHsiY29sb3VyIjpbMjQwLDYwLDYwXX0sWzI0MCw2MCw2MCwxXV0sWzAsMywidnUiLDAseyJjb2xvdXIiOlswLDYwLDYwXX0sWzAsNjAsNjAsMV1dLFszLDYsImsiLDAseyJjb2xvdXIiOlswLDYwLDYwXX0sWzAsNjAsNjAsMV1dLFs2LDcsImsnIiwwLHsiY29sb3VyIjpbMCw2MCw2MF19LFswLDYwLDYwLDFdXSxbMiw2LCJmIiwwLHsic3R5bGUiOnsiYm9keSI6eyJuYW1lIjoiZGFzaGVkIn19fV0sWzYsNCwiZyIsMCx7InN0eWxlIjp7ImJvZHkiOnsibmFtZSI6ImRhc2hlZCJ9fX1dLFs0LDgsImgiLDAseyJzdHlsZSI6eyJib2R5Ijp7Im5hbWUiOiJkYXNoZWQifX19XSxbNSw4LCJUaSIsMl0sWzcsNSwiVHUiLDJdLFsyLDksImknIiwwLHsiY29sb3VyIjpbMTIwLDYwLDYwXX0sWzEyMCw2MCw2MCwxXV0sWzksNywiIiwwLHsibGV2ZWwiOjIsInN0eWxlIjp7ImhlYWQiOnsibmFtZSI6Im5vbmUifX19XV0=
              \[\begin{tikzcd}
                      &&&& X \\
                      &&& Y \\
                      &&&& Z \\
                      & {Z'} &&& {Y'} & {X'} & {TZ'} \\
                      TX &&&& TX && TY
                      \arrow["u"', color={rgb,255:red,92;green,214;blue,92}, from=1-5, to=2-4]
                      \arrow["i"', color={rgb,255:red,92;green,214;blue,92}, from=2-4, to=4-2]
                      \arrow["v"', color={rgb,255:red,92;green,92;blue,214}, from=2-4, to=3-5]
                      \arrow["j", color={rgb,255:red,92;green,92;blue,214}, from=3-5, to=4-6]
                      \arrow["{j'}"', color={rgb,255:red,92;green,92;blue,214}, from=4-6, to=5-7]
                      \arrow["vu", color={rgb,255:red,214;green,92;blue,92}, from=1-5, to=3-5]
                      \arrow["k", color={rgb,255:red,214;green,92;blue,92}, from=3-5, to=4-5]
                      \arrow["{k'}", color={rgb,255:red,214;green,92;blue,92}, from=4-5, to=5-5]
                      \arrow["f", dashed, from=4-2, to=4-5]
                      \arrow["g", dashed, from=4-5, to=4-6]
                      \arrow["h", dashed, from=4-6, to=4-7]
                      \arrow["Ti"', from=5-7, to=4-7]
                      \arrow["Tu"', from=5-5, to=5-7]
                      \arrow["{i'}", color={rgb,255:red,92;green,214;blue,92}, from=4-2, to=5-1]
                      \arrow[Rightarrow, no head, from=5-1, to=5-5]
                  \end{tikzcd}.\]
    \end{itemize}
\end{definition}

\begin{definition}[三角子范畴]
    称三角范畴的子范畴 $\mathcal C'\subset \mathcal C$ 为三角子范畴, 若满足以下命题
    \begin{enumerate}
        \item 任取 $\mathcal C$ 中同构的好三角, 若一者为 $\mathcal C'$ 中的好三角, 则另一者亦然.
        \item $T$ 也是范畴 $\mathcal C'$ 的自同构. 换言之, $\mathcal C'$ 是 $\mathcal C$ 的 $T$-不变子空间.
        \item 给定 $\mathcal C$ 中好三角 $(X,Y,Z,u,v,w)$, 若 $X,Z\in \mathsf{Ob}(\mathcal C)$, 则 $Y\in \mathsf{Ob}(\mathcal C)$.
    \end{enumerate}
\end{definition}

\begin{definition}[三角函子]
    称三角范畴间的加法函子 $F:\mathcal C\to \mathcal C'$ 为三角函子, 若存在自然同构 $\varphi: FT\simeq T'F$.
\end{definition}

\begin{remark}
    依照范畴等价/同构, 定义三角函子(或三角范畴间)的同构/等价为三角同构/三角等价.
\end{remark}

\begin{remark}
    也称好三角为正合列. 相应地, 三角函子也称作正合函子.
\end{remark}

\begin{example}
    三角函子的核给出三角子范畴.
\end{example}

\begin{proposition}
    三角函子 $F:\mathcal C\to \mathcal C'$ 对 $\mathsf{Ob}$ 保持单, 对 $\mathsf{Mor}$ 保持满, 则对 $\mathsf{Mor}$ 保持单(忠实).
    \begin{proof}
        任取 $v$ 使得 $Fv=0$, 下证明 $v=0$ 即可. 考虑如下好三角的交换图
        % https://q.uiver.app/#q=WzAsOCxbMSwwLCJZIl0sWzIsMCwiWiJdLFswLDAsIlgiXSxbMywwLCJUWCJdLFswLDEsIkZYIl0sWzEsMSwiRlkiXSxbMiwxLCJGWiJdLFszLDEsIkZUWCJdLFswLDEsInYiXSxbMiwwLCJ1Il0sWzEsMywidyJdLFs0LDUsIkZ1IiwyXSxbNiw3LCJGdyIsMl0sWzIsNF0sWzAsNV0sWzEsNl0sWzMsN10sWzUsNiwiMCIsMl0sWzAsMiwidSciLDAseyJjdXJ2ZSI6LTIsInN0eWxlIjp7ImJvZHkiOnsibmFtZSI6ImRhc2hlZCJ9fX1dXQ==
        \[\begin{tikzcd}
                X & Y & Z & TX \\
                FX & FY & FZ & FTX
                \arrow["v", from=1-2, to=1-3]
                \arrow["u", from=1-1, to=1-2]
                \arrow["w", from=1-3, to=1-4]
                \arrow["Fu"', from=2-1, to=2-2]
                \arrow["Fw"', from=2-3, to=2-4]
                \arrow[from=1-1, to=2-1]
                \arrow[from=1-2, to=2-2]
                \arrow[from=1-3, to=2-3]
                \arrow[from=1-4, to=2-4]
                \arrow["0"', from=2-2, to=2-3]
                \arrow["{u'}", curve={height=-12pt}, dashed, from=1-2, to=1-1]
            \end{tikzcd}.\]
        由题设知 $Fv=0$, 故 $Fu$ 可裂满. 由于 $F$ 对 $\mathsf{Mor}$ 保持满, 则存在 $u'$ 使得 $(Fu)(Fu')=F(uu')=\mathrm{id}_{FY}$. 此时考虑如下好三角的同态
        % https://q.uiver.app/#q=WzAsOCxbMCwxLCJGWSJdLFsxLDEsIkZZIl0sWzIsMSwiMCJdLFszLDEsIlQnRlkiXSxbMCwwLCJZIl0sWzEsMCwiWSJdLFsyLDAsIlciXSxbMywwLCJUWSJdLFswLDEsIkYodXUnKSIsMl0sWzEsMl0sWzIsM10sWzQsNSwidXUnIl0sWzYsN10sWzUsNl0sWzQsMF0sWzUsMV0sWzYsMiwiIiwxLHsic3R5bGUiOnsiYm9keSI6eyJuYW1lIjoiZGFzaGVkIn19fV0sWzcsM11d
        \[\begin{tikzcd}
                Y & Y & W & TY \\
                FY & FY & 0 & {T'FY}
                \arrow["{F(uu')}"', from=2-1, to=2-2]
                \arrow[from=2-2, to=2-3]
                \arrow[from=2-3, to=2-4]
                \arrow["{uu'}", from=1-1, to=1-2]
                \arrow[from=1-3, to=1-4]
                \arrow[from=1-2, to=1-3]
                \arrow[from=1-1, to=2-1]
                \arrow[from=1-2, to=2-2]
                \arrow[dashed, from=1-3, to=2-3]
                \arrow[from=1-4, to=2-4]
            \end{tikzcd}.\]
        由于 $F$ 对 $\mathsf{Ob}$ 保持单, 且 $FW=0$, 故 $W=0$. 此时 $uu'$ 为 $Y$ 的自同构, 遂 $v=0$.
    \end{proof}
\end{proposition}

\begin{proposition}
    对三角范畴间的伴随对, 一者为三角函子当且仅当另一者为三角函子.
    \begin{proof}
        $\textcolor{red}{\text{待补充.}}$
    \end{proof}
\end{proposition}

\begin{remark}
    一般地, Abel 范畴间的正合函子仅有``左伴随右正合-右伴随左正合''一对应; 对三角范畴而言, 有``左伴随正合-右伴随正合''一对应.
\end{remark}

\section{基变换}

\begin{definition}[八面体公理]
    将定义 \ref{triangular} 中的命题改写如下: 对任意映射链 $\begin{tikzcd}
            X & Y & Z
            \arrow["{u_1}", from=1-1, to=1-2]
            \arrow["{u_2}", from=1-2, to=1-3]
        \end{tikzcd}$, 存在如下交换图使得前两行与中间两列为好三角.
    % https://q.uiver.app/#q=WzAsMTMsWzAsMCwiWCJdLFsxLDAsIlkiXSxbMSwxLCJaIl0sWzAsMSwiWCJdLFsyLDAsIlonIl0sWzIsMSwiWSciXSxbMywwLCJUWCJdLFsyLDIsIlgnIl0sWzEsMiwiWCciXSxbMSwzLCJUWSJdLFsyLDMsIlRaJyJdLFszLDEsIlRYIl0sWzMsMiwiVFkiXSxbMCwxLCJ1XzEiXSxbMSwyLCJ1XzIiLDJdLFswLDMsIiIsMix7ImxldmVsIjoyLCJzdHlsZSI6eyJoZWFkIjp7Im5hbWUiOiJub25lIn19fV0sWzMsMiwidV8zIl0sWzEsNCwidl8xIl0sWzIsNSwidl8zIl0sWzQsNiwid18xIl0sWzQsNSwiXFxhbHBoYSJdLFs4LDcsIiIsMCx7ImxldmVsIjoyLCJzdHlsZSI6eyJoZWFkIjp7Im5hbWUiOiJub25lIn19fV0sWzIsOCwidl8yIiwyXSxbNSw3LCJcXGJldGEiXSxbOCw5LCJ3XzIiLDJdLFs5LDEwLCJUdl8xIl0sWzYsMTEsIiIsMCx7ImxldmVsIjoyLCJzdHlsZSI6eyJoZWFkIjp7Im5hbWUiOiJub25lIn19fV0sWzUsMTEsIndfMyJdLFs3LDEwLCJcXGdhbW1hIl0sWzcsMTIsIndfMiJdLFsxMSwxMiwiVHVfMSJdXQ==
    \[\begin{tikzcd}
            X & Y & {Z'} & TX \\
            X & Z & {Y'} & TX \\
            & {X'} & {X'} & TY \\
            & TY & {TZ'}
            \arrow["{u_1}", from=1-1, to=1-2]
            \arrow["{u_2}"', from=1-2, to=2-2]
            \arrow[Rightarrow, no head, from=1-1, to=2-1]
            \arrow["{u_3}", from=2-1, to=2-2]
            \arrow["{v_1}", from=1-2, to=1-3]
            \arrow["{v_3}", from=2-2, to=2-3]
            \arrow["{w_1}", from=1-3, to=1-4]
            \arrow["\alpha", from=1-3, to=2-3]
            \arrow[Rightarrow, no head, from=3-2, to=3-3]
            \arrow["{v_2}"', from=2-2, to=3-2]
            \arrow["\beta", from=2-3, to=3-3]
            \arrow["{w_2}"', from=3-2, to=4-2]
            \arrow["{Tv_1}", from=4-2, to=4-3]
            \arrow[Rightarrow, no head, from=1-4, to=2-4]
            \arrow["{w_3}", from=2-3, to=2-4]
            \arrow["\gamma", from=3-3, to=4-3]
            \arrow["{w_2}", from=3-3, to=3-4]
            \arrow["{Tu_1}", from=2-4, to=3-4]
        \end{tikzcd}.\]
    称该命题为``八面体公理''.
\end{definition}

\begin{proposition}[基变换]
    八面体公理等价于如下命题: 对好三角 $(X,Y,Z,u_1,v_1,w_1)$ 与态射 $Z'\overset\varepsilon\longrightarrow Z$, 有交换图
    % https://q.uiver.app/#q=WzAsMTMsWzAsMiwiWCJdLFsxLDIsIlkiXSxbMiwyLCJaIl0sWzMsMiwiVFgiXSxbMiwxLCJaJyJdLFswLDEsIlgiXSxbMywxLCJUWCJdLFsxLDEsIlknIl0sWzEsMCwiRSJdLFsyLDAsIkUiXSxbMiwzLCJURSJdLFsxLDMsIlRFIl0sWzMsMywiVFknIl0sWzAsMSwidV8xIl0sWzEsMiwidl8xIl0sWzIsMywid18xIl0sWzUsNywidV8yIiwwLHsic3R5bGUiOnsiYm9keSI6eyJuYW1lIjoiZGFzaGVkIn19fV0sWzcsNCwidl8yIiwwLHsic3R5bGUiOnsiYm9keSI6eyJuYW1lIjoiZGFzaGVkIn19fV0sWzQsNiwid18yIiwwLHsic3R5bGUiOnsiYm9keSI6eyJuYW1lIjoiZGFzaGVkIn19fV0sWzQsMiwiXFx2YXJlcHNpbG9uIl0sWzYsMywiIiwxLHsibGV2ZWwiOjIsInN0eWxlIjp7ImJvZHkiOnsibmFtZSI6ImRhc2hlZCJ9LCJoZWFkIjp7Im5hbWUiOiJub25lIn19fV0sWzUsMCwiIiwxLHsibGV2ZWwiOjIsInN0eWxlIjp7ImJvZHkiOnsibmFtZSI6ImRhc2hlZCJ9LCJoZWFkIjp7Im5hbWUiOiJub25lIn19fV0sWzcsMSwiXFxiZXRhIiwyLHsic3R5bGUiOnsiYm9keSI6eyJuYW1lIjoiZGFzaGVkIn19fV0sWzgsOSwiIiwwLHsibGV2ZWwiOjIsInN0eWxlIjp7ImJvZHkiOnsibmFtZSI6ImRhc2hlZCJ9LCJoZWFkIjp7Im5hbWUiOiJub25lIn19fV0sWzgsNywiXFxhbHBoYSIsMix7InN0eWxlIjp7ImJvZHkiOnsibmFtZSI6ImRhc2hlZCJ9fX1dLFs5LDQsIlxcZGVsdGEiLDAseyJzdHlsZSI6eyJib2R5Ijp7Im5hbWUiOiJkYXNoZWQifX19XSxbMSwxMSwiXFxnYW1tYSIsMix7InN0eWxlIjp7ImJvZHkiOnsibmFtZSI6ImRhc2hlZCJ9fX1dLFsyLDEwLCJcXGV0YSIsMCx7InN0eWxlIjp7ImJvZHkiOnsibmFtZSI6ImRhc2hlZCJ9fX1dLFsxMSwxMCwiIiwyLHsibGV2ZWwiOjIsInN0eWxlIjp7ImJvZHkiOnsibmFtZSI6ImRhc2hlZCJ9LCJoZWFkIjp7Im5hbWUiOiJub25lIn19fV0sWzEwLDEyLCJUXFxhbHBoYSIsMCx7InN0eWxlIjp7ImJvZHkiOnsibmFtZSI6ImRhc2hlZCJ9fX1dLFszLDEyLCJUdV8yIiwwLHsic3R5bGUiOnsiYm9keSI6eyJuYW1lIjoiZGFzaGVkIn19fV1d
    \[\begin{tikzcd}
            & E & E \\
            X & {Y'} & {Z'} & TX \\
            X & Y & Z & TX \\
            & TE & TE & {TY'}
            \arrow["{u_1}", from=3-1, to=3-2]
            \arrow["{v_1}", from=3-2, to=3-3]
            \arrow["{w_1}", from=3-3, to=3-4]
            \arrow["{u_2}", dashed, from=2-1, to=2-2]
            \arrow["{v_2}", dashed, from=2-2, to=2-3]
            \arrow["{w_2}", dashed, from=2-3, to=2-4]
            \arrow["\varepsilon", from=2-3, to=3-3]
            \arrow[Rightarrow, dashed, no head, from=2-4, to=3-4]
            \arrow[Rightarrow, dashed, no head, from=2-1, to=3-1]
            \arrow["\beta"', dashed, from=2-2, to=3-2]
            \arrow[Rightarrow, dashed, no head, from=1-2, to=1-3]
            \arrow["\alpha"', dashed, from=1-2, to=2-2]
            \arrow["\delta", dashed, from=1-3, to=2-3]
            \arrow["\gamma"', dashed, from=3-2, to=4-2]
            \arrow["\eta", dashed, from=3-3, to=4-3]
            \arrow[Rightarrow, dashed, no head, from=4-2, to=4-3]
            \arrow["T\alpha", dashed, from=4-3, to=4-4]
            \arrow["{Tu_2}", dashed, from=3-4, to=4-4]
        \end{tikzcd}.\]
    \begin{proof}
        由八面体公理推得基变换: 依照八面体公理补全 $\begin{tikzcd}
                {Z'} & Z \\
                {Z'} & TX
                \arrow["\varepsilon", from=1-1, to=1-2]
                \arrow["{w_1}", from=1-2, to=2-2]
                \arrow[Rightarrow, no head, from=1-1, to=2-1]
                \arrow[from=2-1, to=2-2]
            \end{tikzcd}$ 即可; 反之, 任意映射链 $\begin{tikzcd}
                X & Y & Z
                \arrow["{u}", from=1-1, to=1-2]
                \arrow["{v}", from=1-2, to=1-3]
            \end{tikzcd}$ 总能嵌入好三角 $\begin{tikzcd}
                && X \\
                {T^{-1}Z} & {X'} & Y & Z
                \arrow["v", from=2-3, to=2-4]
                \arrow["{T^{-1}w'}", from=2-1, to=2-2]
                \arrow["{u'}", from=2-2, to=2-3]
                \arrow["u", from=1-3, to=2-3]
            \end{tikzcd}$, 而后应用基变换定理即得八面体公理.
    \end{proof}
\end{proposition}

\begin{definition}[余基变换]
    八面体公理等价于如下命题: 对好三角 $(X,Y,Z,u_1,v_1,w_1)$ 与态射 $X\overset \delta\longrightarrow X'$, 有交换图
    % https://q.uiver.app/#q=WzAsMTMsWzIsMSwiWCJdLFszLDEsIlkiXSxbNCwxLCJaIl0sWzIsMiwiWCciXSxbMywyLCJZJyJdLFs0LDIsIloiXSxbMCwyLCJUXnstMX1aIl0sWzAsMSwiVF57LTF9WiJdLFsyLDMsIlRGIl0sWzMsMCwiRiJdLFsyLDAsIkYiXSxbMywzLCJURiJdLFs0LDMsIlRYIl0sWzcsMCwiLVReey0xfXdfMiIsMCx7InN0eWxlIjp7ImJvZHkiOnsibmFtZSI6ImRhc2hlZCJ9fX1dLFswLDEsInVfMiIsMCx7InN0eWxlIjp7ImJvZHkiOnsibmFtZSI6ImRhc2hlZCJ9fX1dLFsxLDIsInZfMiIsMCx7InN0eWxlIjp7ImJvZHkiOnsibmFtZSI6ImRhc2hlZCJ9fX1dLFs2LDMsIi1UXnstMX13XzEiXSxbMyw0LCJ1XzEiXSxbNCw1LCJ2XzEiXSxbNyw2LCIiLDEseyJsZXZlbCI6Miwic3R5bGUiOnsiYm9keSI6eyJuYW1lIjoiZGFzaGVkIn0sImhlYWQiOnsibmFtZSI6Im5vbmUifX19XSxbMiw1LCIiLDEseyJsZXZlbCI6Miwic3R5bGUiOnsiYm9keSI6eyJuYW1lIjoiZGFzaGVkIn0sImhlYWQiOnsibmFtZSI6Im5vbmUifX19XSxbMCwzLCJcXGRlbHRhIiwyXSxbMSw0LCJcXGJldGEiLDIseyJzdHlsZSI6eyJib2R5Ijp7Im5hbWUiOiJkYXNoZWQifX19XSxbMTAsOSwiIiwwLHsibGV2ZWwiOjIsInN0eWxlIjp7ImJvZHkiOnsibmFtZSI6ImRhc2hlZCJ9LCJoZWFkIjp7Im5hbWUiOiJub25lIn19fV0sWzEwLDAsIlxcZXRhIiwyLHsic3R5bGUiOnsiYm9keSI6eyJuYW1lIjoiZGFzaGVkIn19fV0sWzksMSwiXFxhbHBoYSIsMix7InN0eWxlIjp7ImJvZHkiOnsibmFtZSI6ImRhc2hlZCJ9fX1dLFs0LDExLCJcXGdhbW1hIiwyLHsic3R5bGUiOnsiYm9keSI6eyJuYW1lIjoiZGFzaGVkIn19fV0sWzgsMTEsIiIsMix7ImxldmVsIjoyLCJzdHlsZSI6eyJib2R5Ijp7Im5hbWUiOiJkYXNoZWQifSwiaGVhZCI6eyJuYW1lIjoibm9uZSJ9fX1dLFszLDgsIlxcdmFyZXBzaWxvbiIsMix7InN0eWxlIjp7ImJvZHkiOnsibmFtZSI6ImRhc2hlZCJ9fX1dLFsxMSwxMiwiLVRcXGV0YSIsMCx7InN0eWxlIjp7ImJvZHkiOnsibmFtZSI6ImRhc2hlZCJ9fX1dLFs1LDEyLCItd18yIiwyLHsic3R5bGUiOnsiYm9keSI6eyJuYW1lIjoiZGFzaGVkIn19fV1d
    \[\begin{tikzcd}
            && F & F \\
            {T^{-1}Z} && X & Y & Z \\
            {T^{-1}Z} && {X'} & {Y'} & Z \\
            && TF & TF & TX
            \arrow["{-T^{-1}w_2}", dashed, from=2-1, to=2-3]
            \arrow["{u_2}", dashed, from=2-3, to=2-4]
            \arrow["{v_2}", dashed, from=2-4, to=2-5]
            \arrow["{-T^{-1}w_1}", from=3-1, to=3-3]
            \arrow["{u_1}", from=3-3, to=3-4]
            \arrow["{v_1}", from=3-4, to=3-5]
            \arrow[Rightarrow, dashed, no head, from=2-1, to=3-1]
            \arrow[Rightarrow, dashed, no head, from=2-5, to=3-5]
            \arrow["\delta"', from=2-3, to=3-3]
            \arrow["\beta"', dashed, from=2-4, to=3-4]
            \arrow[Rightarrow, dashed, no head, from=1-3, to=1-4]
            \arrow["\eta"', dashed, from=1-3, to=2-3]
            \arrow["\alpha"', dashed, from=1-4, to=2-4]
            \arrow["\gamma"', dashed, from=3-4, to=4-4]
            \arrow[Rightarrow, dashed, no head, from=4-3, to=4-4]
            \arrow["\varepsilon"', dashed, from=3-3, to=4-3]
            \arrow["{-T\eta}", dashed, from=4-4, to=4-5]
            \arrow["{-w_2}"', dashed, from=3-5, to=4-5]
        \end{tikzcd}.\]
\end{definition}

\begin{remark}
    余基变换, 基变换, 以及八面体公理彼此等价.
\end{remark}

\begin{proposition}[$4\times 4$ 引理]
    交换图 $\begin{tikzcd}
            {X_1} & {X_2} \\
            {Y_1} & {Y_2}
            \arrow["{\alpha_1}", from=1-1, to=1-2]
            \arrow["{\alpha_2}", from=2-1, to=2-2]
            \arrow["{u_1}", from=1-1, to=2-1]
            \arrow["{u_2}", from=1-2, to=2-2]
        \end{tikzcd}$ 总能补全作如下四行四列的好三角
    % https://q.uiver.app/#q=WzAsMTYsWzAsMCwiWF8xIl0sWzEsMCwiWF8yIl0sWzIsMCwiWF8zIl0sWzMsMCwiVFhfMSJdLFswLDEsIllfMSJdLFsxLDEsIllfMiJdLFsyLDEsIllfMyJdLFszLDEsIlRZXzIiXSxbMCwyLCJaXzEiXSxbMSwyLCJaXzIiXSxbMiwyLCJaXzMiXSxbMywyLCJUWl8xIl0sWzAsMywiVFhfMSJdLFsxLDMsIlRYXzIiXSxbMiwzLCJUWF8zIl0sWzMsMywiVF4yWF8xIl0sWzAsMSwiXFxhbHBoYV8xIl0sWzEsMiwiXFxiZXRhXzEiXSxbMiwzLCJcXGdhbW1hXzEiXSxbNCw1LCJcXGFscGhhXzIiXSxbOCw5LCJcXGFscGhhXzMiXSxbMTIsMTMsIlRcXGFscGhhXzEiXSxbNSw2LCJcXGJldGFfMiJdLFs5LDEwLCJcXGJldGFfMyJdLFsxMywxNCwiVFxcYmV0YV8xIl0sWzYsNywiXFxnYW1tYV8yIl0sWzEwLDExLCJcXGdhbW1hXzMiXSxbMTQsMTUsIi1UXFxnYW1tYV8xIl0sWzAsNCwidV8xIl0sWzEsNSwidV8yIl0sWzIsNiwidV8zIl0sWzMsNywiVHVfMSJdLFs0LDgsInZfMSJdLFs1LDksInZfMiJdLFs2LDEwLCJ2XzMiXSxbNywxMSwiVHZfMSJdLFs4LDEyLCJ3XzEiXSxbOSwxMywid18yIl0sWzEwLDE0LCJ3XzMiXSxbMTEsMTUsIi1Ud18xIl1d
    \[\begin{tikzcd}
            {X_1} & {X_2} & {X_3} & {TX_1} \\
            {Y_1} & {Y_2} & {Y_3} & {TY_1} \\
            {Z_1} & {Z_2} & {Z_3} & {TZ_1} \\
            {TX_1} & {TX_2} & {TX_3} & {T^2X_1}
            \arrow["{\alpha_1}", from=1-1, to=1-2]
            \arrow["{\beta_1}", from=1-2, to=1-3]
            \arrow["{\gamma_1}", from=1-3, to=1-4]
            \arrow["{\alpha_2}", from=2-1, to=2-2]
            \arrow["{\alpha_3}", from=3-1, to=3-2]
            \arrow["{T\alpha_1}", from=4-1, to=4-2]
            \arrow["{\beta_2}", from=2-2, to=2-3]
            \arrow["{\beta_3}", from=3-2, to=3-3]
            \arrow["{T\beta_1}", from=4-2, to=4-3]
            \arrow["{\gamma_2}", from=2-3, to=2-4]
            \arrow["{\gamma_3}", from=3-3, to=3-4]
            \arrow["{-T\gamma_1}", from=4-3, to=4-4]
            \arrow["{u_1}", from=1-1, to=2-1]
            \arrow["{u_2}", from=1-2, to=2-2]
            \arrow["{u_3}", from=1-3, to=2-3]
            \arrow["{Tu_1}", from=1-4, to=2-4]
            \arrow["{v_1}", from=2-1, to=3-1]
            \arrow["{v_2}", from=2-2, to=3-2]
            \arrow["{v_3}", from=2-3, to=3-3]
            \arrow["{Tv_1}", from=2-4, to=3-4]
            \arrow["{w_1}", from=3-1, to=4-1]
            \arrow["{w_2}", from=3-2, to=4-2]
            \arrow["{w_3}", from=3-3, to=4-3]
            \arrow["{-Tw_1}", from=3-4, to=4-4]
        \end{tikzcd}.\]
    其中, 右下角方块反交换, 其余方块交换.
    \begin{proof}
        注意到如下交换图
        % https://q.uiver.app/#q=WzAsMjYsWzAsMCwiWF8xIixbMCw2MCw2MCwxXV0sWzEsMCwiWF8yIixbMCw2MCw2MCwxXV0sWzEsMSwiWV8yIixbMCw2MCw2MCwxXV0sWzAsMSwiWF8xIl0sWzIsMCwiWF8zIixbMCw2MCw2MCwxXV0sWzMsMCwiVFhfMSIsWzAsNjAsNjAsMV1dLFsxLDIsIlpfMiIsWzAsNjAsNjAsMV1dLFsxLDMsIlRYXzIiLFswLDYwLDYwLDFdXSxbMiwxLCJXIl0sWzMsMSwiVFhfMSJdLFsyLDIsIlpfMiJdLFsyLDMsIlRYXzMiLFswLDYwLDYwLDFdXSxbMywyLCJUWF8yIl0sWzQsMCwiWF8xIixbMCw2MCw2MCwxXV0sWzUsMCwiWF8xIl0sWzQsMSwiWV8xIixbMCw2MCw2MCwxXV0sWzQsMiwiWl8xIixbMCw2MCw2MCwxXV0sWzQsMywiVFhfMSIsWzAsNjAsNjAsMV1dLFs1LDEsIllfMiIsWzAsNjAsNjAsMV1dLFs1LDIsIlciXSxbNiwxLCJZXzMiLFswLDYwLDYwLDFdXSxbNSwzLCJUWF8xIl0sWzcsMSwiVFlfMSIsWzAsNjAsNjAsMV1dLFs2LDIsIllfMyJdLFs3LDIsIlRaXzEiLFswLDYwLDYwLDFdXSxbNiwzLCJUWV8xIl0sWzAsMSwiXFxhbHBoYV8xIl0sWzEsMiwidV8yIiwyXSxbMCwzLCIiLDIseyJsZXZlbCI6Miwic3R5bGUiOnsiaGVhZCI6eyJuYW1lIjoibm9uZSJ9fX1dLFszLDJdLFsxLDQsIlxcYmV0YV8xIl0sWzQsNSwiXFxnYW1tYV8xIl0sWzIsNiwidl8yIiwyXSxbNiw3LCJ3XzIiLDJdLFsyLDhdLFs0LDhdLFs1LDksIiIsMCx7ImxldmVsIjoyLCJzdHlsZSI6eyJoZWFkIjp7Im5hbWUiOiJub25lIn19fV0sWzgsOV0sWzYsMTAsIiIsMCx7ImxldmVsIjoyLCJzdHlsZSI6eyJoZWFkIjp7Im5hbWUiOiJub25lIn19fV0sWzgsMTBdLFs3LDExLCJUXFxiZXRhXzEiXSxbMTAsMTFdLFsxMCwxMiwid18yIl0sWzksMTIsIlRcXGFscGhhXzEiXSxbMTMsMTQsIiIsMCx7ImxldmVsIjoyLCJzdHlsZSI6eyJoZWFkIjp7Im5hbWUiOiJub25lIn19fV0sWzEzLDE1LCJ1XzEiLDJdLFsxNSwxNiwidl8xIiwyXSxbMTYsMTcsIndfMSIsMl0sWzE1LDE4LCJcXGFscGhhXzIiXSxbMTQsMThdLFsxOCwyMCwiXFxiZXRhXzIiXSxbMTgsMTldLFsxOSwyMV0sWzE2LDE5XSxbMTcsMjEsIiIsMix7ImxldmVsIjoyLCJzdHlsZSI6eyJoZWFkIjp7Im5hbWUiOiJub25lIn19fV0sWzIwLDIyLCJcXGdhbW1hXzIiXSxbMjAsMjMsIiIsMCx7ImxldmVsIjoyLCJzdHlsZSI6eyJoZWFkIjp7Im5hbWUiOiJub25lIn19fV0sWzE5LDIzXSxbMjMsMjRdLFsyMiwyNCwiVHZfMSJdLFsyMSwyNSwiVHVfMSJdLFsyMywyNSwiXFxnYW1tYV8yIl1d
        \[\begin{tikzcd}
                \textcolor{rgb,255:red,214;green,92;blue,92}{X_1} & \textcolor{rgb,255:red,214;green,92;blue,92}{X_2} & \textcolor{rgb,255:red,214;green,92;blue,92}{X_3} & \textcolor{rgb,255:red,214;green,92;blue,92}{TX_1} & \textcolor{rgb,255:red,214;green,92;blue,92}{X_1} & {X_1} \\
                {X_1} & \textcolor{rgb,255:red,214;green,92;blue,92}{Y_2} & W & {TX_1} & \textcolor{rgb,255:red,214;green,92;blue,92}{Y_1} & \textcolor{rgb,255:red,214;green,92;blue,92}{Y_2} & \textcolor{rgb,255:red,214;green,92;blue,92}{Y_3} & \textcolor{rgb,255:red,214;green,92;blue,92}{TY_1} \\
                & \textcolor{rgb,255:red,214;green,92;blue,92}{Z_2} & {Z_2} & {TX_2} & \textcolor{rgb,255:red,214;green,92;blue,92}{Z_1} & W & {Y_3} & \textcolor{rgb,255:red,214;green,92;blue,92}{TZ_1} \\
                & \textcolor{rgb,255:red,214;green,92;blue,92}{TX_2} & \textcolor{rgb,255:red,214;green,92;blue,92}{TX_3} && \textcolor{rgb,255:red,214;green,92;blue,92}{TX_1} & {TX_1} & {TY_1}
                \arrow["{\alpha_1}", from=1-1, to=1-2]
                \arrow["{u_2}"', from=1-2, to=2-2]
                \arrow[Rightarrow, no head, from=1-1, to=2-1]
                \arrow[from=2-1, to=2-2]
                \arrow["{\beta_1}", from=1-2, to=1-3]
                \arrow["{\gamma_1}", from=1-3, to=1-4]
                \arrow["{v_2}"', from=2-2, to=3-2]
                \arrow["{w_2}"', from=3-2, to=4-2]
                \arrow[from=2-2, to=2-3]
                \arrow[from=1-3, to=2-3]
                \arrow[Rightarrow, no head, from=1-4, to=2-4]
                \arrow[from=2-3, to=2-4]
                \arrow[Rightarrow, no head, from=3-2, to=3-3]
                \arrow[from=2-3, to=3-3]
                \arrow["{T\beta_1}", from=4-2, to=4-3]
                \arrow[from=3-3, to=4-3]
                \arrow["{w_2}", from=3-3, to=3-4]
                \arrow["{T\alpha_1}", from=2-4, to=3-4]
                \arrow[Rightarrow, no head, from=1-5, to=1-6]
                \arrow["{u_1}"', from=1-5, to=2-5]
                \arrow["{v_1}"', from=2-5, to=3-5]
                \arrow["{w_1}"', from=3-5, to=4-5]
                \arrow["{\alpha_2}", from=2-5, to=2-6]
                \arrow[from=1-6, to=2-6]
                \arrow["{\beta_2}", from=2-6, to=2-7]
                \arrow[from=2-6, to=3-6]
                \arrow[from=3-6, to=4-6]
                \arrow[from=3-5, to=3-6]
                \arrow[Rightarrow, no head, from=4-5, to=4-6]
                \arrow["{\gamma_2}", from=2-7, to=2-8]
                \arrow[Rightarrow, no head, from=2-7, to=3-7]
                \arrow[from=3-6, to=3-7]
                \arrow[from=3-7, to=3-8]
                \arrow["{Tv_1}", from=2-8, to=3-8]
                \arrow["{Tu_1}", from=4-6, to=4-7]
                \arrow["{\gamma_2}", from=3-7, to=4-7]
            \end{tikzcd}.\]
        $\textcolor{red}{\text{未完待续.}}$
    \end{proof}
\end{proposition}

\section{第一章习题}

\begin{problem}
设 $u:X\longrightarrow Y$ 是预三角范畴 $\mathcal C$ 的态射. 则 $u$ 是同构当且仅当 $\begin{tikzcd}
        X & Y & 0 & TX
        \arrow["u", from=1-1, to=1-2]
        \arrow[from=1-2, to=1-3]
        \arrow[from=1-3, to=1-4]
    \end{tikzcd}$ 是好三角.
\begin{proof}
    $u$ 是同构当且仅当以下两条同时成立
    \begin{itemize}
        \item $u$ 可裂单, 即, 任意好三角 $\begin{tikzcd}
                      X & Y & Z & TX
                      \arrow["u", from=1-1, to=1-2]
                      \arrow["w", from=1-3, to=1-4]
                      \arrow["v", from=1-2, to=1-3]
                  \end{tikzcd}$ 中 $w=0$;
        \item $u$ 可裂满, 即, 任意好三角 $\begin{tikzcd}
                      X & Y & Z & TX
                      \arrow["u", from=1-1, to=1-2]
                      \arrow["w", from=1-3, to=1-4]
                      \arrow["v", from=1-2, to=1-3]
                  \end{tikzcd}$ 中 $v=0$.
    \end{itemize}
    从而 $u$ 是同构当且仅当 $\begin{tikzcd}
            X & Y & 0 & TX
            \arrow["u", from=1-1, to=1-2]
            \arrow[from=1-2, to=1-3]
            \arrow[from=1-3, to=1-4]
        \end{tikzcd}$ 是好三角.
\end{proof}

\end{problem}

\begin{problem}
设 $\mathcal C$ 是预三角范畴. 若 $(X,Y,Z,0,v,w)$ 是好三角, 则 $Z\simeq T(X)\oplus Y$. 反之, $(X,Y,T(X)\oplus Y,0,\binom01,(-1,0))$ 是好三角.
\begin{proof}
    前一问依照``三推二''法则, 遂有同构 $\varphi$ 使得下图交换
    % https://q.uiver.app/#q=WzAsOCxbMCwwLCJZIl0sWzEsMCwiWiJdLFsyLDAsIlRYIl0sWzEsMSwiVChYKVxcb3BsdXMgWSJdLFswLDEsIlkiXSxbMiwxLCJUWCJdLFszLDAsIlRZIl0sWzMsMSwiVFkiXSxbMSwyLCJ3Il0sWzAsMSwidiJdLFsyLDYsIi1UdSJdLFswLDQsIiIsMix7ImxldmVsIjoyLCJzdHlsZSI6eyJoZWFkIjp7Im5hbWUiOiJub25lIn19fV0sWzYsNywiIiwxLHsibGV2ZWwiOjIsInN0eWxlIjp7ImhlYWQiOnsibmFtZSI6Im5vbmUifX19XSxbMiw1LCIiLDIseyJsZXZlbCI6Miwic3R5bGUiOnsiaGVhZCI6eyJuYW1lIjoibm9uZSJ9fX1dLFsxLDMsIlxcdmFycGhpIl0sWzUsNywiMCIsMl0sWzMsNSwiKC0xLDApIiwyXSxbNCwzLCJcXGJpbm9tMDEiLDJdXQ==
    \[\begin{tikzcd}
            Y & Z & TX & TY \\
            Y & {T(X)\oplus Y} & TX & TY
            \arrow["w", from=1-2, to=1-3]
            \arrow["v", from=1-1, to=1-2]
            \arrow["{-Tu}", from=1-3, to=1-4]
            \arrow[Rightarrow, no head, from=1-1, to=2-1]
            \arrow[Rightarrow, no head, from=1-4, to=2-4]
            \arrow[Rightarrow, no head, from=1-3, to=2-3]
            \arrow["\varphi", from=1-2, to=2-2]
            \arrow["0"', from=2-3, to=2-4]
            \arrow["{(-1,0)}"', from=2-2, to=2-3]
            \arrow["\binom01"', from=2-1, to=2-2]
        \end{tikzcd}.\]
    其中, 第二行是两个基本好三角的直和. 后一问显然.
\end{proof}
\end{problem}

\begin{problem}
设 $(\mathcal C,T)$ 是预三角范畴, $\mathcal A$ 是 Abel 范畴. 设 $\mathrm H:\mathcal C\longrightarrow \mathcal A$ 是上同调函子. 则对于 $\mathcal C$ 中的好三角之间的三角射
% https://q.uiver.app/#q=WzAsOCxbMCwwLCJYIl0sWzEsMCwiWSJdLFsyLDAsIloiXSxbMywwLCJUWCJdLFswLDEsIlgnIl0sWzEsMSwiWSciXSxbMiwxLCJaJyJdLFszLDEsIlRYJyJdLFswLDEsInUiXSxbMSwyLCJ2Il0sWzIsMywidyJdLFs0LDUsInUnIl0sWzUsNiwidiciXSxbNiw3LCJ3JyJdLFswLDQsImYiLDJdLFsxLDUsImciLDJdLFsyLDYsImgiLDJdLFszLDcsIlRmIiwyXV0=
\[\begin{tikzcd}
        X & Y & Z & TX \\
        {X'} & {Y'} & {Z'} & {TX'}
        \arrow["u", from=1-1, to=1-2]
        \arrow["v", from=1-2, to=1-3]
        \arrow["w", from=1-3, to=1-4]
        \arrow["{u'}", from=2-1, to=2-2]
        \arrow["{v'}", from=2-2, to=2-3]
        \arrow["{w'}", from=2-3, to=2-4]
        \arrow["f"', from=1-1, to=2-1]
        \arrow["g"', from=1-2, to=2-2]
        \arrow["h"', from=1-3, to=2-3]
        \arrow["Tf"', from=1-4, to=2-4]
    \end{tikzcd}\]
有 $\mathcal A$ 中长正合列的交换图
% https://q.uiver.app/#q=WzAsMTIsWzAsMCwiXFxjZG90cyJdLFswLDEsIlxcY2RvdHMiXSxbMSwwLCJcXG1hdGhybSBIXm4oWCkiXSxbMSwxLCJcXG1hdGhybSBIXm4oWCcpIl0sWzIsMCwiXFxtYXRocm0gSF5uKFkpIl0sWzIsMSwiXFxtYXRocm0gSF5uKFknKSJdLFszLDAsIlxcbWF0aHJtIEhebihaKSJdLFszLDEsIlxcbWF0aHJtIEhebihaJykiXSxbNCwwLCJcXG1hdGhybSBIXntuKzF9KFgpIl0sWzQsMSwiXFxtYXRocm0gSF57bisxfShYJykiXSxbNSwwLCJcXGNkb3RzIl0sWzUsMSwiXFxjZG90cyJdLFswLDJdLFsyLDQsIlxcbWF0aHJtIEhebih1KSJdLFs0LDYsIlxcbWF0aHJtIEhebih2KSJdLFs2LDgsIlxcbWF0aHJtIEhebih3KSJdLFszLDUsIlxcbWF0aHJtIEhebih1JykiXSxbNSw3LCJcXG1hdGhybSBIXm4odicpIl0sWzcsOSwiXFxtYXRocm0gSF5uKHcnKSJdLFs5LDExXSxbOCwxMF0sWzIsMywiXFxtYXRocm0gSF5uKGYpIl0sWzQsNSwiXFxtYXRocm0gSF5uKGcpIl0sWzYsNywiXFxtYXRocm0gSF5uKGgpIl0sWzgsOSwiXFxtYXRocm0gSF57bisxfShmKSJdLFsxLDNdXQ==
\[\begin{tikzcd}
        \cdots & {\mathrm H^n(X)} & {\mathrm H^n(Y)} & {\mathrm H^n(Z)} & {\mathrm H^{n+1}(X)} & \cdots \\
        \cdots & {\mathrm H^n(X')} & {\mathrm H^n(Y')} & {\mathrm H^n(Z')} & {\mathrm H^{n+1}(X')} & \cdots
        \arrow[from=1-1, to=1-2]
        \arrow["{\mathrm H^n(u)}", from=1-2, to=1-3]
        \arrow["{\mathrm H^n(v)}", from=1-3, to=1-4]
        \arrow["{\mathrm H^n(w)}", from=1-4, to=1-5]
        \arrow["{\mathrm H^n(u')}", from=2-2, to=2-3]
        \arrow["{\mathrm H^n(v')}", from=2-3, to=2-4]
        \arrow["{\mathrm H^n(w')}", from=2-4, to=2-5]
        \arrow[from=2-5, to=2-6]
        \arrow[from=1-5, to=1-6]
        \arrow["{\mathrm H^n(f)}", from=1-2, to=2-2]
        \arrow["{\mathrm H^n(g)}", from=1-3, to=2-3]
        \arrow["{\mathrm H^n(h)}", from=1-4, to=2-4]
        \arrow["{\mathrm H^{n+1}(f)}", from=1-5, to=2-5]
        \arrow[from=2-1, to=2-2]
    \end{tikzcd}\]
这里 $\mathrm H^i(X):=\mathrm H(T^iX)$, $\mathrm H^i(u):=\mathrm H(T^iu)$. 换言之, 预三角范畴的上同调函子的连接态射是自然的.
\begin{proof}
    证明预三角范畴的态射范畴为预三角范畴即可. 关键步骤是证明态射范畴的态射范畴之态射也可补全作三角. 换言之, 任取态射范畴的态射范畴中的对象 $\begin{tikzcd}
            X & Y \\
            {X'} & {Y'}
            \arrow[from=1-1, to=1-2]
            \arrow[from=2-1, to=2-2]
            \arrow[from=1-1, to=2-1]
            \arrow[from=1-2, to=2-2]
        \end{tikzcd}$ 与 $\begin{tikzcd}
            A & B \\
            {A'} & {B'}
            \arrow[from=1-1, to=1-2]
            \arrow[from=1-2, to=2-2]
            \arrow[from=1-1, to=2-1]
            \arrow[from=2-1, to=2-2]
        \end{tikzcd}$, 其间的任意态射可嵌入好三角. 换言之, 存在交换图 $\begin{tikzcd}
            P & Q \\
            {P'} & {Q'}
            \arrow["3"{description}, from=1-2, to=2-2]
            \arrow["2"{description}, from=1-1, to=2-1]
            \arrow["4"{description}, from=2-1, to=2-2]
            \arrow["1"{description}, from=1-1, to=1-2]
        \end{tikzcd}$ 下图交换
    % https://q.uiver.app/#q=WzAsMTIsWzAsMiwiWCJdLFsyLDIsIlkiXSxbMCw0LCJYJyJdLFsyLDQsIlknIl0sWzMsMywiQSciXSxbNSwzLCJCJyJdLFszLDEsIkEiXSxbNSwxLCJCIl0sWzYsMCwiUCJdLFs4LDAsIlEiXSxbNiwyLCJQJyJdLFs4LDIsIlEnIl0sWzAsMV0sWzIsM10sWzAsMl0sWzEsM10sWzYsNF0sWzcsNV0sWzQsNV0sWzYsN10sWzAsNl0sWzIsNF0sWzEsN10sWzMsNV0sWzYsOCwiIiwxLHsic3R5bGUiOnsiYm9keSI6eyJuYW1lIjoiZGFzaGVkIn19fV0sWzQsMTAsIiIsMSx7InN0eWxlIjp7ImJvZHkiOnsibmFtZSI6ImRhc2hlZCJ9fX1dLFs1LDExLCIiLDEseyJzdHlsZSI6eyJib2R5Ijp7Im5hbWUiOiJkYXNoZWQifX19XSxbNyw5LCIiLDEseyJzdHlsZSI6eyJib2R5Ijp7Im5hbWUiOiJkYXNoZWQifX19XSxbOCw5LCIxIiwxXSxbOCwxMCwiMiIsMV0sWzksMTEsIjMiLDFdLFsxMCwxMSwiNCIsMV0sWzAsMywiIiwxLHsic3R5bGUiOnsiYm9keSI6eyJuYW1lIjoic3F1aWdnbHkifX19XSxbNiw1LCIiLDEseyJzdHlsZSI6eyJib2R5Ijp7Im5hbWUiOiJzcXVpZ2dseSJ9fX1dLFs4LDExLCIiLDEseyJzdHlsZSI6eyJib2R5Ijp7Im5hbWUiOiJzcXVpZ2dseSJ9fX1dXQ==
    \[\begin{tikzcd}
            &&&&&& P && Q \\
            &&& A && B \\
            X && Y &&&& {P'} && {Q'} \\
            &&& {A'} && {B'} \\
            {X'} && {Y'}
            \arrow[from=3-1, to=3-3]
            \arrow[from=5-1, to=5-3]
            \arrow[from=3-1, to=5-1]
            \arrow[from=3-3, to=5-3]
            \arrow[from=2-4, to=4-4]
            \arrow[from=2-6, to=4-6]
            \arrow[from=4-4, to=4-6]
            \arrow[from=2-4, to=2-6]
            \arrow[from=3-1, to=2-4]
            \arrow[from=5-1, to=4-4]
            \arrow[from=3-3, to=2-6]
            \arrow[from=5-3, to=4-6]
            \arrow[dashed, from=2-4, to=1-7]
            \arrow[dashed, from=4-4, to=3-7]
            \arrow[dashed, from=4-6, to=3-9]
            \arrow[dashed, from=2-6, to=1-9]
            \arrow["1"{description}, from=1-7, to=1-9]
            \arrow["2"{description}, from=1-7, to=3-7]
            \arrow["3"{description}, from=1-9, to=3-9]
            \arrow["4"{description}, from=3-7, to=3-9]
            \arrow[squiggly, from=3-1, to=5-3]
            \arrow[squiggly, from=2-4, to=4-6]
            \arrow[squiggly, from=1-7, to=3-9]
        \end{tikzcd}.\]
    实际上, 可通过上, 下, 左, 右侧构造态射 $\{1,2,3,4\}$. 依照对角面(波浪线处)构造态射 $P\curvearrowright Q'$, 从而检验 $3\circ 1=4\circ 2$.
\end{proof}
\end{problem}

\begin{problem}
设 $\mathcal D$ 是三角范畴 $\mathcal C=(\mathcal C,T,\mathcal E)$ 的全子加法范畴. 设 $\mathcal D$ 对于同构封闭, 并且 $T$ 是 $\mathcal D$ 的自同构. 则 $\mathcal D$ 是 $\mathcal C$ 的三角子范畴当且仅当
\begin{quote}
    若好三角 $\begin{tikzcd}
            X & Y & Z & TX
            \arrow[from=1-1, to=1-2]
            \arrow[from=1-2, to=1-3]
            \arrow[from=1-3, to=1-4]
        \end{tikzcd}$ 中 $X$ 和 $Y$ 属于 $\mathcal D$, 则 $Z\in \mathcal D$;
\end{quote}
也当且仅当
\begin{quote}
    若好三角 $\begin{tikzcd}
            X & Y & Z & TX
            \arrow[from=1-1, to=1-2]
            \arrow[from=1-2, to=1-3]
            \arrow[from=1-3, to=1-4]
        \end{tikzcd}$ 中 $Y$ 和 $Z$ 属于 $\mathcal D$, 则 $X\in \mathcal D$.
\end{quote}
\begin{proof}
    考虑顺时针旋转与逆时针旋转即可转化之为等价命题.
\end{proof}
\end{problem}

\begin{problem}
三角范畴 $\mathcal C=(\mathcal C,T,\mathcal E)$ 的一个全子范畴 $\mathcal D$ 是 $\mathcal C$ 的三角子范畴当且仅当 $\mathcal D$ 对于同构封闭, 并且 $(\mathcal D, T,\mathcal E\cap \mathcal D)$ 是三角范畴, 其中 $\mathcal E\cap \mathcal D$ 是指三项均在 $\mathcal D$ 中的 $\mathcal E$ 中的三角作成的类.
\begin{proof}
    显然.
\end{proof}
\end{problem}

\begin{problem}
设 $\mathcal C$ 是预三角范畴, $\begin{tikzcd}
        X & Y & Z & TX
        \arrow["u", from=1-1, to=1-2]
        \arrow["v", from=1-2, to=1-3]
        \arrow["w", from=1-3, to=1-4]
    \end{tikzcd}$ 是好三角, $f:W\longrightarrow Z$. 则 $wf=0$ 当且仅当存在 $f':W\longrightarrow Y$ 使得 $vf'=f$.
\begin{proof}
    若 $wf=0$, 则补全好三角间的同态
    % https://q.uiver.app/#q=WzAsOCxbMCwxLCJYIl0sWzEsMSwiWSJdLFsyLDEsIloiXSxbMywxLCJUWCJdLFsyLDAsIlciXSxbMywwLCJUWCJdLFswLDAsIlgiXSxbMSwwLCJYXFxvcGx1cyBXIl0sWzAsMSwidSIsMl0sWzEsMiwidiIsMl0sWzIsMywidyIsMl0sWzQsMiwiZiIsMl0sWzQsNSwiMCJdLFs1LDMsIiIsMSx7ImxldmVsIjoyLCJzdHlsZSI6eyJoZWFkIjp7Im5hbWUiOiJub25lIn19fV0sWzYsNywiZV8xIl0sWzcsNCwicF8yIl0sWzYsMCwiIiwxLHsibGV2ZWwiOjIsInN0eWxlIjp7ImhlYWQiOnsibmFtZSI6Im5vbmUifX19XSxbNywxLCJcXHZhcnBoaSIsMix7InN0eWxlIjp7ImJvZHkiOnsibmFtZSI6ImRhc2hlZCJ9fX1dXQ==
    \[\begin{tikzcd}
            X & {X\oplus W} & W & TX \\
            X & Y & Z & TX
            \arrow["u"', from=2-1, to=2-2]
            \arrow["v"', from=2-2, to=2-3]
            \arrow["w"', from=2-3, to=2-4]
            \arrow["f"', from=1-3, to=2-3]
            \arrow["0", from=1-3, to=1-4]
            \arrow[Rightarrow, no head, from=1-4, to=2-4]
            \arrow["{e_1}", from=1-1, to=1-2]
            \arrow["{p_2}", from=1-2, to=1-3]
            \arrow[Rightarrow, no head, from=1-1, to=2-1]
            \arrow["\varphi"', dashed, from=1-2, to=2-2]
        \end{tikzcd}.\]
    其中 $\{e_i,p_i\}_{i=1,2}$ 是结构态射. 根据满态射的右消去性, 有 $v\varphi e_2=f$. 取 $f'=\varphi e_2$ 即可. 若存在 $f'$ 使得 $vf'=f$, 则 $wf=(wv)f'=0$.
\end{proof}
\end{problem}

\begin{problem}
设 $\mathcal C$ 是预三角范畴, $(X,Y,Z,u,v,w)$, $(X',Y',Z',u',v',w')$ 是好三角, $g:Y\longrightarrow Y'$. 则 $v'gu=0$ 当且仅当存在丛第一个三角到第二个三角的三角射 $(f,g,h)$.
\begin{quote}
    此时若 $\mathrm{Hom}_{\mathcal C}(X,T^{-1}Z')=0$, 则 $f$, $h$ 由 $g$ 唯一确定.
\end{quote}
\begin{proof}
    一方面, $v'gv=0$ 表明 $v'g$ 通过 $X'$ 分解, 从而构造 $f$. 依照``二推三''得三角射. 反之显然. 唯一性间\ref{unique 2->3}.
\end{proof}
\end{problem}

\begin{problem}
设 $\mathcal C$ 是预三角范畴, $(X,Y,Z,u,v,w)$ 是好三角. 若 $\mathrm{Hom}_{\mathcal C}(TX,Z)=0$, 则 $w$ 是唯一的态射使得 $(X,Y,Z,u,v,w)$ 是好三角.
\begin{proof}
    依照 \ref{unique 2->3}, 下图虚线处的态射 $\varphi$ 是唯一的
    % https://q.uiver.app/#q=WzAsOCxbMCwwLCJYIl0sWzEsMCwiWSJdLFsyLDAsIloiXSxbMywwLCJUWCJdLFswLDEsIlgiXSxbMSwxLCJZIl0sWzIsMSwiWiJdLFszLDEsIlRYIl0sWzAsMSwidSJdLFsxLDIsInYiXSxbMiwzLCJ3Il0sWzYsNywidyciLDJdLFs0LDUsInUiLDJdLFs1LDYsInYiLDJdLFszLDcsIlRcXHZhcnBoaSAiLDAseyJzdHlsZSI6eyJib2R5Ijp7Im5hbWUiOiJkYXNoZWQifX19XSxbMSw1LCIiLDEseyJsZXZlbCI6Miwic3R5bGUiOnsiaGVhZCI6eyJuYW1lIjoibm9uZSJ9fX1dLFsyLDYsIiIsMSx7ImxldmVsIjoyLCJzdHlsZSI6eyJoZWFkIjp7Im5hbWUiOiJub25lIn19fV0sWzAsNCwiXFx2YXJwaGkiLDIseyJzdHlsZSI6eyJib2R5Ijp7Im5hbWUiOiJkYXNoZWQifX19XV0=
    \[\begin{tikzcd}
            X & Y & Z & TX \\
            X & Y & Z & TX
            \arrow["u", from=1-1, to=1-2]
            \arrow["v", from=1-2, to=1-3]
            \arrow["w", from=1-3, to=1-4]
            \arrow["{w'}"', from=2-3, to=2-4]
            \arrow["u"', from=2-1, to=2-2]
            \arrow["v"', from=2-2, to=2-3]
            \arrow["{T\varphi }", dashed, from=1-4, to=2-4]
            \arrow[Rightarrow, no head, from=1-2, to=2-2]
            \arrow[Rightarrow, no head, from=1-3, to=2-3]
            \arrow["\varphi"', dashed, from=1-1, to=2-1]
        \end{tikzcd}.\]
    下仅需证明 $\varphi=\mathrm{id}_X$. 由于 $u(\mathrm{id}_X-\varphi)=0$, 故 $(\mathrm{id}_X-\varphi)$ 通过 $T^{-1}w$ 分解. 依照 $\mathrm{Hom}_{\mathcal C}(TX,Z)=0$ 知 $\varphi =\mathrm{id}_X$.
\end{proof}
\end{problem}

\begin{problem}
从预三角范畴 $\mathcal C$ 到 Abel 范畴 $\mathcal A$ 的一个加法函子 $F:\mathcal C\to \mathcal A$ 是上同调函子当且仅当对任一好三角 $\begin{tikzcd}
        X & Y & Z & TX
        \arrow["u", from=1-1, to=1-2]
        \arrow["v", from=1-2, to=1-3]
        \arrow["w", from=1-3, to=1-4]
    \end{tikzcd}$ 和任意 $i$, $\begin{tikzcd}
        {H(T^iX)} & {H(T^iY)} & {H(T^iZ)}
        \arrow[from=1-1, to=1-2]
        \arrow[from=1-2, to=1-3]
    \end{tikzcd}$ 正合.
\begin{proof}
    置 $X'=T^iX$, $Y'=T^iY$, $Z'=T^iZ$ 即可.
\end{proof}
\end{problem}

\begin{problem}
设 $F:\mathcal C\to \mathcal C'$ 是三角函子. 则 $\mathrm{ker\,}F=\{X\in \mathcal C\mid F(X)\simeq 0\}$ 是 $\mathcal C$ 的三角子范畴.
\begin{proof}
    易观察到 $\mathrm{Ker\,}F$ 关于同构封闭, 与 $T$ 交换. 若 $\begin{tikzcd}
            X & Y & Z & TX
            \arrow["u", from=1-1, to=1-2]
            \arrow["v", from=1-2, to=1-3]
            \arrow["w", from=1-3, to=1-4]
        \end{tikzcd}$ 中 $X,Z\in \mathrm{Ker\,}F$, 则好三角 $\begin{tikzcd}
            FX & FY & FZ & F(TX)
            \arrow["Fu", from=1-1, to=1-2]
            \arrow["Fv", from=1-2, to=1-3]
            \arrow["Fw", from=1-3, to=1-4]
        \end{tikzcd}$ 中 $Fu=Fv=Fw=0$, 因此 $FY\simeq FX\simeq 0$. 从而 $Y\in \mathrm{Ker\,}F$.
\end{proof}
\end{problem}

\begin{problem}
设 $\mathcal C$ 是预三角范畴. 则 $T^2:\mathcal C\to \mathcal C$ 是三角自同构.
\begin{proof}
    $T^2$ 即顺时针旋转六次, 自然是 $\mathcal C$ 的三角自同构.
\end{proof}
\end{problem}

\begin{problem}
对任意三角范畴 $(\mathcal C,T)$, $(T,-\mathrm{Id}_{T^2})$ 是 $\mathcal C$ 到自己的三角同构, 其中 $-\mathrm{Id}_{T^2}:T^2\to T^2$ 是下述的自然同构: 对于任一 $X\in \mathcal C$, $-\mathrm{Id}_{T^2}(X)$ 定义为 $-\mathrm{Id}_{T^2X}$.
\begin{proof}
    三角范畴 $T:(\mathcal C,T)\to (\mathcal C,T)$, 其中自然同构 $(-\mathrm{Id}_{T^2})_X:T^2X\overset\sim \to -T^2X$. $T$ 良定义. 注意到 $\begin{tikzcd}
            X & Y & Z & TX
            \arrow["u", from=1-1, to=1-2]
            \arrow["v", from=1-2, to=1-3]
            \arrow["w", from=1-3, to=1-4]
        \end{tikzcd}$ 是 $\mathcal C$ 中好三角当且仅当 $\begin{tikzcd}
            {-TX} & {-TY} & {-TZ} & {-T^2X}
            \arrow["{-Tu}", from=1-1, to=1-2]
            \arrow["{-Tv}", from=1-2, to=1-3]
            \arrow["{-Tw}", from=1-3, to=1-4]
        \end{tikzcd}$ 亦然, 其态射范畴同理.
\end{proof}
\end{problem}

\begin{problem}
对三角范畴 $(\mathcal C,[1])$ 和整数 $n$, $([n],(-1)^n\mathrm{Id}_{[n+1]}):\mathcal C\to \mathcal C$ 是三角同构.
\begin{proof}
    即上题之推广. 证明略.
\end{proof}
\end{problem}

\begin{problem}
\begin{proof}

\end{proof}
\end{problem}

\begin{problem}
\begin{proof}

\end{proof}
\end{problem}

\begin{problem}
\begin{proof}

\end{proof}
\end{problem}

\begin{problem}
\begin{proof}

\end{proof}
\end{problem}

\begin{problem}
\begin{proof}

\end{proof}
\end{problem}

\begin{problem}
\begin{proof}

\end{proof}
\end{problem}

\begin{problem}
\begin{proof}

\end{proof}
\end{problem}

\begin{problem}
\begin{proof}

\end{proof}
\end{problem}

\begin{problem}
\begin{proof}

\end{proof}
\end{problem}

\end{document}