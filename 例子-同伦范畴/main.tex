\documentclass{MainStyle}

\usepackage{amsthm, amsfonts, amsmath, amssymb, quiver, mathrsfs, newclude, tikz-cd, ctex}

% Customise href Colours.
\usepackage[colorlinks = true,
            linkcolor = blue,
            urlcolor  = blue,
            citecolor = blue,
            anchorcolor = blue]{hyperref}

\newcommand{\changeurlcolor}[1]{\hypersetup{urlcolor=#1}}       

\newcommand*{\name}{张陈成}
\newcommand*{\id}{023071910029}
\newcommand*{\course}{三角范畴抄书笔记}
\newcommand*{\assignment}{例子: 同伦范畴}

\theoremstyle{definition}
\newtheorem{example}{例}

\theoremstyle{definition}
\newtheorem{slogan}{原旨}

\theoremstyle{definition}
\newtheorem{definition}{定义}

\theoremstyle{definition}
\newtheorem{proposition}{命题}

\theoremstyle{definition}
\newtheorem{problem}{问题}

\theoremstyle{definition}
\newtheorem{assumption}{假定}

\theoremstyle{definition}
\newtheorem{theorem}{定理}

\theoremstyle{remark}
\newtheorem{remark}{注}

\theoremstyle{remark}
\newtheorem{lemma}{引理}
\allowdisplaybreaks

\begin{document}
\maketitle
\tableofcontents

\section{链复形}

\begin{slogan}
    默认 $\mathcal A$ 是加法范畴.
\end{slogan}

\begin{definition}[上链复形]
    $\mathcal A$ 上的一个上链复形是形如如下映射链
    % https://q.uiver.app/#q=WzAsNSxbMSwwLCJYXntpLTF9Il0sWzIsMCwiWF5pIl0sWzMsMCwiWF57aSsxfSJdLFs0LDAsIlxcY2RvdHMiXSxbMCwwLCJcXGNkb3RzIl0sWzQsMCwiZF57aS0yfSJdLFswLDEsImRee2ktMX0iXSxbMSwyLCJkXmkiXSxbMiwzLCJkXntpKzF9Il1d
    \[\begin{tikzcd}
            \cdots & {X^{i-1}} & {X^i} & {X^{i+1}} & \cdots
            \arrow["{d^{i-2}}", from=1-1, to=1-2]
            \arrow["{d^{i-1}}", from=1-2, to=1-3]
            \arrow["{d^i}", from=1-3, to=1-4]
            \arrow["{d^{i+1}}", from=1-4, to=1-5]
        \end{tikzcd}\]
    其中相邻两项复合为 $0$, 即, $d^{i+1}d^{i}$ 对一切 $i\in \mathbb Z$ 成立. 记上链复形 $X:=X^\bullet:=(X^n,d_X^n)_{n\in \mathbb Z}$.
\end{definition}

\begin{definition}[上链复形间同态]
    称 $f:X\to Y$ 是上链复形复形的同态, 若 $\left(\begin{tikzcd}
            {X^n} \\
            {Y^n}
            \arrow["{f^n}", from=1-1, to=2-1]
        \end{tikzcd},(d_X^n,d_Y^n)\right)$ 是态射范畴的上链复形.
\end{definition}

\begin{remark}
    链复形即反变的上链复形.
\end{remark}

\begin{slogan}
    以下简称上链复形为复形.
\end{slogan}

\begin{definition}[复形范畴]
    $\mathcal A$ 的复形范畴为 $C(\mathcal A)$, 其对象为复形, 态射为复形间同态.
\end{definition}

\begin{proposition}
    加法范畴(相应地, Abel 范畴)的复形范畴仍为加法范畴(相应地, Abel 范畴).
    \begin{proof}
        给定加法范畴 $\mathcal A$, 可自然地定义零复形, 复形的有限直和, 复形态射的加法群, 从而 $C(\mathcal A)$ 是加法范畴. \par
        若 $\mathcal A$ 是 Abel 范畴, 下给出 $f:X\to Y$ 的核. 依次构造 $\begin{tikzcd}
                {K^n} & {X^n} & {Y^n}
                \arrow["{f^n}", from=1-2, to=1-3]
                \arrow["{\iota^n}", hook, from=1-1, to=1-2]
            \end{tikzcd}$. 根据核的泛性质, 存在唯一的态射 $d_K^n$ 使得下图交换
        % https://q.uiver.app/#q=WzAsMTUsWzQsMSwiWF5uIl0sWzQsMiwiWV5uIl0sWzQsMCwiS15uIl0sWzYsMCwiS157bisxfSJdLFs2LDEsIlhee24rMX0iXSxbNiwyLCJZXntuKzF9Il0sWzIsMCwiS157bi0xfSJdLFsyLDEsIlhee24tMX0iXSxbMiwyLCJZXntuLTF9Il0sWzAsMCwiXFxjZG90cyJdLFswLDEsIlxcY2RvdHMiXSxbMCwyLCJcXGNkb3RzIl0sWzgsMCwiXFxjZG90cyJdLFs4LDEsIlxcY2RvdHMiXSxbOCwyLCJcXGNkb3RzIl0sWzAsMSwiZl5uIiwyXSxbMiwwLCJcXGlvdGFebiIsMix7InN0eWxlIjp7InRhaWwiOnsibmFtZSI6Imhvb2siLCJzaWRlIjoidG9wIn19fV0sWzQsNSwiZl57bisxfSJdLFszLDQsIlxcaW90YV57bisxfSIsMCx7InN0eWxlIjp7InRhaWwiOnsibmFtZSI6Imhvb2siLCJzaWRlIjoidG9wIn19fV0sWzEsNSwiZF9ZXm4iLDJdLFswLDQsImRfWF5uIl0sWzIsMywiZF9LXm4iLDAseyJzdHlsZSI6eyJib2R5Ijp7Im5hbWUiOiJkYXNoZWQifX19XSxbOCwxLCJkX1lee24tMX0iLDJdLFs3LDgsImZee24tMX0iLDJdLFs2LDcsIlxcaW90YV57bi0xfSIsMix7InN0eWxlIjp7InRhaWwiOnsibmFtZSI6Imhvb2siLCJzaWRlIjoidG9wIn19fV0sWzcsMCwiZF9YXntuLTF9Il0sWzYsMiwiZF9LXntuLTF9IiwwLHsic3R5bGUiOnsiYm9keSI6eyJuYW1lIjoiZGFzaGVkIn19fV0sWzksNiwiIiwyLHsic3R5bGUiOnsiYm9keSI6eyJuYW1lIjoiZGFzaGVkIn19fV0sWzEwLDddLFsxMSw4XSxbMywxMiwiIiwyLHsic3R5bGUiOnsiYm9keSI6eyJuYW1lIjoiZGFzaGVkIn19fV0sWzQsMTNdLFs1LDE0XV0=
        \[\begin{tikzcd}
                \cdots && {K^{n-1}} && {K^n} && {K^{n+1}} && \cdots \\
                \cdots && {X^{n-1}} && {X^n} && {X^{n+1}} && \cdots \\
                \cdots && {Y^{n-1}} && {Y^n} && {Y^{n+1}} && \cdots
                \arrow["{f^n}"', from=2-5, to=3-5]
                \arrow["{\iota^n}"', hook, from=1-5, to=2-5]
                \arrow["{f^{n+1}}", from=2-7, to=3-7]
                \arrow["{\iota^{n+1}}", hook, from=1-7, to=2-7]
                \arrow["{d_Y^n}"', from=3-5, to=3-7]
                \arrow["{d_X^n}", from=2-5, to=2-7]
                \arrow["{d_K^n}", dashed, from=1-5, to=1-7]
                \arrow["{d_Y^{n-1}}"', from=3-3, to=3-5]
                \arrow["{f^{n-1}}"', from=2-3, to=3-3]
                \arrow["{\iota^{n-1}}"', hook, from=1-3, to=2-3]
                \arrow["{d_X^{n-1}}", from=2-3, to=2-5]
                \arrow["{d_K^{n-1}}", dashed, from=1-3, to=1-5]
                \arrow[dashed, from=1-1, to=1-3]
                \arrow[from=2-1, to=2-3]
                \arrow[from=3-1, to=3-3]
                \arrow[dashed, from=1-7, to=1-9]
                \arrow[from=2-7, to=2-9]
                \arrow[from=3-7, to=3-9]
            \end{tikzcd}.\]
        依照交换图知 $\iota^{n+1}d_K^nd_K^{n-1}=0$. 根据单态射的左消去律, $K$ 为复形. 同理地, $C(\mathcal A)$ 中映射有唯一的余核. 再同理地, 核之余和等于余核之核, 即像. 因此 $C(\mathcal A)$ 为 Abel 范畴.
    \end{proof}
\end{proposition}



\end{document}