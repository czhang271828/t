\documentclass{MainStyle}

\usepackage{amsthm, amsfonts, amsmath, amssymb, quiver, mathrsfs, newclude, tikz-cd, ctex}

% Customise href Colours.
\usepackage[colorlinks = true,
 linkcolor = blue,
 urlcolor = blue,
 citecolor = blue,
 anchorcolor = blue]{hyperref}

\newcommand{\changeurlcolor}[1]{\hypersetup{urlcolor=#1}} 

\newcommand*{\name}{张陈成}
\newcommand*{\id}{023071910029}
\newcommand*{\course}{三角范畴抄书笔记}
\newcommand*{\assignment}{例子: 同伦范畴}

\theoremstyle{definition}
\newtheorem{example}{例}

\theoremstyle{definition}
\newtheorem{slogan}{原旨}

\theoremstyle{definition}
\newtheorem{definition}{定义}

\theoremstyle{definition}
\newtheorem{proposition}{命题}

\theoremstyle{definition}
\newtheorem{problem}{问题}

\theoremstyle{definition}
\newtheorem{assumption}{假定}

\theoremstyle{definition}
\newtheorem{theorem}{定理}

\theoremstyle{remark}
\newtheorem{remark}{注}

\theoremstyle{remark}
\newtheorem{lemma}{引理}
\allowdisplaybreaks

\begin{document}
\maketitle
\tableofcontents

\section{链复形}

\begin{slogan}
    默认 $\mathcal A$ 是加法范畴.
\end{slogan}

\begin{definition}[上链复形]
    $\mathcal A$ 上的一个上链复形是形如如下映射链
    % https://q.uiver.app/#q=WzAsNSxbMSwwLCJYXntpLTF9Il0sWzIsMCwiWF5pIl0sWzMsMCwiWF57aSsxfSJdLFs0LDAsIlxcY2RvdHMiXSxbMCwwLCJcXGNkb3RzIl0sWzQsMCwiZF57aS0yfSJdLFswLDEsImRee2ktMX0iXSxbMSwyLCJkXmkiXSxbMiwzLCJkXntpKzF9Il1d
    \[\begin{tikzcd}
            \cdots & {X^{i-1}} & {X^i} & {X^{i+1}} & \cdots
            \arrow["{d^{i-2}}", from=1-1, to=1-2]
            \arrow["{d^{i-1}}", from=1-2, to=1-3]
            \arrow["{d^i}", from=1-3, to=1-4]
            \arrow["{d^{i+1}}", from=1-4, to=1-5]
        \end{tikzcd}\]
    其中相邻两项复合为 $0$, 即, $d^{i+1}d^{i}$ 对一切 $i\in \mathbb Z$ 成立. 记上链复形 $X:=X^\bullet:=(X^n,d_X^n)_{n\in \mathbb Z}$.
\end{definition}

\begin{definition}[上链复形间同态]
    称 $f:X\to Y$ 是上链复形复形的同态, 若 $\left(\begin{tikzcd}
            {X^n} \\
            {Y^n}
            \arrow["{f^n}", from=1-1, to=2-1]
        \end{tikzcd},(d_X^n,d_Y^n)\right)$ 是态射范畴的上链复形.
\end{definition}

\begin{remark}
    链复形即反变的上链复形.
\end{remark}

\begin{slogan}
    以下简称上链复形为复形.
\end{slogan}

\begin{definition}[复形范畴]
    $\mathcal A$ 的复形范畴为 $C(\mathcal A)$, 其对象为复形, 态射为复形间同态.
\end{definition}

\begin{proposition}
    加法范畴(相应地, Abel 范畴)的复形范畴仍为加法范畴(相应地, Abel 范畴).
    \begin{proof}
        给定加法范畴 $\mathcal A$, 可自然地定义零复形, 复形的有限直和, 复形态射的加法群, 从而 $C(\mathcal A)$ 是加法范畴. \par
        若 $\mathcal A$ 是 Abel 范畴, 下给出 $f:X\to Y$ 的核. 依次构造 $\begin{tikzcd}
                {K^n} & {X^n} & {Y^n}
                \arrow["{f^n}", from=1-2, to=1-3]
                \arrow["{\iota^n}", hook, from=1-1, to=1-2]
            \end{tikzcd}$. 根据核的泛性质, 存在唯一的态射 $d_K^n$ 使得下图交换
        % https://q.uiver.app/#q=WzAsMTUsWzQsMSwiWF5uIl0sWzQsMiwiWV5uIl0sWzQsMCwiS15uIl0sWzYsMCwiS157bisxfSJdLFs2LDEsIlhee24rMX0iXSxbNiwyLCJZXntuKzF9Il0sWzIsMCwiS157bi0xfSJdLFsyLDEsIlhee24tMX0iXSxbMiwyLCJZXntuLTF9Il0sWzAsMCwiXFxjZG90cyJdLFswLDEsIlxcY2RvdHMiXSxbMCwyLCJcXGNkb3RzIl0sWzgsMCwiXFxjZG90cyJdLFs4LDEsIlxcY2RvdHMiXSxbOCwyLCJcXGNkb3RzIl0sWzAsMSwiZl5uIiwyXSxbMiwwLCJcXGlvdGFebiIsMix7InN0eWxlIjp7InRhaWwiOnsibmFtZSI6Imhvb2siLCJzaWRlIjoidG9wIn19fV0sWzQsNSwiZl57bisxfSJdLFszLDQsIlxcaW90YV57bisxfSIsMCx7InN0eWxlIjp7InRhaWwiOnsibmFtZSI6Imhvb2siLCJzaWRlIjoidG9wIn19fV0sWzEsNSwiZF9ZXm4iLDJdLFswLDQsImRfWF5uIl0sWzIsMywiZF9LXm4iLDAseyJzdHlsZSI6eyJib2R5Ijp7Im5hbWUiOiJkYXNoZWQifX19XSxbOCwxLCJkX1lee24tMX0iLDJdLFs3LDgsImZee24tMX0iLDJdLFs2LDcsIlxcaW90YV57bi0xfSIsMix7InN0eWxlIjp7InRhaWwiOnsibmFtZSI6Imhvb2siLCJzaWRlIjoidG9wIn19fV0sWzcsMCwiZF9YXntuLTF9Il0sWzYsMiwiZF9LXntuLTF9IiwwLHsic3R5bGUiOnsiYm9keSI6eyJuYW1lIjoiZGFzaGVkIn19fV0sWzksNiwiIiwyLHsic3R5bGUiOnsiYm9keSI6eyJuYW1lIjoiZGFzaGVkIn19fV0sWzEwLDddLFsxMSw4XSxbMywxMiwiIiwyLHsic3R5bGUiOnsiYm9keSI6eyJuYW1lIjoiZGFzaGVkIn19fV0sWzQsMTNdLFs1LDE0XV0=
        \[\begin{tikzcd}
                \cdots && {K^{n-1}} && {K^n} && {K^{n+1}} && \cdots \\
                \cdots && {X^{n-1}} && {X^n} && {X^{n+1}} && \cdots \\
                \cdots && {Y^{n-1}} && {Y^n} && {Y^{n+1}} && \cdots
                \arrow["{f^n}"', from=2-5, to=3-5]
                \arrow["{\iota^n}"', hook, from=1-5, to=2-5]
                \arrow["{f^{n+1}}", from=2-7, to=3-7]
                \arrow["{\iota^{n+1}}", hook, from=1-7, to=2-7]
                \arrow["{d_Y^n}"', from=3-5, to=3-7]
                \arrow["{d_X^n}", from=2-5, to=2-7]
                \arrow["{d_K^n}", dashed, from=1-5, to=1-7]
                \arrow["{d_Y^{n-1}}"', from=3-3, to=3-5]
                \arrow["{f^{n-1}}"', from=2-3, to=3-3]
                \arrow["{\iota^{n-1}}"', hook, from=1-3, to=2-3]
                \arrow["{d_X^{n-1}}", from=2-3, to=2-5]
                \arrow["{d_K^{n-1}}", dashed, from=1-3, to=1-5]
                \arrow[dashed, from=1-1, to=1-3]
                \arrow[from=2-1, to=2-3]
                \arrow[from=3-1, to=3-3]
                \arrow[dashed, from=1-7, to=1-9]
                \arrow[from=2-7, to=2-9]
                \arrow[from=3-7, to=3-9]
            \end{tikzcd}.\]
        依照交换图知 $\iota^{n+1}d_K^nd_K^{n-1}=0$. 根据单态射的左消去律, $K$ 为复形. 同理地, $C(\mathcal A)$ 中映射有唯一的余核. 再同理地, 核之余和等于余核之核, 即像. 因此 $C(\mathcal A)$ 为 Abel 范畴.
    \end{proof}
\end{proposition}

\begin{definition}[零伦]
    称 $X\overset h\longrightarrow Y$ 为复形间的零伦映射, 若存在一组映射 $(s^n:X^n\to Y^{n-1})_{n\in \mathbb Z}$ 使得下图中
    % https://q.uiver.app/#q=WzAsMTAsWzAsMCwiXFxjZG90cyJdLFswLDEsIlxcY2RvdHMiXSxbMSwwLCJYXntuLTF9Il0sWzIsMCwiWF5uIl0sWzMsMCwiWF57bisxfSJdLFs0LDAsIlxcY2RvdHMiXSxbMSwxLCJZXntuLTF9Il0sWzIsMSwiWV5uIl0sWzMsMSwiWV57bisxfSJdLFs0LDEsIlxcY2RvdHMiXSxbMCwyXSxbMiwzLCJkX1hee24tMX0iXSxbMSw2XSxbNiw3LCJkX1lee24tMX0iLDJdLFs3LDgsImRfWV5uIiwyXSxbOCw5XSxbNCw1XSxbMiw2LCJoXntuLTF9IiwxXSxbMyw3LCJoXm4iLDFdLFs0LDgsImhee24rMX0iLDFdLFszLDQsImRfWF5uIl0sWzMsNiwic15uIiwxLHsic3R5bGUiOnsiYm9keSI6eyJuYW1lIjoiZGFzaGVkIn19fV0sWzQsNywic157bisxfSIsMSx7InN0eWxlIjp7ImJvZHkiOnsibmFtZSI6ImRhc2hlZCJ9fX1dXQ==
    \[\begin{tikzcd}
            \cdots & {X^{n-1}} & {X^n} & {X^{n+1}} & \cdots \\
            \cdots & {Y^{n-1}} & {Y^n} & {Y^{n+1}} & \cdots
            \arrow[from=1-1, to=1-2]
            \arrow["{d_X^{n-1}}", from=1-2, to=1-3]
            \arrow[from=2-1, to=2-2]
            \arrow["{d_Y^{n-1}}"', from=2-2, to=2-3]
            \arrow["{d_Y^n}"', from=2-3, to=2-4]
            \arrow[from=2-4, to=2-5]
            \arrow[from=1-4, to=1-5]
            \arrow["{h^{n-1}}"{description}, from=1-2, to=2-2]
            \arrow["{h^n}"{description}, from=1-3, to=2-3]
            \arrow["{h^{n+1}}"{description}, from=1-4, to=2-4]
            \arrow["{d_X^n}", from=1-3, to=1-4]
            \arrow["{s^n}"{description}, dashed, from=1-3, to=2-2]
            \arrow["{s^{n+1}}"{description}, dashed, from=1-4, to=2-3]
        \end{tikzcd}\]
    $h^n=d_Y^{n-1}s^n+s^{n+1}d_X^n$ 恒成立.
\end{definition}

\begin{definition}[同伦]
    若存在复形同态 $f,g:X\to Y$ 使得 $f-g$ 关于某组 $(s^n:X^n\to Y^{n-1})_{n\in \mathbb Z}$ 零伦, 则称 $f$ 与 $g$ 关于 $s$ 同伦. 记作 $f\overset s\sim g$ 或 $s:f\sim g$. 一般地, 表述映射同伦时不强调 $s$.
\end{definition}

\begin{remark}
    $\mathrm{Hom}_{C(\mathcal A)}(X,Y)$ 关于同伦关系划分作等价类.
\end{remark}

\begin{definition}[同伦范畴]
    复形范畴 $C(\mathcal A)$ 的同伦范畴 $K(\mathcal A)$ 定义如下加法范畴:
    \begin{itemize}
        \item $\mathsf{Ob}(C(\mathcal A))=\mathsf{Ob}(K(\mathcal A))$ 为加法范畴 $\mathcal A$ 上的复形全体;
        \item $\mathrm{Hom}_{K(\mathcal A)}(X,Y):=\dfrac{\mathrm{Hom}_{C(\mathcal A)}(X,Y)}{\mathrm{Htp}(X,Y)}$ 为商 Abel 群, 其中, 子群 $\mathrm{Htp}(X,Y)$ 由 $\mathrm{Hom}_{C(\mathcal A)}(X,Y)$ 中的零伦映射全体组成.
    \end{itemize}
\end{definition}

\begin{definition}[同伦等价]
    定义两个复形在 $C(\mathcal A)$ 范畴中的同伦等价为其在 $K(\mathcal A)$ 范畴中的同构.
\end{definition}

\begin{definition}[可缩复形]
    可缩复形为 $K(\mathcal A)$ 范畴中的零对象.
\end{definition}

\begin{definition}[上同调对象]
    复形 $X$ 的 $n$ 次上同调(协变)函子为
    \begin{align*}
        \mathrm H^n:C(\mathcal A)\longrightarrow \mathrm{Ab},\quad X \longmapsto \dfrac{\mathrm{ker}(d^n_X)}{\mathrm{im}(d^{n-1}_X)}.
    \end{align*}
    可检验对任意 $n\in \mathbb Z$, $\mathrm H^n$ 是加法函子.
\end{definition}

\begin{definition}[无环复形]
    无环复形即(长)正合列.
\end{definition}

\begin{definition}[可裂复形]
    称 $X$ 是可裂复形, 当且仅当存在一列 $s^n:X^n\to X^{n-1}$ 使得 $d^ns^{n+1}d^n=d^n$. 形象地, ``右左右等于右''.
\end{definition}

\begin{proposition}
    无环复形未必可缩. Abel 范畴中, 可缩复形即可裂的无环复形.
    \begin{proof}
        首先证明无环复形未必可缩. 例如下图第一行的短正合列即无环复形
        % https://q.uiver.app/#q=WzAsMTAsWzAsMCwiMCJdLFsxLDAsIlxcbWF0aGJiIFoiXSxbMiwwLCJcXG1hdGhiYiBaIl0sWzMsMCwiXFxtYXRoYmIgWi8yXFxtYXRoYmIgWiJdLFs0LDAsIjAiXSxbMSwxLCJcXG1hdGhiYiBaIl0sWzIsMSwiXFxtYXRoYmIgWiJdLFszLDEsIlxcbWF0aGJiIFovMlxcbWF0aGJiIFoiXSxbMCwxLCIwIl0sWzQsMSwiMCJdLFswLDFdLFsxLDIsIlxcY2RvdCAyIl0sWzIsM10sWzMsNF0sWzgsNV0sWzUsNiwiXFxjZG90IDIiLDJdLFs2LDddLFs3LDldLFsxLDUsIiIsMix7ImxldmVsIjoyLCJzdHlsZSI6eyJoZWFkIjp7Im5hbWUiOiJub25lIn19fV0sWzIsNiwiIiwyLHsibGV2ZWwiOjIsInN0eWxlIjp7ImhlYWQiOnsibmFtZSI6Im5vbmUifX19XSxbMyw3LCIiLDEseyJsZXZlbCI6Miwic3R5bGUiOnsiaGVhZCI6eyJuYW1lIjoibm9uZSJ9fX1dLFsyLDUsInNeMSIsMSx7InN0eWxlIjp7ImJvZHkiOnsibmFtZSI6ImRhc2hlZCJ9fX1dLFszLDYsInNeMiIsMSx7InN0eWxlIjp7ImJvZHkiOnsibmFtZSI6ImRhc2hlZCJ9fX1dLFswLDgsIiIsMSx7ImxldmVsIjoyLCJzdHlsZSI6eyJoZWFkIjp7Im5hbWUiOiJub25lIn19fV0sWzQsOSwiIiwxLHsibGV2ZWwiOjIsInN0eWxlIjp7ImhlYWQiOnsibmFtZSI6Im5vbmUifX19XV0=
        \[\begin{tikzcd}
                0 & {\mathbb Z} & {\mathbb Z} & {\mathbb Z/2\mathbb Z} & 0 \\
                0 & {\mathbb Z} & {\mathbb Z} & {\mathbb Z/2\mathbb Z} & 0
                \arrow[from=1-1, to=1-2]
                \arrow["{\cdot 2}", from=1-2, to=1-3]
                \arrow[from=1-3, to=1-4]
                \arrow[from=1-4, to=1-5]
                \arrow[from=2-1, to=2-2]
                \arrow["{\cdot 2}"', from=2-2, to=2-3]
                \arrow[from=2-3, to=2-4]
                \arrow[from=2-4, to=2-5]
                \arrow[Rightarrow, no head, from=1-2, to=2-2]
                \arrow[Rightarrow, no head, from=1-3, to=2-3]
                \arrow[Rightarrow, no head, from=1-4, to=2-4]
                \arrow["{s^1}"{description}, dashed, from=1-3, to=2-2]
                \arrow["{s^2}"{description}, dashed, from=1-4, to=2-3]
                \arrow[Rightarrow, no head, from=1-1, to=2-1]
                \arrow[Rightarrow, no head, from=1-5, to=2-5]
            \end{tikzcd},\]
        但其恒等映射非零伦. 若不然, 记恒等映射关于 $s$ 零伦, 则 $s^2=0$, 此时 $s^1$ 不存在. 矛盾. \par
        下证明 Abel 范畴中可裂无环复形等价于可缩复形. 一方面, 若 $X$ 可缩, 则 $\mathrm{id}_X\sim 0$. 此时存在一列 $s^n:X^n\to X^{n-1}$ 使得 $\mathrm{id}_{X^n}= d^{n-1}s^n+s^{n+1}d^n$. 对上式右侧复合 $d^{n-1}$ 即得 $d^{n-1}=d^{n-1}s^nd^{n-1}$. \par
        反之, 若 $X$ 是可裂无环复形, 则存在幂等映射 $\varphi_n:=(s^{n+1}d^n):X^n\to X^n$. 此时有交换图
        % https://q.uiver.app/#q=WzAsNSxbMCwxLCJYXm4iXSxbMSwxLCJYXm4iXSxbMCwwLCJcXG1hdGhybXtpbX0oXFx2YXJwaGlfbikiXSxbMiwxLCJYXm4iXSxbMiwwLCJcXG1hdGhybXtpbX0oXFx2YXJwaGlfbikiXSxbMCwxLCJcXHZhcnBoaV9uIiwyXSxbMCwyLCJcXHdpZGV0aWxkZXtcXHZhcnBoaV9ufSIsMCx7InN0eWxlIjp7ImhlYWQiOnsibmFtZSI6ImVwaSJ9fX1dLFsyLDEsIm1fe1xcdmFycGhpX259IiwwLHsic3R5bGUiOnsidGFpbCI6eyJuYW1lIjoiaG9vayIsInNpZGUiOiJ0b3AifX19XSxbMSwzLCJcXHZhcnBoaV9uIiwyXSxbMSw0LCJcXHdpZGV0aWxkZXtcXHZhcnBoaV9ufSIsMCx7InN0eWxlIjp7ImhlYWQiOnsibmFtZSI6ImVwaSJ9fX1dLFs0LDMsIm1fe1xcdmFycGhpX259IiwwLHsic3R5bGUiOnsidGFpbCI6eyJuYW1lIjoiaG9vayIsInNpZGUiOiJ0b3AifX19XSxbMiw0LCJcXG1hdGhybXtpZH1fe1xcbWF0aHJte2ltfShcXHZhcnBoaV9uKX0iLDAseyJzdHlsZSI6eyJib2R5Ijp7Im5hbWUiOiJkYXNoZWQifX19XV0=
        \[\begin{tikzcd}
                {\mathrm{im}(\varphi_n)} && {\mathrm{im}(\varphi_n)} \\
                {X^n} & {X^n} & {X^n}
                \arrow["{\varphi_n}"', from=2-1, to=2-2]
                \arrow["{\widetilde{\varphi_n}}", two heads, from=2-1, to=1-1]
                \arrow["{m_{\varphi_n}}", hook, from=1-1, to=2-2]
                \arrow["{\varphi_n}"', from=2-2, to=2-3]
                \arrow["{\widetilde{\varphi_n}}", two heads, from=2-2, to=1-3]
                \arrow["{m_{\varphi_n}}", hook, from=1-3, to=2-3]
                \arrow["{\mathrm{id}_{\mathrm{im}(\varphi_n)}}", dashed, from=1-1, to=1-3]
            \end{tikzcd}.\]
        其中, 根据 Abel 范畴中满-单分解的在同构意义下的唯一性, 不妨记虚线处为恒等映射. 此时短正合列
        % https://q.uiver.app/#q=WzAsNSxbMiwwLCJYXm4iXSxbMSwwLCJcXG1hdGhybXtrZXJ9KFxcdmFycGhpX24pIl0sWzMsMCwiXFxtYXRocm17aW19KFxcdmFycGhpX24pIl0sWzAsMCwiMCJdLFs0LDAsIjAiXSxbMCwyLCJcXHdpZGV0aWxkZXtcXHZhcnBoaV9ufSJdLFsxLDAsIlxcaW90YV9uIl0sWzMsMV0sWzIsNF1d
        \[\begin{tikzcd}
                0 & {\mathrm{ker}(\varphi_n)} & {X^n} & {\mathrm{im}(\varphi_n)} & 0
                \arrow["{\widetilde{\varphi_n}}", from=1-3, to=1-4]
                \arrow["{\iota_n}", from=1-2, to=1-3]
                \arrow[from=1-1, to=1-2]
                \arrow[from=1-4, to=1-5]
            \end{tikzcd}\]
        可裂($\widetilde{\varphi^n}$ 可裂满). 因此, $X$ 同构于可缩复形的直和
        \begin{align*}
            X\simeq \bigoplus_{n\in \mathbb Z}\left[\cdots \longrightarrow 0\longrightarrow \mathrm{ker}(\varphi_n)\overset{d^n}\longrightarrow \mathrm{im}(\varphi_n)\longrightarrow 0\longrightarrow \cdots\right].
        \end{align*}
    \end{proof}
\end{proposition}

\begin{proposition}[正合范畴的强形式蛇引理(章-荣引理)]
    略.
\end{proposition}

\begin{remark}[同调代数基本定理]
    以上命题的推论是同调代数基本定理: 对任意 Abel 范畴的短正合列
    % https://q.uiver.app/#q=WzAsNSxbMCwwLCIwIl0sWzEsMCwiWCJdLFsyLDAsIlkiXSxbMywwLCJaIl0sWzQsMCwiMCJdLFswLDFdLFsxLDIsIlxcaW90YSJdLFsyLDMsIlxccGkiXSxbMyw0XV0=
    \[\begin{tikzcd}
            0 & X & Y & Z & 0
            \arrow[from=1-1, to=1-2]
            \arrow["\iota", from=1-2, to=1-3]
            \arrow["\pi", from=1-3, to=1-4]
            \arrow[from=1-4, to=1-5]
        \end{tikzcd},\]
    总有长正合列
    % https://q.uiver.app/#q=WzAsNyxbMiwwLCJcXG1hdGhybSBIXm4oWCkiXSxbMywwLCJcXG1hdGhybSBIXm4oWSkiXSxbNCwwLCJcXG1hdGhybSBIXm4oWikiXSxbMSwwLCJcXG1hdGhybSBIXntuLTF9KFopIl0sWzUsMCwiXFxtYXRocm0gSF57bisxfShYKSJdLFswLDAsIlxcY2RvdHMiXSxbNiwwLCJcXGNkb3RzIl0sWzMsMCwiXFxwYXJ0aWFsIF57bi0xfSJdLFswLDEsIlxcbWF0aHJtIEhebihcXGlvdGEpIl0sWzEsMiwiXFxtYXRocm0gSF5uKFxccGkpIl0sWzIsNCwiXFxwYXJ0aWFsXm4iXSxbNSwzXSxbNCw2XV0=
    \[\begin{tikzcd}
            \cdots & {\mathrm H^{n-1}(Z)} & {\mathrm H^n(X)} & {\mathrm H^n(Y)} & {\mathrm H^n(Z)} & {\mathrm H^{n+1}(X)} & \cdots
            \arrow["{\partial ^{n-1}}", from=1-2, to=1-3]
            \arrow["{\mathrm H^n(\iota)}", from=1-3, to=1-4]
            \arrow["{\mathrm H^n(\pi)}", from=1-4, to=1-5]
            \arrow["{\partial^n}", from=1-5, to=1-6]
            \arrow[from=1-1, to=1-2]
            \arrow[from=1-6, to=1-7]
        \end{tikzcd}.\]
    对态射范畴使用蛇引理, 易知连接态射 $\partial$ 自然.
\end{remark}

\begin{definition}[拟同构]
    称 $X\overset f\longrightarrow Y$ 为拟同构, 若对一切 $n\in \mathbb Z$, $\mathrm H^n(f)$ 均给出 Abel 群的同构.
\end{definition}

\begin{example}
    注意以下相似命题.
    \begin{enumerate}
        \item ``可缩复形''``无环复形''用于描述单个复形的性质. Abel 范畴中, 可缩复形即可裂的无环复形.
        \item ``同构''``同伦等价''``拟同构''用于描述两个复形对象的关系. 其中,
              % https://q.uiver.app/#q=WzAsMyxbMCwwLCJcXHRleHR75ZCM5p6EfSJdLFsyLDAsIlxcdGV4dHvlkIzkvKbnrYnku7d9Il0sWzQsMCwiXFx0ZXh0e+aLn+WQjOaehH0iXSxbMCwxLCJcXHRleHR75Lil5qC85by65LqOfSJdLFsxLDIsIlxcdGV4dHvkuKXmoLzlvLrkuo59Il1d
              \[\begin{tikzcd}
                      {\text{同构}} && {\text{同伦等价}} && {\text{拟同构}}
                      \arrow["{\text{严格强于}}", from=1-1, to=1-3]
                      \arrow["{\text{严格强于}}", from=1-3, to=1-5]
                  \end{tikzcd}.\]
        \item ``零伦''用于描述两个复形对象间的某一态射.
        \item ``映射相等''``映射同伦''``映射诱导相同的上同调态射''用于描述复形对象间的两个态射. 其中,
              % https://q.uiver.app/#q=WzAsMyxbMCwwLCJcXHRleHR75pig5bCE55u4562JfSJdLFsyLDAsIlxcdGV4dHvmmKDlsITlkIzkvKZ9Il0sWzQsMCwiXFx0ZXh0e+aYoOWwhOivseWvvOebuOWQjOeahOS4iuWQjOiwg+aAgeWwhH0iXSxbMCwxLCJcXHRleHR75Lil5qC85by65LqOfSJdLFsxLDIsIlxcdGV4dHvkuKXmoLzlvLrkuo59Il1d
              \[\begin{tikzcd}
                      {\text{映射相等}} && {\text{映射同伦}} && {\text{映射诱导相同的上同调态射}}
                      \arrow["{\text{严格强于}}", from=1-1, to=1-3]
                      \arrow["{\text{严格强于}}", from=1-3, to=1-5]
                  \end{tikzcd}\]
    \end{enumerate}
    对第二点说明如下.
    \begin{itemize}
        \item 拟同构而非同伦等价的例子如下:
              % https://q.uiver.app/#q=WzAsOCxbMCwwLCIwIl0sWzEsMCwiWCJdLFsxLDEsIlhcXG9wbHVzIFgiXSxbMywwLCIwIl0sWzAsMSwiMCJdLFsyLDAsIjAiXSxbMywxLCIwIl0sWzIsMSwiWCJdLFswLDFdLFsxLDIsIlxcYmlub20xMCIsMl0sWzUsN10sWzcsNl0sWzEsNV0sWzAsNF0sWzMsNl0sWzUsM10sWzQsMl0sWzIsNywiKDAsMSkiLDJdLFsxLDQsInNeMSIsMix7InN0eWxlIjp7ImJvZHkiOnsibmFtZSI6ImRhc2hlZCJ9fX1dLFs1LDIsInNeMiIsMix7InN0eWxlIjp7ImJvZHkiOnsibmFtZSI6ImRhc2hlZCJ9fX1dXQ==
              \[\begin{tikzcd}
                      0 & X & 0 & 0 \\
                      0 & {X\oplus X} & X & 0
                      \arrow[from=1-1, to=1-2]
                      \arrow["\binom10"', from=1-2, to=2-2]
                      \arrow[from=1-3, to=2-3]
                      \arrow[from=2-3, to=2-4]
                      \arrow[from=1-2, to=1-3]
                      \arrow[from=1-1, to=2-1]
                      \arrow[from=1-4, to=2-4]
                      \arrow[from=1-3, to=1-4]
                      \arrow[from=2-1, to=2-2]
                      \arrow["{(0,1)}"', from=2-2, to=2-3]
                      \arrow["{s^1}"', dashed, from=1-2, to=2-1]
                      \arrow["{s^2}"', dashed, from=1-3, to=2-2]
                  \end{tikzcd}.\]
              上图中, 链复形的上同调对象均同构; 但不存在 $s^1$ 与 $s^2$ 使得 $\binom 10$ 是零伦映射.
        \item 同伦等价而非同构的例子如非零可缩复形与零复形.
    \end{itemize}
    对第四点说明如下.
    \begin{enumerate}
        \item 诱导相同的上同调态射未必同伦, 例如取无环但不可裂的复形 $X$, 则 $0$ 和 $\mathrm{id}_X$ 诱导了相同的上同调态射但不同伦.
        \item 同伦而非相等的例子如零伦映射与零映射.
    \end{enumerate}
\end{example}

\section{复形同伦范畴是三角范畴}

\begin{definition}[映射锥]
    对加法范畴上的复形间态射 $X\overset u\longrightarrow Y$, 定义映射锥 $\mathrm{cone}(u)$ 为如下复形
    % https://q.uiver.app/#q=WzAsNSxbMywwLCJYXntuKzF9XFxvcGx1cyBZXm4iXSxbNSwwLCJYXntuKzJ9XFxvcGx1cyBZXntuKzF9Il0sWzEsMCwiWF5uXFxvcGx1cyBZXntuLTF9Il0sWzAsMCwiXFxjZG90cyJdLFs2LDAsIlxcY2RvdHMiXSxbMywyXSxbMiwwLCJcXGJpbm9tey1kX1heblxcLFxcLFxcLFxcLFxcLFxcLFxcLFxcLFxcLFxcLFxcLFxcLFxcLH17XFwsXFwsdV57bn1cXCxcXCxcXCxcXCxkX1lee24tMX19Il0sWzAsMSwiXFxiaW5vbXstZF9YXntuKzF9XFwsXFwsXFwsXFwsXFwsXFwsXFwsXFwsXFwsXFwsfXtcXCxcXCx1XntuKzF9XFwsXFwsXFwsXFwsZF9ZXntufX0iXSxbMSw0XV0=
    \[\begin{tikzcd}
            \cdots & {X^n\oplus Y^{n-1}} && {X^{n+1}\oplus Y^n} && {X^{n+2}\oplus Y^{n+1}} & \cdots
            \arrow[from=1-1, to=1-2]
            \arrow["{\binom{-d_X^n\,\,\,\,\,\,\,\,\,\,\,\,\,}{\,\,u^{n}\,\,\,\,d_Y^{n-1}}}", from=1-2, to=1-4]
            \arrow["{\binom{-d_X^{n+1}\,\,\,\,\,\,\,\,\,\,}{\,\,u^{n+1}\,\,\,\,d_Y^{n}}}", from=1-4, to=1-6]
            \arrow[from=1-6, to=1-7]
        \end{tikzcd}.\]
\end{definition}

\begin{slogan}
    定义后移运算 $[1]:(X^n,d^n_X)_{n\in \mathbb Z}\to (X^{n+1},d^{n+1}_X)_{n\in \mathbb Z}$ 为 $C(\mathcal A)$ (相应地, $K(\mathcal A)$)的自同构函子.
\end{slogan}

\begin{proposition}
    映射锥 $X\overset u\longrightarrow Y$ 确定 $C(\mathcal A)$ 中的态射序列
    % https://q.uiver.app/#q=WzAsNCxbMCwwLCJYIl0sWzEsMCwiWSJdLFsyLDAsIlxcbWF0aHJte2NvbmV9KHUpIl0sWzMsMCwiWFsxXSJdLFswLDEsInUiXSxbMSwyLCJcXGJpbm9tMDEiXSxbMiwzLCIoMSwwKSJdXQ==
    \[\begin{tikzcd}
            X & Y & {\mathrm{cone}(u)} & {X[1]}
            \arrow["u", from=1-1, to=1-2]
            \arrow["\binom01", from=1-2, to=1-3]
            \arrow["{(1,0)}", from=1-3, to=1-4]
        \end{tikzcd}.\]
    记 $[f]$ 为 $f$ 在 $K(\mathcal A)$ 中的像, 则 $[u]$ 给出良定义的 $K(\mathcal A)$ 中态射序列.
    \begin{proof}
        考虑 $C(\mathcal A)$ 范畴. 今仅需证明对同伦的映射 $s:u\sim v$, 总有同伦的交换图
        % https://q.uiver.app/#q=WzAsOCxbMCwwLCJYIl0sWzEsMCwiWSJdLFsyLDAsIlxcbWF0aHJte2NvbmV9KHUpIl0sWzMsMCwiWFsxXSJdLFswLDEsIlgiXSxbMSwxLCJZIl0sWzIsMSwiXFxtYXRocm17Y29uZX0odikiXSxbMywxLCJYWzFdIl0sWzAsMSwidSJdLFsxLDIsIlxcYmlub20wMSJdLFsyLDMsIigxLDApIl0sWzQsNSwidiIsMl0sWzUsNiwiXFxiaW5vbTAxIiwyXSxbNiw3LCIoMSwwKSIsMl0sWzAsNCwiIiwxLHsibGV2ZWwiOjIsInN0eWxlIjp7ImhlYWQiOnsibmFtZSI6Im5vbmUifX19XSxbMSw1LCIiLDEseyJsZXZlbCI6Miwic3R5bGUiOnsiaGVhZCI6eyJuYW1lIjoibm9uZSJ9fX1dLFszLDcsIiIsMSx7ImxldmVsIjoyLCJzdHlsZSI6eyJoZWFkIjp7Im5hbWUiOiJub25lIn19fV0sWzIsNiwiXFx2YXJwaGkiLDJdXQ==
        \[\begin{tikzcd}
                X & Y & {\mathrm{cone}(u)} & {X[1]} \\
                X & Y & {\mathrm{cone}(v)} & {X[1]}
                \arrow["u", from=1-1, to=1-2]
                \arrow["\binom01", from=1-2, to=1-3]
                \arrow["{(1,0)}", from=1-3, to=1-4]
                \arrow["v"', from=2-1, to=2-2]
                \arrow["\binom01"', from=2-2, to=2-3]
                \arrow["{(1,0)}"', from=2-3, to=2-4]
                \arrow[Rightarrow, no head, from=1-1, to=2-1]
                \arrow[Rightarrow, no head, from=1-2, to=2-2]
                \arrow[Rightarrow, no head, from=1-4, to=2-4]
                \arrow["\varphi"', from=1-3, to=2-3]
            \end{tikzcd}.\]
        注意到
        \begin{align*}
            \begin{pmatrix}\mathrm{id}_{X^{n+2}}&0\\s^n&\mathrm{id}_{Y^{n+1}}\end{pmatrix}\cdot \begin{pmatrix}-d_X^{n+1}&0\\u^{n+1}&d_Y^n\end{pmatrix} & =\begin{pmatrix}-d_X^{n+1}&0\\u^{n+1}-s^nd_X^{n+1}&d_Y^n\end{pmatrix}, \\
            \begin{pmatrix}-d_X^{n+1}&0\\v^{n+1}&d_Y^n\end{pmatrix}\cdot \begin{pmatrix}\mathrm{id}_{X^{n+1}}&0\\s^{n-1}&\mathrm{id}_{Y^{n}}\end{pmatrix} & =\begin{pmatrix}-d_X^{n+1} & 0 \\v^{n+1}+d_Y^{n}s^{n-1}&d_Y^n
            \end{pmatrix}.
        \end{align*}
        因此, $s:u\sim v$ 当且仅当 $\varphi:\begin{pmatrix}\mathrm{id}&\\s&\mathrm{id}\end{pmatrix}:\mathrm{cone}(u)\to \mathrm{cone}(v)$ 使得上图交换. 此时 $\varphi\sim \mathrm{id}_{X[1]\oplus Y}$.
    \end{proof}
\end{proposition}

\begin{proposition}[复形同伦范畴为三角范畴]
    记 $\mathcal E$ 为映射锥诱导的三角类, 即, 对任意三角 $\begin{tikzcd}
            {X'} & {Y'} & {Z'} & {X'[1]}
            \arrow["u", from=1-1, to=1-2]
            \arrow["v", from=1-2, to=1-3]
            \arrow["w", from=1-3, to=1-4]
        \end{tikzcd}$, 总存在复形 $X,Y$, 态射 $u$, 以及同伦等价 $f,g,h$, 使得下图交换.
    % https://q.uiver.app/#q=WzAsOCxbMCwwLCJYJyJdLFsxLDAsIlknIl0sWzIsMCwiWiciXSxbMywwLCJYJ1sxXSJdLFswLDEsIlgiXSxbMSwxLCJZIl0sWzIsMSwiXFxtYXRocm17Y29uZX0odSkiXSxbMywxLCJYWzFdIl0sWzAsMSwidSJdLFsxLDIsInYiXSxbMiwzLCJ3Il0sWzQsNSwidSIsMl0sWzUsNiwiXFxiaW5vbTAxIiwyXSxbNiw3LCIoMSwwKSIsMl0sWzAsNCwiZiIsMl0sWzEsNSwiZyIsMl0sWzIsNiwiaCJdLFszLDcsImZbMV0iXV0=
    \[\begin{tikzcd}
            {X'} & {Y'} & {Z'} & {X'[1]} \\
            X & Y & {\mathrm{cone}(u)} & {X[1]}
            \arrow["u", from=1-1, to=1-2]
            \arrow["v", from=1-2, to=1-3]
            \arrow["w", from=1-3, to=1-4]
            \arrow["u"', from=2-1, to=2-2]
            \arrow["\binom01"', from=2-2, to=2-3]
            \arrow["{(1,0)}"', from=2-3, to=2-4]
            \arrow["f"', from=1-1, to=2-1]
            \arrow["g"', from=1-2, to=2-2]
            \arrow["h", from=1-3, to=2-3]
            \arrow["{f[1]}", from=1-4, to=2-4]
        \end{tikzcd}.\]
    则 $(K(\mathcal A),[1],\mathcal E)$ 为三角范畴.
    \begin{proof}
        下依次验证如下几条:
        \begin{enumerate}
            \item $(X,X,0,\mathrm{id}_X,0,0)\in \mathcal E$;
            \item $\mathcal E$ 关于``顺时针旋转''封闭, 即, $\begin{tikzcd}
                          Y & {\mathrm{cone}(u)} & {X[1]} & {Y[1]}
                          \arrow["\binom01", from=1-1, to=1-2]
                          \arrow["{(0,1)}", from=1-2, to=1-3]
                          \arrow["{-u[1]}", from=1-3, to=1-4]
                      \end{tikzcd}\in \mathcal E$;
            \item ``二推三''成立, 即, 给定任意同伦交换图($f,g$ 为同伦等价)
                  % https://q.uiver.app/#q=WzAsOCxbMSwwLCJZIl0sWzIsMCwiXFxtYXRocm17Y29uZX0odSkiXSxbMywwLCJYWzFdIl0sWzAsMCwiWCJdLFswLDEsIlgnIl0sWzEsMSwiWSciXSxbMiwxLCJcXG1hdGhybXtjb25lfSh1JykiXSxbMywxLCJYJ1sxXSJdLFswLDEsIlxcYmlub20wMSJdLFsxLDIsIigwLDEpIl0sWzMsMCwidSJdLFs0LDUsInUnIiwyXSxbMyw0LCJmIiwyXSxbMCw1LCJnIiwyXSxbNiw3LCIoMCwxKSIsMl0sWzUsNiwiXFxiaW5vbTAxIiwyXSxbMiw3LCJmWzFdIl0sWzEsNiwiIiwxLHsic3R5bGUiOnsiYm9keSI6eyJuYW1lIjoiZGFzaGVkIn19fV1d
                  \[\begin{tikzcd}
                          X & Y & {\mathrm{cone}(u)} & {X[1]} \\
                          {X'} & {Y'} & {\mathrm{cone}(u')} & {X'[1]}
                          \arrow["\binom01", from=1-2, to=1-3]
                          \arrow["{(0,1)}", from=1-3, to=1-4]
                          \arrow["u", from=1-1, to=1-2]
                          \arrow["{u'}"', from=2-1, to=2-2]
                          \arrow["f"', from=1-1, to=2-1]
                          \arrow["g"', from=1-2, to=2-2]
                          \arrow["{(0,1)}"', from=2-3, to=2-4]
                          \arrow["\binom01"', from=2-2, to=2-3]
                          \arrow["{f[1]}", from=1-4, to=2-4]
                          \arrow[dashed, from=1-3, to=2-3]
                      \end{tikzcd},\]
                  总有虚线处的同伦等价使得上图交换;
            \item 八面体公理成立.
        \end{enumerate}
        为书写方便, 下省略 $d$, $u$, $1$ 等同态的角标. 若明确同态的来源与去向, 如此省略不会引起混淆.
        \begin{enumerate}
            \item 作如下 $K(\mathcal A)$ 中的交换图
                  % https://q.uiver.app/#q=WzAsOCxbMSwxLCJYIl0sWzIsMSwiXFxtYXRocm17Y29uZX0oXFxtYXRocm17aWR9X1gpIl0sWzMsMSwiWFsxXSJdLFswLDEsIlgiXSxbMCwwLCJYIl0sWzEsMCwiWCJdLFszLDAsIlhbMV0iXSxbMiwwLCIwIl0sWzAsMSwiXFxiaW5vbTAxIiwyXSxbMSwyLCIoMCwxKSIsMl0sWzQsNSwiIiwyLHsibGV2ZWwiOjIsInN0eWxlIjp7ImhlYWQiOnsibmFtZSI6Im5vbmUifX19XSxbNCwzLCIiLDEseyJsZXZlbCI6Miwic3R5bGUiOnsiaGVhZCI6eyJuYW1lIjoibm9uZSJ9fX1dLFs2LDIsIiIsMCx7ImxldmVsIjoyLCJzdHlsZSI6eyJoZWFkIjp7Im5hbWUiOiJub25lIn19fV0sWzMsMCwiIiwyLHsibGV2ZWwiOjIsInN0eWxlIjp7ImhlYWQiOnsibmFtZSI6Im5vbmUifX19XSxbNSwwLCIiLDIseyJsZXZlbCI6Miwic3R5bGUiOnsiaGVhZCI6eyJuYW1lIjoibm9uZSJ9fX1dLFs1LDddLFs3LDZdLFs3LDFdXQ==
                  \[\begin{tikzcd}
                          X & X & 0 & {X[1]} \\
                          X & X & {\mathrm{cone}(\mathrm{id}_X)} & {X[1]}
                          \arrow["\binom01"', from=2-2, to=2-3]
                          \arrow["{(0,1)}"', from=2-3, to=2-4]
                          \arrow[Rightarrow, no head, from=1-1, to=1-2]
                          \arrow[Rightarrow, no head, from=1-1, to=2-1]
                          \arrow[Rightarrow, no head, from=1-4, to=2-4]
                          \arrow[Rightarrow, no head, from=2-1, to=2-2]
                          \arrow[Rightarrow, no head, from=1-2, to=2-2]
                          \arrow[from=1-2, to=1-3]
                          \arrow[from=1-3, to=1-4]
                          \arrow[from=1-3, to=2-3]
                      \end{tikzcd}.\]
                  下证明 $\binom01:X\to \mathrm{cone}(\mathrm{id}_X)$ 是零伦的. 注意到下图即可
                  % https://q.uiver.app/#q=WzAsMTAsWzMsMCwiWF5uIl0sWzUsMCwiWF57bisxfSJdLFsxLDAsIlhee24tMX0iXSxbNiwwLCJcXGNkb3RzIl0sWzAsMCwiXFxjZG90cyJdLFsxLDEsIlhee259XFxvcGx1cyBYXntuLTF9Il0sWzMsMSwiWF57bisxfVxcb3BsdXMgWF5uIl0sWzUsMSwiWF57bisyfVxcb3BsdXMgWF57bisxfSJdLFswLDEsIlxcY2RvdHMiXSxbNiwxLCJcXGNkb3RzIl0sWzIsMCwiZF9YXntuLTF9Il0sWzAsMSwiZF9YXntufSJdLFs0LDJdLFsxLDNdLFs4LDVdLFs3LDldLFsyLDUsIlxcYmlub20wMSIsMl0sWzAsNiwiXFxiaW5vbTAxIiwyXSxbMSw3LCJcXGJpbm9tMDEiLDJdLFs1LDYsIlxcYmlub217LWRcXCxcXCxcXCxcXCwwXFwsXFwsXFwsfXsxXFwsXFwsXFwsXFwsXFwsZH0iLDJdLFs2LDcsIlxcYmlub217LWRcXCxcXCxcXCxcXCwwXFwsXFwsXFwsfXsxXFwsXFwsXFwsXFwsXFwsZH0iLDJdLFswLDUsIlxcYmlub20xMCIsMSx7InN0eWxlIjp7ImJvZHkiOnsibmFtZSI6ImRhc2hlZCJ9fX1dLFsxLDYsIlxcYmlub20xMCIsMSx7InN0eWxlIjp7ImJvZHkiOnsibmFtZSI6ImRhc2hlZCJ9fX1dXQ==
                  \[\begin{tikzcd}
                          \cdots & {X^{n-1}} && {X^n} && {X^{n+1}} & \cdots \\
                          \cdots & {X^{n}\oplus X^{n-1}} && {X^{n+1}\oplus X^n} && {X^{n+2}\oplus X^{n+1}} & \cdots
                          \arrow["{d_X^{n-1}}", from=1-2, to=1-4]
                          \arrow["{d_X^{n}}", from=1-4, to=1-6]
                          \arrow[from=1-1, to=1-2]
                          \arrow[from=1-6, to=1-7]
                          \arrow[from=2-1, to=2-2]
                          \arrow[from=2-6, to=2-7]
                          \arrow["\binom01"', from=1-2, to=2-2]
                          \arrow["\binom01"', from=1-4, to=2-4]
                          \arrow["\binom01"', from=1-6, to=2-6]
                          \arrow["{\binom{-d\,\,\,\,0\,\,\,}{1\,\,\,\,\,d}}"', from=2-2, to=2-4]
                          \arrow["{\binom{-d\,\,\,\,0\,\,\,}{1\,\,\,\,\,d}}"', from=2-4, to=2-6]
                          \arrow["\binom10"{description}, dashed, from=1-4, to=2-2]
                          \arrow["\binom10"{description}, dashed, from=1-6, to=2-4]
                      \end{tikzcd}.\]
                  最后验证 $0\to \mathrm{cone}(\mathrm{id}_X)$ 是同伦等价, 即, $\mathrm{cone}(\mathrm{id}_X)$ 是可缩复形. 事实上, 有下图
                  % https://q.uiver.app/#q=WzAsMTAsWzEsMCwiWF57bn1cXG9wbHVzIFhee24tMX0iXSxbMywwLCJYXntuKzF9XFxvcGx1cyBYXntufSJdLFs1LDAsIlhee24rMn1cXG9wbHVzIFhee24rMX0iXSxbMSwxLCJYXntufVxcb3BsdXMgWF57bi0xfSJdLFszLDEsIlhee24rMX1cXG9wbHVzIFhee259Il0sWzUsMSwiWF57bisyfVxcb3BsdXMgWF57bisxfSJdLFswLDAsIlxcY2RvdHMiXSxbMCwxLCJcXGNkb3RzIl0sWzYsMCwiXFxjZG90cyJdLFs2LDEsIlxcY2RvdHMiXSxbNiwwXSxbMCwxLCJcXGJpbm9tey1kXFwsXFwsXFwsXFwsXFwsXFwsXFwsfXsxXFwsXFwsXFwsXFwsZH0iXSxbMSwyLCJcXGJpbm9tey1kXFwsXFwsXFwsXFwsXFwsXFwsXFwsfXsxXFwsXFwsXFwsXFwsZH0iXSxbMiw4XSxbNywzXSxbMyw0LCJcXGJpbm9tey1kXFwsXFwsXFwsXFwsXFwsXFwsXFwsfXsxXFwsXFwsXFwsXFwsZH0iLDJdLFs0LDUsIlxcYmlub217LWRcXCxcXCxcXCxcXCxcXCxcXCxcXCx9ezFcXCxcXCxcXCxcXCxkfSIsMl0sWzUsOV0sWzAsMywiIiwwLHsibGV2ZWwiOjIsInN0eWxlIjp7ImhlYWQiOnsibmFtZSI6Im5vbmUifX19XSxbMSw0LCIiLDEseyJsZXZlbCI6Miwic3R5bGUiOnsiaGVhZCI6eyJuYW1lIjoibm9uZSJ9fX1dLFsyLDUsIiIsMSx7ImxldmVsIjoyLCJzdHlsZSI6eyJoZWFkIjp7Im5hbWUiOiJub25lIn19fV0sWzEsMywiXFxiaW5vbXtcXCxcXCxcXCxcXCxcXCxcXCwxfXtcXCxcXCxcXCxcXCx9IiwxLHsic3R5bGUiOnsiYm9keSI6eyJuYW1lIjoiZGFzaGVkIn19fV0sWzIsNCwiXFxiaW5vbXtcXCxcXCxcXCxcXCxcXCxcXCwxfXtcXCxcXCxcXCxcXCx9IiwxLHsic3R5bGUiOnsiYm9keSI6eyJuYW1lIjoiZGFzaGVkIn19fV1d
                  \[\begin{tikzcd}
                          \cdots & {X^{n}\oplus X^{n-1}} && {X^{n+1}\oplus X^{n}} && {X^{n+2}\oplus X^{n+1}} & \cdots \\
                          \cdots & {X^{n}\oplus X^{n-1}} && {X^{n+1}\oplus X^{n}} && {X^{n+2}\oplus X^{n+1}} & \cdots
                          \arrow[from=1-1, to=1-2]
                          \arrow["{\binom{-d\,\,\,\,\,\,\,}{1\,\,\,\,d}}", from=1-2, to=1-4]
                          \arrow["{\binom{-d\,\,\,\,\,\,\,}{1\,\,\,\,d}}", from=1-4, to=1-6]
                          \arrow[from=1-6, to=1-7]
                          \arrow[from=2-1, to=2-2]
                          \arrow["{\binom{-d\,\,\,\,\,\,\,}{1\,\,\,\,d}}"', from=2-2, to=2-4]
                          \arrow["{\binom{-d\,\,\,\,\,\,\,}{1\,\,\,\,d}}"', from=2-4, to=2-6]
                          \arrow[from=2-6, to=2-7]
                          \arrow[Rightarrow, no head, from=1-2, to=2-2]
                          \arrow[Rightarrow, no head, from=1-4, to=2-4]
                          \arrow[Rightarrow, no head, from=1-6, to=2-6]
                          \arrow["{\binom{\,\,\,\,\,\,1}{\,\,\,\,}}"{description}, dashed, from=1-4, to=2-2]
                          \arrow["{\binom{\,\,\,\,\,\,1}{\,\,\,\,}}"{description}, dashed, from=1-6, to=2-4]
                      \end{tikzcd}.\]
            \item 只需证明对存在同伦等价 $\varphi$ 使得下图在 $K(\mathcal A)$ 中交换
                  % https://q.uiver.app/#q=WzAsOCxbMCwwLCJZIl0sWzEsMCwiXFxtYXRocm17Y29uZX0odSkiXSxbMiwwLCJYWzFdIl0sWzAsMSwiWSJdLFsxLDEsIlxcbWF0aHJte2NvbmV9KHUpIl0sWzIsMSwiXFxtYXRocm17Y29uZX0oXFxiaW5vbTAxKSJdLFszLDEsIllbMV0iXSxbMywwLCJZWzFdIl0sWzAsMSwiXFxiaW5vbTAxIl0sWzEsMiwiKDEsMCkiXSxbMyw0LCJcXGJpbm9tMDEiLDJdLFs0LDUsIlxcYmlub217MH17MX0iLDJdLFs1LDYsIigxLDApIiwyXSxbMiw3LCItdVsxXSJdLFswLDMsIiIsMSx7ImxldmVsIjoyLCJzdHlsZSI6eyJoZWFkIjp7Im5hbWUiOiJub25lIn19fV0sWzEsNCwiIiwxLHsibGV2ZWwiOjIsInN0eWxlIjp7ImhlYWQiOnsibmFtZSI6Im5vbmUifX19XSxbNyw2LCIiLDAseyJsZXZlbCI6Miwic3R5bGUiOnsiaGVhZCI6eyJuYW1lIjoibm9uZSJ9fX1dLFsyLDUsIlxcdmFycGhpICIsMl1d
                  \[\begin{tikzcd}
                          Y & {\mathrm{cone}(u)} & {X[1]} & {Y[1]} \\
                          Y & {\mathrm{cone}(u)} & {\mathrm{cone}(\binom01)} & {Y[1]}
                          \arrow["\binom01", from=1-1, to=1-2]
                          \arrow["{(1,0)}", from=1-2, to=1-3]
                          \arrow["\binom01"', from=2-1, to=2-2]
                          \arrow["{\binom{0}{1}}"', from=2-2, to=2-3]
                          \arrow["{(1,0)}"', from=2-3, to=2-4]
                          \arrow["{-u[1]}", from=1-3, to=1-4]
                          \arrow[Rightarrow, no head, from=1-1, to=2-1]
                          \arrow[Rightarrow, no head, from=1-2, to=2-2]
                          \arrow[Rightarrow, no head, from=1-4, to=2-4]
                          \arrow["{\varphi }"', from=1-3, to=2-3]
                      \end{tikzcd},\]
                  兹断言下图即为所求
                  % https://q.uiver.app/#q=WzAsOCxbMCwwLCJZIl0sWzEsMCwiWFsxXVxcb3BsdXMgWSJdLFsyLDAsIlhbMV0iXSxbMCwxLCJZIl0sWzEsMSwiWFsxXVxcb3BsdXMgWSJdLFsyLDEsIllbMV1cXG9wbHVzIFhbMV1cXG9wbHVzIFkiXSxbMywxLCJZWzFdIl0sWzMsMCwiWVsxXSJdLFswLDEsIlxcYmlub20wMSJdLFsxLDIsIigxLDApIl0sWzMsNCwiXFxiaW5vbTAxIiwyXSxbNCw1LCJcXGxlZnQoXFxzdWJzdGFja3swXFwsXFwsMFxcXFwxXFwsXFwsMFxcXFwwXFwsXFwsMX1cXHJpZ2h0KSIsMl0sWzUsNiwiKDEsMCwwKSIsMl0sWzIsNywiLXVbMV0iXSxbMCwzLCIiLDEseyJsZXZlbCI6Miwic3R5bGUiOnsiaGVhZCI6eyJuYW1lIjoibm9uZSJ9fX1dLFsxLDQsIiIsMSx7ImxldmVsIjoyLCJzdHlsZSI6eyJoZWFkIjp7Im5hbWUiOiJub25lIn19fV0sWzcsNiwiIiwwLHsibGV2ZWwiOjIsInN0eWxlIjp7ImhlYWQiOnsibmFtZSI6Im5vbmUifX19XSxbMiw1LCJcXGxlZnQoXFxzdWJzdGFja3stdVsxXVxcXFwxXFxcXDB9XFxyaWdodCkiLDIseyJzdHlsZSI6eyJib2R5Ijp7Im5hbWUiOiJkYXNoZWQifX19XV0=
                  \[\begin{tikzcd}
                          Y & {X[1]\oplus Y} & {X[1]} & {Y[1]} \\
                          Y & {X[1]\oplus Y} & {Y[1]\oplus X[1]\oplus Y} & {Y[1]}
                          \arrow["\binom01", from=1-1, to=1-2]
                          \arrow["{(1,0)}", from=1-2, to=1-3]
                          \arrow["\binom01"', from=2-1, to=2-2]
                          \arrow["{\left(\substack{0\,\,0\\1\,\,0\\0\,\,1}\right)}"', from=2-2, to=2-3]
                          \arrow["{(1,0,0)}"', from=2-3, to=2-4]
                          \arrow["{-u[1]}", from=1-3, to=1-4]
                          \arrow[Rightarrow, no head, from=1-1, to=2-1]
                          \arrow[Rightarrow, no head, from=1-2, to=2-2]
                          \arrow[Rightarrow, no head, from=1-4, to=2-4]
                          \arrow["{\left(\substack{-u[1]\\1\\0}\right)}"', dashed, from=1-3, to=2-3]
                      \end{tikzcd}.\]
                  \begin{enumerate}
                      \item 先证明上图在 $K(\mathcal A)$ 中交换. 仅需验证中间方块的交换性. 等价地, 证明 $\begin{pmatrix}u[1]&0\\0&0\\0&1\end{pmatrix}: X[1]\oplus Y\longrightarrow Y[1]\oplus X[1]\oplus Y$ 是零伦的. 注意到
                            % https://q.uiver.app/#q=WzAsMTAsWzMsMCwiWF57bisxfVxcb3BsdXMgWV5uIl0sWzUsMCwiWF57bisyfVxcb3BsdXMgWV57bisxfSJdLFszLDIsIllee24rMX1cXG9wbHVzIFhee24rMX1cXG9wbHVzIFlebiJdLFsxLDIsIllee259XFxvcGx1cyBYXntufVxcb3BsdXMgWV57bi0xfSJdLFsxLDAsIlhee259XFxvcGx1cyBZXntuLTF9Il0sWzUsMiwiWV57bisyfVxcb3BsdXMgWF57bisyfVxcb3BsdXMgWV57bisxfSJdLFswLDAsIlxcY2RvdHMiXSxbMCwyLCJcXGNkb3RzIl0sWzYsMCwiXFxjZG90cyJdLFs2LDIsIlxcY2RvdHMiXSxbMCwxLCJcXGJpbm9tey1kXFwsXFwsXFwsXFwsXFwsXFwsXFwsfXt1XFwsXFwsXFwsZH0iXSxbMywyLCJcXGxlZnQoXFxzdWJzdGFja3stZFxcLFxcLFxcLFxcLFxcLFxcLFxcLFxcLFxcLFxcLFxcLFxcLFxcLFxcLFxcLFxcLFxcXFxcXCxcXCxcXCwtZFxcLFxcLFxcLFxcLFxcLFxcXFwxXFwsXFwsXFwsXFwsdVxcLFxcLFxcLFxcLGR9XFxyaWdodCkiLDJdLFs0LDAsIlxcYmlub217LWRcXCxcXCxcXCxcXCxcXCxcXCxcXCx9e3VcXCxcXCxcXCxkfSJdLFsyLDUsIlxcbGVmdChcXHN1YnN0YWNrey1kXFwsXFwsXFwsXFwsXFwsXFwsXFwsXFwsXFwsXFwsXFwsXFwsXFwsXFwsXFwsXFwsXFxcXFxcLFxcLFxcLC1kXFwsXFwsXFwsXFwsXFwsXFxcXDFcXCxcXCxcXCxcXCx1XFwsXFwsXFwsXFwsZH1cXHJpZ2h0KSIsMl0sWzAsMiwiXFxsZWZ0KFxcc3Vic3RhY2t7dVxcLFxcLFxcLFxcXFxcXCxcXCxcXFxcXFwsXFwsXFwsXFwsMX1cXHJpZ2h0KSJdLFswLDMsIlxcbGVmdChcXHN1YnN0YWNre1xcLFxcLFxcLDFcXFxcXFwsXFwsXFxcXFxcLFxcLFxcLFxcLH1cXHJpZ2h0KSIsMSx7InN0eWxlIjp7ImJvZHkiOnsibmFtZSI6ImRhc2hlZCJ9fX1dLFsxLDIsIlxcbGVmdChcXHN1YnN0YWNre1xcLFxcLFxcLDFcXFxcXFwsXFwsXFxcXFxcLFxcLFxcLFxcLH1cXHJpZ2h0KSIsMSx7InN0eWxlIjp7ImJvZHkiOnsibmFtZSI6ImRhc2hlZCJ9fX1dLFs0LDMsIlxcbGVmdChcXHN1YnN0YWNre3VcXCxcXCxcXCxcXFxcXFwsXFwsXFxcXFxcLFxcLFxcLFxcLDF9XFxyaWdodCkiXSxbMSw1LCJcXGxlZnQoXFxzdWJzdGFja3t1XFwsXFwsXFwsXFxcXFxcLFxcLFxcXFxcXCxcXCxcXCxcXCwxfVxccmlnaHQpIl0sWzYsNF0sWzcsM10sWzEsOF0sWzUsOV1d
                            \[\begin{tikzcd}
                                    \cdots & {X^{n}\oplus Y^{n-1}} && {X^{n+1}\oplus Y^n} && {X^{n+2}\oplus Y^{n+1}} & \cdots \\
                                    \\
                                    \cdots & {Y^{n}\oplus X^{n}\oplus Y^{n-1}} && {Y^{n+1}\oplus X^{n+1}\oplus Y^n} && {Y^{n+2}\oplus X^{n+2}\oplus Y^{n+1}} & \cdots
                                    \arrow["{\binom{-d\,\,\,\,\,\,\,}{u\,\,\,d}}", from=1-4, to=1-6]
                                    \arrow["{\left(\substack{-d\,\,\,\,\,\,\,\,\,\,\,\,\,\,\,\,\\\,\,\,-d\,\,\,\,\,\\1\,\,\,\,u\,\,\,\,d}\right)}"', from=3-2, to=3-4]
                                    \arrow["{\binom{-d\,\,\,\,\,\,\,}{u\,\,\,d}}", from=1-2, to=1-4]
                                    \arrow["{\left(\substack{-d\,\,\,\,\,\,\,\,\,\,\,\,\,\,\,\,\\\,\,\,-d\,\,\,\,\,\\1\,\,\,\,u\,\,\,\,d}\right)}"', from=3-4, to=3-6]
                                    \arrow["{\left(\substack{u\,\,\,\\\,\,\\\,\,\,\,1}\right)}", from=1-4, to=3-4]
                                    \arrow["{\left(\substack{\,\,\,1\\\,\,\\\,\,\,\,}\right)}"{description}, dashed, from=1-4, to=3-2]
                                    \arrow["{\left(\substack{\,\,\,1\\\,\,\\\,\,\,\,}\right)}"{description}, dashed, from=1-6, to=3-4]
                                    \arrow["{\left(\substack{u\,\,\,\\\,\,\\\,\,\,\,1}\right)}", from=1-2, to=3-2]
                                    \arrow["{\left(\substack{u\,\,\,\\\,\,\\\,\,\,\,1}\right)}", from=1-6, to=3-6]
                                    \arrow[from=1-1, to=1-2]
                                    \arrow[from=3-1, to=3-2]
                                    \arrow[from=1-6, to=1-7]
                                    \arrow[from=3-6, to=3-7]
                                \end{tikzcd}.\]
                      \item 继而证明 $\begin{pmatrix}-u[1]\\1\\0\end{pmatrix}$ 是同伦等价. 考虑映射 $(0,1,0)$, 下证明 $\begin{pmatrix}-u[1]\\1\\0\end{pmatrix}(0,1,0)=\begin{pmatrix}0&-u&0\\0&1&0\\0&0&0\end{pmatrix}$, $(0,1,0)\begin{pmatrix}-u[1]\\1\\0\end{pmatrix}=1$ 均与单位矩阵同伦. 后者显然, 前者由如下零伦关系给出
                            % https://q.uiver.app/#q=WzAsMTAsWzUsMCwiWV57bisyfVxcb3BsdXMgWF57bisyfVxcb3BsdXMgWV57bisxfSJdLFs1LDIsIllee24rMn1cXG9wbHVzIFhee24rMn1cXG9wbHVzIFlee24rMX0iXSxbNiwwLCJcXGNkb3RzIl0sWzYsMiwiXFxjZG90cyJdLFszLDAsIllee24rMX1cXG9wbHVzIFhee24rMX1cXG9wbHVzIFlebiJdLFsxLDAsIllee259XFxvcGx1cyBYXntufVxcb3BsdXMgWV57bi0xfSJdLFszLDIsIllee24rMX1cXG9wbHVzIFhee24rMX1cXG9wbHVzIFlebiJdLFsxLDIsIllee259XFxvcGx1cyBYXntufVxcb3BsdXMgWV57bi0xfSJdLFswLDAsIlxcY2RvdHMiXSxbMCwyLCJcXGNkb3RzIl0sWzAsMSwiXFxsZWZ0KFxcc3Vic3RhY2t7MVxcLFxcLFxcLHVcXCxcXCxcXCxcXCxcXCxcXFxcXFxcXFxcLFxcLFxcLFxcLFxcLFxcLFxcLFxcLFxcLDF9XFxyaWdodCkiXSxbNCwwLCJcXGxlZnQoXFxzdWJzdGFja3stZFxcLFxcLFxcLFxcLFxcLFxcLFxcLFxcLFxcLFxcLFxcLFxcLFxcLFxcLFxcLFxcLFxcXFxcXCxcXCxcXCwtZFxcLFxcLFxcLFxcLFxcLFxcXFwxXFwsXFwsXFwsXFwsdVxcLFxcLFxcLFxcLGR9XFxyaWdodCkiXSxbNSw0LCJcXGxlZnQoXFxzdWJzdGFja3stZFxcLFxcLFxcLFxcLFxcLFxcLFxcLFxcLFxcLFxcLFxcLFxcLFxcLFxcLFxcLFxcLFxcXFxcXCxcXCxcXCwtZFxcLFxcLFxcLFxcLFxcLFxcXFwxXFwsXFwsXFwsXFwsdVxcLFxcLFxcLFxcLGR9XFxyaWdodCkiXSxbNiwxLCJcXGxlZnQoXFxzdWJzdGFja3stZFxcLFxcLFxcLFxcLFxcLFxcLFxcLFxcLFxcLFxcLFxcLFxcLFxcLFxcLFxcLFxcLFxcXFxcXCxcXCxcXCwtZFxcLFxcLFxcLFxcLFxcLFxcXFwxXFwsXFwsXFwsXFwsdVxcLFxcLFxcLFxcLGR9XFxyaWdodCkiLDJdLFs3LDYsIlxcbGVmdChcXHN1YnN0YWNrey1kXFwsXFwsXFwsXFwsXFwsXFwsXFwsXFwsXFwsXFwsXFwsXFwsXFwsXFwsXFwsXFwsXFxcXFxcLFxcLFxcLC1kXFwsXFwsXFwsXFwsXFwsXFxcXDFcXCxcXCxcXCxcXCx1XFwsXFwsXFwsXFwsZH1cXHJpZ2h0KSIsMl0sWzQsNywiXFxsZWZ0KFxcc3Vic3RhY2t7XFwsXFwsXFwsXFwsXFwsXFwsXFwsXFwsMVxcXFxcXFxcXFxcXH1cXHJpZ2h0KSIsMSx7InN0eWxlIjp7ImJvZHkiOnsibmFtZSI6ImRhc2hlZCJ9fX1dLFs1LDcsIlxcbGVmdChcXHN1YnN0YWNrezFcXCxcXCxcXCx1XFwsXFwsXFwsXFwsXFwsXFxcXFxcXFxcXCxcXCxcXCxcXCxcXCxcXCxcXCxcXCxcXCwxfVxccmlnaHQpIl0sWzgsNV0sWzksN10sWzAsMl0sWzEsM10sWzQsNiwiXFxsZWZ0KFxcc3Vic3RhY2t7MVxcLFxcLFxcLHVcXCxcXCxcXCxcXCxcXCxcXFxcXFxcXFxcLFxcLFxcLFxcLFxcLFxcLFxcLFxcLFxcLDF9XFxyaWdodCkiXSxbMCw2LCJcXGxlZnQoXFxzdWJzdGFja3tcXCxcXCxcXCxcXCxcXCxcXCxcXCxcXCwxXFxcXFxcXFxcXFxcfVxccmlnaHQpIiwxLHsic3R5bGUiOnsiYm9keSI6eyJuYW1lIjoiZGFzaGVkIn19fV1d
                            \[\begin{tikzcd}
                                    \cdots & {Y^{n}\oplus X^{n}\oplus Y^{n-1}} && {Y^{n+1}\oplus X^{n+1}\oplus Y^n} && {Y^{n+2}\oplus X^{n+2}\oplus Y^{n+1}} & \cdots \\
                                    \\
                                    \cdots & {Y^{n}\oplus X^{n}\oplus Y^{n-1}} && {Y^{n+1}\oplus X^{n+1}\oplus Y^n} && {Y^{n+2}\oplus X^{n+2}\oplus Y^{n+1}} & \cdots
                                    \arrow["{\left(\substack{1\,\,\,u\,\,\,\,\,\\\\\,\,\,\,\,\,\,\,\,1}\right)}", from=1-6, to=3-6]
                                    \arrow["{\left(\substack{-d\,\,\,\,\,\,\,\,\,\,\,\,\,\,\,\,\\\,\,\,-d\,\,\,\,\,\\1\,\,\,\,u\,\,\,\,d}\right)}", from=1-4, to=1-6]
                                    \arrow["{\left(\substack{-d\,\,\,\,\,\,\,\,\,\,\,\,\,\,\,\,\\\,\,\,-d\,\,\,\,\,\\1\,\,\,\,u\,\,\,\,d}\right)}", from=1-2, to=1-4]
                                    \arrow["{\left(\substack{-d\,\,\,\,\,\,\,\,\,\,\,\,\,\,\,\,\\\,\,\,-d\,\,\,\,\,\\1\,\,\,\,u\,\,\,\,d}\right)}"', from=3-4, to=3-6]
                                    \arrow["{\left(\substack{-d\,\,\,\,\,\,\,\,\,\,\,\,\,\,\,\,\\\,\,\,-d\,\,\,\,\,\\1\,\,\,\,u\,\,\,\,d}\right)}"', from=3-2, to=3-4]
                                    \arrow["{\left(\substack{\,\,\,\,\,\,\,\,1\\\\\\}\right)}"{description}, dashed, from=1-4, to=3-2]
                                    \arrow["{\left(\substack{1\,\,\,u\,\,\,\,\,\\\\\,\,\,\,\,\,\,\,\,1}\right)}", from=1-2, to=3-2]
                                    \arrow[from=1-1, to=1-2]
                                    \arrow[from=3-1, to=3-2]
                                    \arrow[from=1-6, to=1-7]
                                    \arrow[from=3-6, to=3-7]
                                    \arrow["{\left(\substack{1\,\,\,u\,\,\,\,\,\\\\\,\,\,\,\,\,\,\,\,1}\right)}", from=1-4, to=3-4]
                                    \arrow["{\left(\substack{\,\,\,\,\,\,\,\,1\\\\\\}\right)}"{description}, dashed, from=1-6, to=3-4]
                                \end{tikzcd}.\]
                  \end{enumerate}
            \item 取虚线处同伦等价为 $f[1]\oplus g$ 即可.
            \item 给定映射链 $\begin{tikzcd}
                          X & Y & Z
                          \arrow["u", from=1-1, to=1-2]
                          \arrow["v", from=1-2, to=1-3]
                      \end{tikzcd}$, 八面体公理由如下 $C(\mathcal A)$ 中的交换图给出
                  % https://q.uiver.app/#q=WzAsMTMsWzAsMCwiWCJdLFsxLDAsIlkiXSxbMSwxLCJaIl0sWzAsMSwiWCJdLFsyLDAsIlxcbWF0aHJte2NvbmV9KHUpIl0sWzMsMCwiWFsxXSJdLFsyLDEsIlxcbWF0aHJte2NvbmV9KHZ1KSJdLFszLDEsIlhbMV0iXSxbMSwyLCJcXG1hdGhybXtjb25lfSh2KSJdLFsxLDMsIllbMV0iXSxbMiwyLCJcXG1hdGhybXtjb25lfSh2KSJdLFsyLDMsIlxcbWF0aHJte2NvbmV9KHUpWzFdIl0sWzMsMiwiWVsxXSJdLFswLDEsInUiXSxbMSwyLCJ2Il0sWzAsMywiIiwyLHsibGV2ZWwiOjIsInN0eWxlIjp7ImhlYWQiOnsibmFtZSI6Im5vbmUifX19XSxbMywyLCJ2dSJdLFsxLDQsIlxcYmlub20wMSJdLFs0LDUsIigxLDApIl0sWzIsNiwiXFxiaW5vbTAxIl0sWzUsNywiIiwwLHsibGV2ZWwiOjIsInN0eWxlIjp7ImhlYWQiOnsibmFtZSI6Im5vbmUifX19XSxbNiw3LCIoMSwwKSIsMl0sWzQsNiwiXFxiaW5vbXsxXFwsXFwsXFwsXFwsXFwsXFwsfXtcXCxcXCxcXCxcXCxcXCxcXCx2fSJdLFsyLDgsIlxcYmlub20wMSIsMl0sWzgsOSwiKDEsMCkiLDJdLFs4LDEwLCIiLDIseyJsZXZlbCI6Miwic3R5bGUiOnsiaGVhZCI6eyJuYW1lIjoibm9uZSJ9fX1dLFs2LDEwLCJcXGJpbm9te3VbMV1cXCxcXCxcXCxcXCxcXCx9e1xcLFxcLFxcLFxcLFxcLFxcLFxcLFxcLFxcLFxcLHZ9IiwyXSxbMTAsMTEsIlxcYmlub217fXsxXFwsXFwsXFwsXFwsXFwsfSIsMl0sWzksMTEsIlxcYmlub20wMSIsMl0sWzcsMTIsInVbMV0iXSxbMTAsMTIsIigxLDApIiwyXV0=
                  \[\begin{tikzcd}
                          X & Y & {\mathrm{cone}(u)} & {X[1]} \\
                          X & Z & {\mathrm{cone}(vu)} & {X[1]} \\
                          & {\mathrm{cone}(v)} & {\mathrm{cone}(v)} & {Y[1]} \\
                          & {Y[1]} & {\mathrm{cone}(u)[1]}
                          \arrow["u", from=1-1, to=1-2]
                          \arrow["v", from=1-2, to=2-2]
                          \arrow[Rightarrow, no head, from=1-1, to=2-1]
                          \arrow["vu", from=2-1, to=2-2]
                          \arrow["\binom01", from=1-2, to=1-3]
                          \arrow["{(1,0)}", from=1-3, to=1-4]
                          \arrow["\binom01", from=2-2, to=2-3]
                          \arrow[Rightarrow, no head, from=1-4, to=2-4]
                          \arrow["{(1,0)}"', from=2-3, to=2-4]
                          \arrow["{\binom{1\,\,\,\,\,\,}{\,\,\,\,\,\,v}}", from=1-3, to=2-3]
                          \arrow["\binom01"', from=2-2, to=3-2]
                          \arrow["{(1,0)}"', from=3-2, to=4-2]
                          \arrow[Rightarrow, no head, from=3-2, to=3-3]
                          \arrow["{\binom{u[1]\,\,\,\,\,}{\,\,\,\,\,\,\,\,\,\,v}}"', from=2-3, to=3-3]
                          \arrow["{\binom{}{1\,\,\,\,\,}}"', from=3-3, to=4-3]
                          \arrow["\binom01"', from=4-2, to=4-3]
                          \arrow["{u[1]}", from=2-4, to=3-4]
                          \arrow["{(1,0)}"', from=3-3, to=3-4]
                      \end{tikzcd}.\]
                  下仍需验证第三列为三角. 作 $K(\mathcal A)$ 中交换图
                  % https://q.uiver.app/#q=WzAsMTYsWzAsMCwiXFxtYXRocm17Y29uZX0odSkiXSxbMiwwLCJcXG1hdGhybXtjb25lfSh2dSkiXSxbNCwwLCJcXG1hdGhybXtjb25lfSh2KSJdLFs2LDAsIlxcbWF0aHJte2NvbmV9KHUpWzFdIl0sWzAsMSwiWFsxXVxcb3BsdXMgWSJdLFsyLDEsIlhbMV1cXG9wbHVzIFoiXSxbNCwxLCJZWzFdXFxvcGx1cyBaIl0sWzYsMSwiWFsyXVxcb3BsdXMgWVsxXSJdLFswLDMsIlhbMV1cXG9wbHVzIFkiXSxbMiwzLCJYWzFdXFxvcGx1cyBaIl0sWzYsMywiWFsyXVxcb3BsdXMgWVsxXSJdLFs0LDMsIlhbMl1cXG9wbHVzIFlbMV1cXG9wbHVzIFhbMV1cXG9wbHVzIFoiXSxbNiw0LCJcXG1hdGhybXtjb25lfSh1KVsxXSJdLFs0LDQsIlxcbWF0aHJte2NvbmV9KFxcYmlub217MVxcLFxcLFxcLFxcLFxcLFxcLH17XFwsXFwsXFwsXFwsXFwsXFwsdn0pIl0sWzIsNCwiXFxtYXRocm17Y29uZX0odikiXSxbMCw0LCJcXG1hdGhybXtjb25lfSh1KSJdLFs0LDUsIlxcYmlub217MVxcLFxcLFxcLFxcLFxcLFxcLH17XFwsXFwsXFwsXFwsXFwsXFwsdn0iXSxbNSw2LCJcXGJpbm9te3VbMV1cXCxcXCxcXCxcXCxcXCx9e1xcLFxcLFxcLFxcLFxcLFxcLFxcLFxcLFxcLFxcLDF9Il0sWzYsNywiXFxiaW5vbXt9ezFcXCxcXCxcXCxcXCxcXCx9Il0sWzQsOCwiIiwyLHsibGV2ZWwiOjIsInN0eWxlIjp7ImhlYWQiOnsibmFtZSI6Im5vbmUifX19XSxbNSw5LCIiLDIseyJsZXZlbCI6Miwic3R5bGUiOnsiaGVhZCI6eyJuYW1lIjoibm9uZSJ9fX1dLFs3LDEwLCIiLDAseyJsZXZlbCI6Miwic3R5bGUiOnsiaGVhZCI6eyJuYW1lIjoibm9uZSJ9fX1dLFs4LDksIlxcYmlub217MVxcLFxcLFxcLFxcLFxcLFxcLH17XFwsXFwsXFwsXFwsXFwsXFwsdn0iLDJdLFswLDQsIiIsMix7ImxldmVsIjoyLCJzdHlsZSI6eyJoZWFkIjp7Im5hbWUiOiJub25lIn19fV0sWzEsNSwiIiwyLHsibGV2ZWwiOjIsInN0eWxlIjp7ImhlYWQiOnsibmFtZSI6Im5vbmUifX19XSxbMiw2LCIiLDIseyJsZXZlbCI6Miwic3R5bGUiOnsiaGVhZCI6eyJuYW1lIjoibm9uZSJ9fX1dLFszLDcsIiIsMix7ImxldmVsIjoyLCJzdHlsZSI6eyJoZWFkIjp7Im5hbWUiOiJub25lIn19fV0sWzksMTEsIlxcYmlub217T197MlxcdGltZXMyfX17SV97MlxcdGltZXMgMn19IiwyXSxbNiwxMSwiRV97MiwxfStFX3s0LDJ9IiwyXSxbMTAsMTIsIiIsMCx7ImxldmVsIjoyLCJzdHlsZSI6eyJoZWFkIjp7Im5hbWUiOiJub25lIn19fV0sWzExLDEzLCIiLDAseyJsZXZlbCI6Miwic3R5bGUiOnsiaGVhZCI6eyJuYW1lIjoibm9uZSJ9fX1dLFs5LDE0LCIiLDIseyJsZXZlbCI6Miwic3R5bGUiOnsiaGVhZCI6eyJuYW1lIjoibm9uZSJ9fX1dLFs4LDE1LCIiLDIseyJsZXZlbCI6Miwic3R5bGUiOnsiaGVhZCI6eyJuYW1lIjoibm9uZSJ9fX1dLFsxMSwxMCwiKElfezJcXHRpbWVzIDJ9LE9fezJcXHRpbWVzIDJ9KSIsMl1d
                  \[\begin{tikzcd}
                          {\mathrm{cone}(u)} && {\mathrm{cone}(vu)} && {\mathrm{cone}(v)} && {\mathrm{cone}(u)[1]} \\
                          {X[1]\oplus Y} && {X[1]\oplus Z} && {Y[1]\oplus Z} && {X[2]\oplus Y[1]} \\
                          \\
                          {X[1]\oplus Y} && {X[1]\oplus Z} && {X[2]\oplus Y[1]\oplus X[1]\oplus Z} && {X[2]\oplus Y[1]} \\
                          {\mathrm{cone}(u)} && {\mathrm{cone}(v)} && {\mathrm{cone}(\binom{1\,\,\,\,\,\,}{\,\,\,\,\,\,v})} && {\mathrm{cone}(u)[1]}
                          \arrow["{\binom{1\,\,\,\,\,\,}{\,\,\,\,\,\,v}}", from=2-1, to=2-3]
                          \arrow["{\binom{u[1]\,\,\,\,\,}{\,\,\,\,\,\,\,\,\,\,1}}", from=2-3, to=2-5]
                          \arrow["{\binom{}{1\,\,\,\,\,}}", from=2-5, to=2-7]
                          \arrow[Rightarrow, no head, from=2-1, to=4-1]
                          \arrow[Rightarrow, no head, from=2-3, to=4-3]
                          \arrow[Rightarrow, no head, from=2-7, to=4-7]
                          \arrow["{\binom{1\,\,\,\,\,\,}{\,\,\,\,\,\,v}}"', from=4-1, to=4-3]
                          \arrow[Rightarrow, no head, from=1-1, to=2-1]
                          \arrow[Rightarrow, no head, from=1-3, to=2-3]
                          \arrow[Rightarrow, no head, from=1-5, to=2-5]
                          \arrow[Rightarrow, no head, from=1-7, to=2-7]
                          \arrow["{\binom{O_{2\times2}}{I_{2\times 2}}}"', from=4-3, to=4-5]
                          \arrow["{E_{2,1}+E_{4,2}}"', from=2-5, to=4-5]
                          \arrow[Rightarrow, no head, from=4-7, to=5-7]
                          \arrow[Rightarrow, no head, from=4-5, to=5-5]
                          \arrow[Rightarrow, no head, from=4-3, to=5-3]
                          \arrow[Rightarrow, no head, from=4-1, to=5-1]
                          \arrow["{(I_{2\times 2},O_{2\times 2})}"', from=4-5, to=4-7]
                      \end{tikzcd}.\]
                  其中, 左侧与右侧方块可换. 为证明中间方块交换, 只需证明以下为零伦映射
                  \begin{align*}
                      u[1]\cdot E_{2,1}-E_{3,1}=\begin{pmatrix}0&0\\u[1]&0\\-1&0\\0&0\end{pmatrix}:X[1]\oplus Z\mapsto X[2]\oplus Y[1]\oplus X[1]\oplus Z.
                  \end{align*}
                  注意到零伦关系
                  % https://q.uiver.app/#q=WzAsMTAsWzIsMiwiWF57bisyfVxcb3BsdXMgWV57bisxfVxcb3BsdXMgWF57bisxfVxcb3BsdXMgWl5uIl0sWzEsMiwiWF57bisxfVxcb3BsdXMgWV57bn1cXG9wbHVzIFhee259XFxvcGx1cyBaXntuLTF9Il0sWzIsMCwiWF57bisxfVxcb3BsdXMgWl5uIl0sWzMsMiwiWF57biszfVxcb3BsdXMgWV57bisyfVxcb3BsdXMgWF57bisyfVxcb3BsdXMgWl57bisxfSJdLFsxLDAsIlheblxcb3BsdXMgWl57bi0xfSJdLFszLDAsIlhee24rMn1cXG9wbHVzIFpee24rMX0iXSxbMCwwLCJcXGNkb3RzIl0sWzAsMiwiXFxjZG90cyJdLFs0LDAsIlxcY2RvdHMiXSxbNCwyLCJcXGNkb3RzIl0sWzQsMiwiXFxiaW5vbXstZFxcLFxcLFxcLFxcLFxcLFxcLFxcLH17XFwsXFwsZFxcLFxcLFxcLFxcLFxcLDF9Il0sWzIsNSwiXFxiaW5vbXstZFxcLFxcLFxcLFxcLFxcLFxcLFxcLH17XFwsXFwsZFxcLFxcLFxcLFxcLFxcLDF9Il0sWzEsMCwiXFxhc3QiXSxbMCwzLCJcXGFzdCJdLFsyLDAsInVcXGNkb3QgRV97MiwxfS1FX3szLDF9IiwxXSxbNCwxLCJ1XFxjZG90IEVfezIsMX0tRV97MywxfSIsMV0sWzUsMywidVxcY2RvdCBFX3syLDF9LUVfezMsMX0iLDFdLFsyLDEsIi1FX3sxLDF9IiwxLHsic3R5bGUiOnsiYm9keSI6eyJuYW1lIjoiZGFzaGVkIn19fV0sWzUsMCwiLUVfezEsMX0iLDEseyJzdHlsZSI6eyJib2R5Ijp7Im5hbWUiOiJkYXNoZWQifX19XSxbNiw0XSxbNywxXSxbNSw4XSxbMyw5XV0=
                  \[\begin{tikzcd}[column sep=scriptsize]
                          \cdots & {X^n\oplus Z^{n-1}} & {X^{n+1}\oplus Z^n} & {X^{n+2}\oplus Z^{n+1}} & \cdots \\
                          \\
                          \cdots & {X^{n+1}\oplus Y^{n}\oplus X^{n}\oplus Z^{n-1}} & {X^{n+2}\oplus Y^{n+1}\oplus X^{n+1}\oplus Z^n} & {X^{n+3}\oplus Y^{n+2}\oplus X^{n+2}\oplus Z^{n+1}} & \cdots
                          \arrow["{\binom{-d\,\,\,\,\,\,\,}{\,\,d\,\,\,\,\,1}}", from=1-2, to=1-3]
                          \arrow["{\binom{-d\,\,\,\,\,\,\,}{\,\,d\,\,\,\,\,1}}", from=1-3, to=1-4]
                          \arrow["\ast", from=3-2, to=3-3]
                          \arrow["\ast", from=3-3, to=3-4]
                          \arrow["{u\cdot E_{2,1}-E_{3,1}}"{description}, from=1-3, to=3-3]
                          \arrow["{u\cdot E_{2,1}-E_{3,1}}"{description}, from=1-2, to=3-2]
                          \arrow["{u\cdot E_{2,1}-E_{3,1}}"{description}, from=1-4, to=3-4]
                          \arrow["{-E_{1,1}}"{description}, dashed, from=1-3, to=3-2]
                          \arrow["{-E_{1,1}}"{description}, dashed, from=1-4, to=3-3]
                          \arrow[from=1-1, to=1-2]
                          \arrow[from=3-1, to=3-2]
                          \arrow[from=1-4, to=1-5]
                          \arrow[from=3-4, to=3-5]
                      \end{tikzcd}.\]
                  其中 $\ast:=\begin{pmatrix}d\\-u&-d\\1&&-d\\&v&vu&d\end{pmatrix}$.
                  最后证明 $E_{2,1}+E_{4,2}:X[1]\oplus Z\longrightarrow X[2]\oplus Y[1]\oplus X[1]\oplus Z$ 是同伦等价. 作左逆同态 $E_{1,2}+u\cdot E_{1,3}+E_{2,4}$, 下仅需证明
                  \begin{align*}
                      (E_{2,1}+u\cdot E_{1,3}+E_{4,2})(E_{1,2}+E_{2,4})=E_{2,2}+u\cdot E_{2,3}+E_{4,4}\in \mathrm{End}_{C(\mathcal A)}(X[2]\oplus Y[1]\oplus X[1]\oplus Z)
                  \end{align*}
                  同伦于恒等映射. 实际上, $-E_{1,3}$, 给出 $\begin{pmatrix}-1\\&&u\\&&-1\\&&&\end{pmatrix}\sim O$.
        \end{enumerate}
    \end{proof}
\end{proposition}

\begin{remark}
    给定映射链
    % https://q.uiver.app/#q=WzAsNyxbMiwwLCJYIl0sWzMsMCwiWSJdLFs0LDAsIlxcbWF0aHJte2NvbmV9KHUpIl0sWzUsMCwiWFsxXSJdLFsxLDAsIlxcbWF0aHJte2NvbmV9KHUpWy0xXSJdLFswLDAsIlxcY2RvdHMiXSxbNiwwLCJcXGNkb3RzIl0sWzQsMCwiKDEsMCkiXSxbMCwxLCJ1Il0sWzEsMiwiXFxiaW5vbTAxIl0sWzIsMywiKDEsMCkiXSxbNSw0XSxbMyw2XV0=
    \[\begin{tikzcd}
            \cdots & {\mathrm{cone}(u)[-1]} & X & Y & {\mathrm{cone}(u)} & {X[1]} & \cdots
            \arrow["{(-1,0)}", from=1-2, to=1-3]
            \arrow["u", from=1-3, to=1-4]
            \arrow["\binom01", from=1-4, to=1-5]
            \arrow["{(1,0)}", from=1-5, to=1-6]
            \arrow[from=1-1, to=1-2]
            \arrow[from=1-6, to=1-7]
        \end{tikzcd},\]
    相邻两项之复合为零伦.
\end{remark}

\section{映射筒}

\begin{definition}[映射筒]
    给定映射 $X\overset u\longrightarrow Y$ 与 $K(\mathcal A)$ 中三角 $\begin{tikzcd}
            X & Y & {\mathrm{cone}(u)} & {X[1]}
            \arrow["u", from=1-1, to=1-2]
            \arrow["\binom01", from=1-2, to=1-3]
            \arrow["{(1,0)}", from=1-3, to=1-4]
        \end{tikzcd}$, 定义映射筒 $\mathrm{cyl}(u)$ 为同伦核的映射锥. 即,
    \begin{align*}
        \mathrm{cyl}(u):=\mathrm{cone}\Big((-1,0):\mathrm{cone}(u)[-1]\to X\Big)=X[1]\oplus Y\oplus X\quad \forall n\in \mathbb Z.
    \end{align*}
    其 $n$-次微分为
    \begin{align*}
        \begin{pmatrix}
            -d_X^{n+1}                             \\
            u^{n+1}                & d_Y^n         \\
            -\mathrm{id}_{X^{n+1}} &       & d_X^n
        \end{pmatrix}:
        X^{n+1}\oplus Y^n\oplus X^n\longrightarrow X^{n+2}\oplus Y^{n+1}\oplus X^{n+1}.
    \end{align*}
\end{definition}

\begin{example}
    映射 $X\overset u\longrightarrow Y$ 诱导了如下 $K(\mathcal A)$ 中正合列, 其中任意三角为好三角.
    % https://q.uiver.app/#q=WzAsMTAsWzIsMCwiWCJdLFs0LDAsIlxcbWF0aHJte2NvbmV9KHUpIl0sWzUsMCwiWFsxXSJdLFsxLDAsIlxcbWF0aHJte2NvbmV9KHUpWy0xXSJdLFs2LDAsIlxcY2RvdHMiXSxbMywwLCJcXG1hdGhybXtjeWx9KHUpIl0sWzEsMSwiWFxcb3BsdXMgWVstMV0iXSxbMywxLCJYWzFdXFxvcGx1cyBZXFxvcGx1cyBYIl0sWzQsMSwiWFsxXVxcb3BsdXMgWSJdLFswLDAsIlxcY2RvdHMiXSxbMSwyLCIoMSwwKSJdLFsyLDRdLFs1LDEsIlxcYmlub217MVxcLFxcLDBcXCxcXCwwfXswXFwsXFwsMVxcLFxcLDB9Il0sWzAsNSwiXFxsZWZ0KFxcc3Vic3RhY2t7MFxcXFwwXFxcXDF9XFxyaWdodCkiXSxbMywwLCIoLTEsMCkiXSxbNiwzLCIiLDAseyJsZXZlbCI6Miwic3R5bGUiOnsiaGVhZCI6eyJuYW1lIjoibm9uZSJ9fX1dLFs3LDUsIiIsMCx7ImxldmVsIjoyLCJzdHlsZSI6eyJoZWFkIjp7Im5hbWUiOiJub25lIn19fV0sWzgsMSwiIiwwLHsibGV2ZWwiOjIsInN0eWxlIjp7ImhlYWQiOnsibmFtZSI6Im5vbmUifX19XSxbOSwzXV0=
    \[\begin{tikzcd}
            \cdots & {\mathrm{cone}(u)[-1]} & X & {\mathrm{cyl}(u)} & {\mathrm{cone}(u)} & {X[1]} & \cdots \\
            & {X\oplus Y[-1]} && {X[1]\oplus Y\oplus X} & {X[1]\oplus Y}
            \arrow["{(1,0)}", from=1-5, to=1-6]
            \arrow[from=1-6, to=1-7]
            \arrow["{\binom{1\,\,0\,\,0}{0\,\,1\,\,0}}", from=1-4, to=1-5]
            \arrow["{\left(\substack{0\\0\\1}\right)}", from=1-3, to=1-4]
            \arrow["{(-1,0)}", from=1-2, to=1-3]
            \arrow[Rightarrow, no head, from=2-2, to=1-2]
            \arrow[Rightarrow, no head, from=2-4, to=1-4]
            \arrow[Rightarrow, no head, from=2-5, to=1-5]
            \arrow[from=1-1, to=1-2]
        \end{tikzcd}.\]
\end{example}

\begin{proposition}
    映射筒给出如下可裂短正合列
    % https://q.uiver.app/#q=WzAsNSxbMSwwLCJYIl0sWzMsMCwiXFxtYXRocm17Y29uZX0odSkiXSxbMiwwLCJcXG1hdGhybXtjeWx9KHUpIl0sWzAsMCwiMCJdLFs0LDAsIjAiXSxbMiwxLCJcXGJpbm9tezFcXCxcXCwwXFwsXFwsMH17MFxcLFxcLDFcXCxcXCwwfSJdLFswLDIsIlxcbGVmdChcXHN1YnN0YWNrezBcXFxcMFxcXFwxfVxccmlnaHQpIl0sWzMsMF0sWzEsNF1d
    \[\begin{tikzcd}
            0 & X & {\mathrm{cyl}(u)} & {\mathrm{cone}(u)} & 0
            \arrow["{\binom{1\,\,0\,\,0}{0\,\,1\,\,0}}", from=1-3, to=1-4]
            \arrow["{\left(\substack{0\\0\\1}\right)}", from=1-2, to=1-3]
            \arrow[from=1-1, to=1-2]
            \arrow[from=1-4, to=1-5]
        \end{tikzcd}.\]
\end{proposition}

\begin{proposition}
    $(0,1,u):\mathrm{cyl}(u)\longrightarrow Y$ 是同伦等价.
    \begin{proof}
        依照态射在好三角中嵌入的唯一性, 作同伦范畴的交换图
        % https://q.uiver.app/#q=WzAsMTIsWzAsMSwiWCJdLFsyLDEsIlxcbWF0aHJte2NvbmV9KHUpIl0sWzEsMSwiXFxtYXRocm17Y3lsfSh1KSJdLFszLDEsIlhbMV0iXSxbMCwyLCJYIl0sWzIsMiwiXFxtYXRocm17Y29uZX0odSkiXSxbMywyLCJYWzFdIl0sWzEsMiwiWSJdLFswLDAsIlgiXSxbMSwwLCJZIl0sWzIsMCwiXFxtYXRocm17Y29uZX0odSkiXSxbMywwLCJYWzFdIl0sWzIsMSwiXFxiaW5vbXsxXFwsXFwsMFxcLFxcLDB9ezBcXCxcXCwxXFwsXFwsMH0iXSxbMSwzLCIoMSwwKSJdLFs1LDYsIigxLDApIiwyXSxbMCw0LCIiLDIseyJsZXZlbCI6Miwic3R5bGUiOnsiaGVhZCI6eyJuYW1lIjoibm9uZSJ9fX1dLFszLDYsIiIsMCx7ImxldmVsIjoyLCJzdHlsZSI6eyJoZWFkIjp7Im5hbWUiOiJub25lIn19fV0sWzQsNywidSIsMl0sWzcsNSwiXFxiaW5vbTAxIiwyXSxbMSw1LCIiLDEseyJsZXZlbCI6Miwic3R5bGUiOnsiaGVhZCI6eyJuYW1lIjoibm9uZSJ9fX1dLFsyLDcsIigwLDEsdSkiLDAseyJzdHlsZSI6eyJib2R5Ijp7Im5hbWUiOiJkYXNoZWQifX19XSxbMCwyLCJcXGxlZnQoXFxzdWJzdGFja3swXFxcXDBcXFxcMX1cXHJpZ2h0KSIsMl0sWzgsOSwidSJdLFs5LDEwLCJcXGJpbm9tMDEiXSxbMTAsMTEsIigxLDApIl0sWzgsMCwiIiwwLHsibGV2ZWwiOjIsInN0eWxlIjp7ImhlYWQiOnsibmFtZSI6Im5vbmUifX19XSxbOSwyLCJcXGxlZnQoXFxzdWJzdGFja3swXFxcXDFcXFxcMH1cXHJpZ2h0KSIsMix7InN0eWxlIjp7ImJvZHkiOnsibmFtZSI6ImRhc2hlZCJ9fX1dLFsxMCwxLCIiLDAseyJsZXZlbCI6Miwic3R5bGUiOnsiaGVhZCI6eyJuYW1lIjoibm9uZSJ9fX1dLFsxMSwzLCIiLDAseyJsZXZlbCI6Miwic3R5bGUiOnsiaGVhZCI6eyJuYW1lIjoibm9uZSJ9fX1dXQ==
        \[\begin{tikzcd}
                X & Y & {\mathrm{cone}(u)} & {X[1]} \\
                X & {\mathrm{cyl}(u)} & {\mathrm{cone}(u)} & {X[1]} \\
                X & Y & {\mathrm{cone}(u)} & {X[1]}
                \arrow["{\binom{1\,\,0\,\,0}{0\,\,1\,\,0}}", from=2-2, to=2-3]
                \arrow["{(1,0)}", from=2-3, to=2-4]
                \arrow["{(1,0)}"', from=3-3, to=3-4]
                \arrow[Rightarrow, no head, from=2-1, to=3-1]
                \arrow[Rightarrow, no head, from=2-4, to=3-4]
                \arrow["u"', from=3-1, to=3-2]
                \arrow["\binom01"', from=3-2, to=3-3]
                \arrow[Rightarrow, no head, from=2-3, to=3-3]
                \arrow["{(0,1,u)}", dashed, from=2-2, to=3-2]
                \arrow["{\left(\substack{0\\0\\1}\right)}"', from=2-1, to=2-2]
                \arrow["u", from=1-1, to=1-2]
                \arrow["\binom01", from=1-2, to=1-3]
                \arrow["{(1,0)}", from=1-3, to=1-4]
                \arrow[Rightarrow, no head, from=1-1, to=2-1]
                \arrow["{\left(\substack{0\\1\\0}\right)}"', dashed, from=1-2, to=2-2]
                \arrow[Rightarrow, no head, from=1-3, to=2-3]
                \arrow[Rightarrow, no head, from=1-4, to=2-4]
            \end{tikzcd}.\]
        显然第一行第二列方块与第二行第一列方块交换, 且上图第二列复合为 $\mathrm{id}_Y$. 依照``二推三''之同构定理, $\begin{pmatrix}0\\1\\0\end{pmatrix}(0,1,u)=\begin{pmatrix}0&0&0\\0&1&u\\0&0&0\end{pmatrix}$ 与 $\mathrm{id}_{\mathrm{cyl}(u)}$ 同伦. 下仅需证明
        \begin{enumerate}
            \item 第一行第一列处方块交换, 即, $\begin{pmatrix}0\\u\\-1\end{pmatrix}:X\to \mathrm{cyl}(u)$ 零伦;
            \item 第二行第二列处方块交换, 即, $\begin{pmatrix}1\\&&-u\end{pmatrix}:\mathrm{cyl}(u)\to \mathrm{cone}(u)$ 零伦;
        \end{enumerate}
        对前者, 考虑
        \begin{align*}
            \begin{pmatrix}
                0 \\
                u \\
                -1
            \end{pmatrix} =\begin{pmatrix}
                -d &   &   \\
                u  & d &   \\
                -1 &   & d
            \end{pmatrix}\begin{pmatrix}
                1 \\
                0 \\0
            \end{pmatrix} +\begin{pmatrix}
                1 \\
                0 \\0
            \end{pmatrix}\begin{pmatrix}
                d
            \end{pmatrix}.
        \end{align*}
        对后者, 考虑
        \begin{align*}
            \begin{pmatrix}
                1 & 0 & 0  \\
                0 & 0 & -u
            \end{pmatrix} =\begin{pmatrix}
                0 & 0 & -1 \\
                0 & 0 & 0
            \end{pmatrix}\begin{pmatrix}
                -d &   &   \\
                u  & d &   \\
                -1 &   & d
            \end{pmatrix} +\begin{pmatrix}
                -d &   \\
                u  & d
            \end{pmatrix}\begin{pmatrix}
                0 & 0 & -1 \\
                0 & 0 & 0
            \end{pmatrix}.
        \end{align*}
    \end{proof}
\end{proposition}

\begin{remark}
    对 Abel 范畴中复形的短正合列 $\begin{tikzcd}
            0 & X & Y & Z & 0
            \arrow[from=1-1, to=1-2]
            \arrow["u", from=1-2, to=1-3]
            \arrow["\pi", from=1-3, to=1-4]
            \arrow[from=1-4, to=1-5]
        \end{tikzcd}$, 有交换图
    % https://q.uiver.app/#q=WzAsMTAsWzAsMCwiMCJdLFsxLDAsIlgiXSxbMiwwLCJcXG1hdGhybXtjeWx9KHUpIl0sWzMsMCwiXFxtYXRocm17Y29uZX0odSkiXSxbNCwwLCIwIl0sWzEsMSwiWCJdLFsyLDEsIlkiXSxbMywxLCJaIl0sWzQsMSwiMCJdLFswLDEsIjAiXSxbMCwxXSxbMSwyLCJcXGxlZnQoXFxzdWJzdGFja3swXFxcXDBcXFxcMX1cXHJpZ2h0KSJdLFsyLDMsIlxcYmlub217MVxcLFxcLDBcXCxcXCwwfXswXFwsXFwsMVxcLFxcLDB9Il0sWzMsNF0sWzksNV0sWzUsNiwidSIsMl0sWzcsOF0sWzEsNSwiIiwxLHsibGV2ZWwiOjIsInN0eWxlIjp7ImhlYWQiOnsibmFtZSI6Im5vbmUifX19XSxbMiw2LCIoMCwxLHUpIl0sWzMsNywiKDAsdikiXSxbNiw3LCJcXHBpIiwyXV0=
    \[\begin{tikzcd}
            0 & X & {\mathrm{cyl}(u)} & {\mathrm{cone}(u)} & 0 \\
            0 & X & Y & Z & 0
            \arrow[from=1-1, to=1-2]
            \arrow["{\left(\substack{0\\0\\1}\right)}", from=1-2, to=1-3]
            \arrow["{\binom{1\,\,0\,\,0}{0\,\,1\,\,0}}", from=1-3, to=1-4]
            \arrow[from=1-4, to=1-5]
            \arrow[from=2-1, to=2-2]
            \arrow["u"', from=2-2, to=2-3]
            \arrow[from=2-4, to=2-5]
            \arrow[Rightarrow, no head, from=1-2, to=2-2]
            \arrow["{(0,1,u)}", from=1-3, to=2-3]
            \arrow["{(0,v)}", from=1-4, to=2-4]
            \arrow["\pi"', from=2-3, to=2-4]
        \end{tikzcd}.\]
    此时, 蛇引理给出长正合列的交换图. 注意到 $(0,1,u)$ 是同伦等价, 从而是拟同构, 因此长正合列同构.此时 $(0,v):\mathrm{cone}(u)\longrightarrow Z$ 为拟同构.
\end{remark}

\begin{definition}[同伦像]
    映射筒即 $\mathrm{cyl}(u):=\mathrm{cone}(\mathrm{hker}(u))$; 相应地, 定义同伦像 $\mathrm{him}(v):=\mathrm{cone}(\mathrm{hcoker}(v))[-1]$. 具体地, 给定 $Y\overset v\longrightarrow Z$, 则 $\mathrm{him}(v)=Z\oplus Y\oplus Z[-1]$ 的 $n$-次上同调为
    \begin{align*}
        \begin{pmatrix}
            d_{Z}^{n}            &           &              \\
                                 & d_{Y}^{n} &              \\
            -\mathrm{id}_{Z^{n}} & -v^{n}    & -d_{Y}^{n-1}
        \end{pmatrix} :Z^{n} \oplus Y^{n} \oplus Z^{n-1} \longrightarrow Z^{n+1} \oplus Y^{n+1} \oplus Z^{n}.
    \end{align*}
\end{definition}

\begin{proposition}[$\mathrm{him}(v)$ 的对偶命题]
    类似映射筒, 有如下关于同伦像的相似命题
    \begin{enumerate}
        \item 有好三角 $\begin{tikzcd}
                      {Z[-1]} & {\mathrm{cone}(v)[-1]} & {\mathrm{him}(v)} & Z
                      \arrow["{\binom0{-1}}", from=1-1, to=1-2]
                      \arrow["{\left(\substack{\,\,\,\,0\,\,\,\,\,\,0\\-1\,\,\,\,\,\,0\\\,\,\,\,0\,\,-1}\right)}", from=1-2, to=1-3]
                      \arrow["{(-1,0,0)}", from=1-3, to=1-4]
                  \end{tikzcd}$.
        \item[1'] 将上一条中的好三角顺时针旋转, 并将前两项反号, 得 $\begin{tikzcd}
                      {\mathrm{cone}(v)[-1]} & {\mathrm{him}(v)} & Z & {\mathrm{cone}(v)}
                      \arrow["{\left(\substack{0\,\,0\\1\,\,0\\0\,\,1}\right)}", from=1-1, to=1-2]
                      \arrow["{(1,0,0)}", from=1-2, to=1-3]
                      \arrow["\binom01", from=1-3, to=1-4]
                  \end{tikzcd}$.
        \item 有可裂短正合列 $\begin{tikzcd}
                      0 & {\mathrm{cone}(v)[-1]} & {\mathrm{him}(v)} & Z & 0
                      \arrow["{\left(\substack{0\,\,0\\1\,\,0\\0\,\,1}\right)}", from=1-2, to=1-3]
                      \arrow["{(1,0,0)}", from=1-3, to=1-4]
                      \arrow[from=1-4, to=1-5]
                      \arrow[from=1-1, to=1-2]
                  \end{tikzcd}$
        \item 有同伦等价的交换图
              % https://q.uiver.app/#q=WzAsMTIsWzAsMCwiXFxtYXRocm17Y29uZX0odilbLTFdIl0sWzEsMCwiXFxtYXRocm17aGltfSh2KSJdLFsyLDAsIloiXSxbMywwLCJcXG1hdGhybXtjb25lfSh2KSJdLFswLDEsIlxcbWF0aHJte2NvbmV9KHYpWy0xXSJdLFsxLDEsIlkiXSxbMiwxLCJaIl0sWzMsMSwiXFxtYXRocm17Y29uZX0odikiXSxbMCwyLCJcXG1hdGhybXtjb25lfSh2KVstMV0iXSxbMiwyLCJaIl0sWzMsMiwiXFxtYXRocm17Y29uZX0odikiXSxbMSwyLCJcXG1hdGhybXtoaW19KHYpIl0sWzAsMSwiXFxsZWZ0KFxcc3Vic3RhY2t7MFxcLFxcLDBcXFxcMVxcLFxcLDBcXFxcMFxcLFxcLDF9XFxyaWdodCkiXSxbMSwyLCIoMSwwLDApIl0sWzIsMywiXFxiaW5vbTAxIl0sWzMsNywiIiwwLHsibGV2ZWwiOjIsInN0eWxlIjp7ImhlYWQiOnsibmFtZSI6Im5vbmUifX19XSxbMiw2LCIiLDAseyJsZXZlbCI6Miwic3R5bGUiOnsiaGVhZCI6eyJuYW1lIjoibm9uZSJ9fX1dLFswLDQsIiIsMix7ImxldmVsIjoyLCJzdHlsZSI6eyJoZWFkIjp7Im5hbWUiOiJub25lIn19fV0sWzQsNSwiKC0xLDApIl0sWzUsNiwidiIsMl0sWzEsNSwiKDAsLTEsMCkiXSxbNiw3LCJcXGJpbm9tMDEiLDJdLFs4LDExLCJcXGxlZnQoXFxzdWJzdGFja3swXFwsXFwsMFxcXFwxXFwsXFwsMFxcXFwwXFwsXFwsMX1cXHJpZ2h0KSIsMl0sWzExLDksIigxLDAsMCkiLDJdLFs5LDEwLCJcXGJpbm9tMDEiLDJdLFs3LDEwLCIiLDIseyJsZXZlbCI6Miwic3R5bGUiOnsiaGVhZCI6eyJuYW1lIjoibm9uZSJ9fX1dLFs2LDksIiIsMix7ImxldmVsIjoyLCJzdHlsZSI6eyJoZWFkIjp7Im5hbWUiOiJub25lIn19fV0sWzQsOCwiIiwyLHsibGV2ZWwiOjIsInN0eWxlIjp7ImhlYWQiOnsibmFtZSI6Im5vbmUifX19XSxbNSwxMSwiXFxsZWZ0KFxcc3Vic3RhY2t7dlxcXFwtMVxcXFwwfVxccmlnaHQpIiwyXV0=
              \[\begin{tikzcd}
                      {\mathrm{cone}(v)[-1]} & {\mathrm{him}(v)} & Z & {\mathrm{cone}(v)} \\
                      {\mathrm{cone}(v)[-1]} & Y & Z & {\mathrm{cone}(v)} \\
                      {\mathrm{cone}(v)[-1]} & {\mathrm{him}(v)} & Z & {\mathrm{cone}(v)}
                      \arrow["{\left(\substack{0\,\,0\\1\,\,0\\0\,\,1}\right)}", from=1-1, to=1-2]
                      \arrow["{(1,0,0)}", from=1-2, to=1-3]
                      \arrow["\binom01", from=1-3, to=1-4]
                      \arrow[Rightarrow, no head, from=1-4, to=2-4]
                      \arrow[Rightarrow, no head, from=1-3, to=2-3]
                      \arrow[Rightarrow, no head, from=1-1, to=2-1]
                      \arrow["{(-1,0)}", from=2-1, to=2-2]
                      \arrow["v"', from=2-2, to=2-3]
                      \arrow["{(0,-1,0)}", from=1-2, to=2-2]
                      \arrow["\binom01"', from=2-3, to=2-4]
                      \arrow["{\left(\substack{0\,\,0\\1\,\,0\\0\,\,1}\right)}"', from=3-1, to=3-2]
                      \arrow["{(1,0,0)}"', from=3-2, to=3-3]
                      \arrow["\binom01"', from=3-3, to=3-4]
                      \arrow[Rightarrow, no head, from=2-4, to=3-4]
                      \arrow[Rightarrow, no head, from=2-3, to=3-3]
                      \arrow[Rightarrow, no head, from=2-1, to=3-1]
                      \arrow["{\left(\substack{v\\-1\\0}\right)}"', from=2-2, to=3-2]
                  \end{tikzcd}.\]
        \item
    \end{enumerate}
\end{proposition}

\end{document}